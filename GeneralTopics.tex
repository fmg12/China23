%%%%%%%%%%%%%%%%%%%%%%%%%%%%%%%%%%%%%%%%%%%%%%%%%%%%%%%%%%%%%%%%%%%%%
%\section{Novel electronic states}

%%%%%%%%%%%%%%%%%%%%%%%%%%%%%%%%%%%%%%%%%%%%%%%%%%%%%%%%%%%%%%%%%%%%%
\subsection{Tunability of correlated electron systems}
%%%%%%%%%%%%%%%%%%%%%%%%%%%%%%%%%%%%%%%%%%%%%%%%%%%%%%%%%%%%%%%%%%%%%%
\begin{frame}[label=ElecLiquid]
\frametitle{Tunability of correlated systems}

\centerline{\multiinclude[<visible@+-| +->][format=pdf,graphics={width=0.9\columnwidth}]{\Figures/FreeElec/ElecLiquid}}
\vspace{2ex}

\visible<3-> {\centerline{\parbox{0.9\columnwidth}{High quality crystals
      $\rightarrow$ tune to border of low temperature order
      $\rightarrow$ anomalous metallic state or new ordered phase.}}}


\end{frame}

%%%%%%%%%%%%%%%%%%%%%%%%%%%%%%%%%%%%%%%%%%%%%%%%%%%%%%%%%%%%%%%%%%%%%%
%\subsection{Superconductivity and magnetism}
%%%%%%%%%%%%%%%%%%%%%%%%%%%%%%%%%%%%%%%%%%%%%%%%%%%%%%%%%%%%%%%%%%%%%%

% %%%%%%%%%%%%%%%%%%%%%%%%%%%%%%%%%%%%%%%%%%%%%%%%%%%%%%%%%%%%%%%%%%%%%%
% \subsection{CePd$_2$Si$_2$}
%%%%%%%%%%%%%%%%%%%%%%%%%%%%%%%%%%%%%%%%%%%%%%%%%%%%%%%%%%%%%%%%%%%%%%
\begin{frame}[label=CPS]
\frametitle{CePd$_2$Si$_2$: heavy-fermion magnet to unconventional superconductor}
\begin{columns}[t]
  \column{0.37\textwidth}
  \begin{itemize}
  \item<1-> Antiferro\-magnet below $T_N\simeq 10 ~\mathrm K$.

  \item<1-> $T_N$ depends on pressure.

  \item<1->  Magnetism suppressed near 2.8 GPa.

  \item<1-> Anomalous resistivity $T$-dependence.
  \end{itemize}

\vspace{1em}
\centerline{\scriptsize [Mathur {\em et al.}, Nature {\bf 394,} (1998) 39]}
  \column{0.63\textwidth}
\vspace{0em}
  \centerline{\multiinclude[<visible@1-|1->][graphics={width=0.95\columnwidth},format=pdf]{\Figures/cps/cpsphasesnormal}}
 
\end{columns}
%\onslide+<1->
\begin{center}
\hl{Superconductivity and anomalous normal state.}
\end{center}
\end{frame}


%%%%%%%%%%%%%%%%%%%%%%%%%%%%%%%%%%%%%%%%%%%%%%%%%%%%%%%%%%%%%%%%%%%%%%
\begin{frame}[label=ThreshMagn]
\frametitle{Magnetic interaction model}

\centerline{\includegraphics[width=\textwidth]{\Figures/CritConcepts/MagnInter2}}

\end{frame}



%%%%%%%%%%%%%%%%%%%%%%%%%%%%%%%%%%%%%%%%%%%%%%%%%%%%%%%%%%%%%%%%%%%%%%
\begin{frame}[label=CPS]
\frametitle{From narrow band, $f$-electron, low-$T_c$ to
  $d$-electron high-$T_c$}
\centerline{\includegraphics[width=0.9\textwidth]{\Figures/Phasedias/CePd2Si2/cpsbfaComparison}}

\begin{columns}[t]
\column{0.5\textwidth}
\centerline{\scriptsize \hspace{5em} [Mathur {\em et al.}, Nature {\bf 394,} 39  (1998)]}

\column{0.5\textwidth}
\centerline{\scriptsize [from Hashimoto Science {\bf 336,} 1554 (2012)]}
\end{columns}
\end{frame}

%%%%%%%%%%%%%%%%%%%%%%%%%%%%%%%%%%%%%%%%%%%%%%%%%%%%%%%%%%%%%%%%%%%%%%
\begin{frame}[label=phasedias]
%\frametitle{Phase diagrams in correlated quantum systems}
%\frametitle{Examine Fermi surface evolution near quantum phase transitions}
%\vspace{-3ex}
\centerline{%
\includegraphics[width=\columnwidth,clip=on]{\Figures/Phasedias/phasedias10}}
\end{frame}



%%%%%%%%%%%%%%%%%%%%%%%%%%%%%%%%%%%%%%%%%%%%%%%%%%%%%%%%%%%%%%%%%%%%%
\subsection{Material space}
%%%%%%%%%%%%%%%%%%%%%%%%%%%%%%%%%%%%%%%%%%%%%%%%%%%%%%%%%%%%%%%%%%%%%

\begin{frame}[label=Diversity]

\frametitle{Plenty of room in material space}
\only<beamer>{%
\only<1>{%
\vspace{2em}
\centerline{\includegraphics[angle=90,width=0.9\columnwidth]{\Figures/Diversity/PerSysen.jpg}}
}
}
\pause                          
%\centerline{\multiinclude[<visible@1->][graphics={width=0.9\columnwidth},format=pdf]{\Figures/Diversity/Diversity}}
\centerline{\multiinclude[<visible@+-| +->][graphics={width=0.9\columnwidth},format=pdf]{\Figures/Diversity/Diversity}}

\end{frame}



% %%%%%%%%%%%%%%%%%%%%%%%%%%%%%%%%%%%%%%%%%%%%%%%%%%%%%%%%%%%%%%%%%%%%%%
% \begin{frame}[label=Funding]
% \frametitle{Networks, fellowships and funding}

% \begin{itemize}
% \item \hl{International networks}: ICAM, 
% \begin{itemize}
% \item
% ICAM

% \end{frame}

% %%%%%%%%%%%%%%%%%%%%%%%%%%%%%%%%%%%%%%%%%%%%%%%%%%%%%%%%%%%%%%%%%%%%%%
% \subsection{Fermi surface instabilities}
% %%%%%%%%%%%%%%%%%%%%%%%%%%%%%%%%%%%%%%%%%%%%%%%%%%%%%%%%%%%%%%%%%%%%%%
% \begin{frame}[label=FermiSurface]
%   \frametitle{Fermi surface: surface in momentum space on which low energy excitations are possible}
% \only<1>{
% \vspace{5em}
% \centerline{\includegraphics[width=\columnwidth]{\Figures/Lectures/ElecGas/FermiSea.jpg}}
% %\centerline{\small A. Schofield (1998)}
% }

% \begin{columns}[t]
% \column{0.5\textwidth}
% \visible<2->{
% \centerline{\includegraphics[height=0.98\columnwidth]{\Figures/FreeElec/copper.jpg}}
% \centerline{Copper}}

% \column{0.5\textwidth}
% \visible<3->{
% \centerline{\includegraphics[height=0.98\columnwidth]{\Figures/NbFe2Overview/band84}}
% \centerline{NbFe$_2$}}
% \end{columns}

% \end{frame}







%%% Local Variables: 
%%% mode: latex
%%% TeX-master: "GroTalk"
%%% End: 
