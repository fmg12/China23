\section{3-4-13 materials}
%%%%%%%%%%%%%%%%%%%%%%%%%%%%%%%%%%%%%%%%%%%%%%%%%%%%%%%%%%%%%%%%%%%%%%
\begin{frame}[label=CIS-1]
\frametitle{Superlattice formation in (Sr/Ca)$_3$Ir$_4$Sn$_{13}$}

\begin{columns}[t]
\column{0.7\textwidth}
\centerline{~}
\centerline{\includegraphics[width=0.8\columnwidth]{\Figures/R3T4X13/Figure1_SQCP.pdf}}
\column{0.3\textwidth}
\centerline{~}
\centerline{\includegraphics[width=\columnwidth]{\Figures/R3T4X13/Figure2a_SQCP.pdf}}
\centerline{\includegraphics[width=\columnwidth]{\Figures/R3T4X13/Figure2b_SQCP.pdf}}
\end{columns}

\begin{itemize}
% \item
% R = earth alkaline or rare earth, T = transition metal, X = group-4 (Ge, Sn).

\item
Two structure types, phase $I$ and $I'$. $I \rightarrow I'$ on cooling.

\item Both structures cubic, $I'$ lattice constant twice that of $I$.

\item $I'$= superlattice distortion on $I$. 

\end{itemize}

\end{frame}


%%%%%%%%%%%%%%%%%%%%%%%%%%%%%%%%%%%%%%%%%%%%%%%%%%%%%%%%%%%%%%%%%%%%%%
%\subsection{Phonon softening}
\begin{frame}[label=CIS-phonons]
\frametitle{Calculated phonon dispersion in (Sr/Ca)$_3$Ir$_4$Sn$_{13}$}
\centerline{\includegraphics[width=\textwidth]{\Figures/R3T4X13/CalcPhonons}}
\begin{itemize}
\item
Numerical study (VASP) reproduces lattice instability at ${\bf Q} =
(1/2 ~ 1/2 ~ 0)$.
\item
Anharmonic effects not included, push up frequency.
\item 
Simulated pressure removes instability.
\end{itemize}
\end{frame}


%%%%%%%%%%%%%%%%%%%%%%%%%%%%%%%%%%%%%%%%%%%%%%%%%%%%%%%%%%%%%%%%%%%%%%
\begin{frame}[label=CIS-4]
\frametitle{Ambient pressure quantum critical point in (Sr/Ca)$_3$Rh$_4$Sn$_{13}$}


\includegraphics[height=0.5\columnwidth]{\Figures/R3T4X13/R3Rh4Sn13-1}
\hfill
\includegraphics[height=0.53\columnwidth]{\Figures/R3T4X13/R3Rh4Sn13-2}

\centerline{\small [S. K. Goh et al. PRL {\bf 114,} 097002 (2015)]}
\begin{itemize}
\item
Rh smaller than Ir, causes further chemical pressure shift.

\item
Structural transition suppressed in Ca$_{0.9}$ Sr$_{0.1}$Rh$_4$Sn$_{13}$.

\end{itemize}
\end{frame}


%%%%%%%%%%%%%%%%%%%%%%%%%%%%%%%%%%%%%%%%%%%%%%%%%%%%%%%%%%%%%%%%%%%%%%
\subsection{Structure of 3-4-13 compounds}
%%%%%%%%%%%%%%%%%%%%%%%%%%%%%%%%%%%%%%%%%%%%%%%%%%%%%%%%%%%%%%%%%%%%%%
\begin{frame}[label=CIS-0]
\frametitle{Cubic quasiskutterudites R$_3$T$_4$X$_{13}$}

\centerline {\multiinclude[<visible@+-| +>][format=png,graphics={width=0.65\columnwidth}]{\Figures/Structures/R3T4X13/Ca3Ir4Sn13}}
%\centerline{\includegraphics[width=0.65\columnwidth]{\Figures/Structures/R3T4X13/Ca3Ir4Sn13-4}}
\begin{itemize}
\item %3-4-13 'Remeika phases'
 \only<1>{X$_1$ (e.g. Sn) atoms $\rightarrow$ bcc lattice. }
 \only<2>{X$_1$ and R (e.g. Sr, Ca) atoms $\rightarrow$ A15. }
 \only<3>{T (e.g. Ir) atoms $\rightarrow$ simple cubic lattice.}
 \only<4>{X$_2$ 12-cages around X$_1$, with T $\rightarrow$ filled
 skutterudite. }
\only<5->{Structure of R$_3$T$_4$X$_{13}$}

\item %\visible<5-> 
{Combination of
  filled skutterudite (LaRu$_4$P$_{12}$) and A15 structure (Nb$_3$Sn):
  (X'R$_3$) T$_4$X$_{12}$. }
\end{itemize}
\end{frame}

%%%%%%%%%%%%%%%%%%%%%%%%%%%%%%%%%%%%%%%%%%%%%%%%%%%%%%%%%%%%%%%%%%%%%%
\subsection{Phase diagram}
%%%%%%%%%%%%%%%%%%%%%%%%%%%%%%%%%%%%%%%%%%%%%%%%%%%%%%%%%%%%%%%%%%%%%%
\begin{frame}[label=CIS-3]
\frametitle{Structural quantum critical point in (Sr/Ca)$_3$Ir$_4$Sn$_{13}$}
\centerline{\includegraphics[width=0.6\columnwidth]{\Figures/R3T4X13/Figure5_SQCP.pdf}}

\begin{itemize}
% \item
% Combined alloying (chemical pressure) and hydrostatic pressure phase diagram.
\item
Structural transition suppressed at $p_c = 1.8 ~{\rm GPa}$ in 
Ca$_3$Ir$_4$Sn$_{13}$. Composition-tuned qcp in (Sr/Ca)$_3$Rh$_4$Sn$_{13}$.

\item
%Superconducting $T_c$ has broad dome-structure. Associated with 
Softening of optical phonon branch around $p_c$.
\end{itemize}

\vspace*{\fill}
%\vspace{3.5em}
\centerline{\makebox[\linewidth]{\rule{0.85\textwidth}{0.4pt}}}
\centerline{\scriptsize L. Klintberg PRL {\bf 109,} 237008 (2012), S. Goh  PRL {\bf 114,}  097002 (2015)}
\end{frame}


%\section{3-4-13 materials}
%%%%%%%%%%%%%%%%%%%%%%%%%%%%%%%%%%%%%%%%%%%%%%%%%%%%%%%%%%%%%%%%%%%%%%
\begin{frame}[label=CIS-1]
\frametitle{Superlattice formation in (Sr/Ca)$_3$Ir$_4$Sn$_{13}$}

\begin{columns}[t]
\column{0.7\textwidth}
\centerline{~}
\centerline{\includegraphics[width=0.8\columnwidth]{\Figures/R3T4X13/Figure1_SQCP.pdf}}
\column{0.3\textwidth}
\centerline{~}
\centerline{\includegraphics[width=\columnwidth]{\Figures/R3T4X13/Figure2a_SQCP.pdf}}
\centerline{\includegraphics[width=\columnwidth]{\Figures/R3T4X13/Figure2b_SQCP.pdf}}
\end{columns}

\begin{itemize}
% \item
% R = earth alkaline or rare earth, T = transition metal, X = group-4 (Ge, Sn).

\item
Two structure types, phase $I$ and $I'$. $I \rightarrow I'$ on cooling.

\item Both structures cubic, $I'$ lattice constant twice that of $I$.

\item $I'$= superlattice distortion on $I$. 

\end{itemize}

\end{frame}


%%%%%%%%%%%%%%%%%%%%%%%%%%%%%%%%%%%%%%%%%%%%%%%%%%%%%%%%%%%%%%%%%%%%%%
%\subsection{Phonon softening}
\begin{frame}[label=CIS-phonons]
\frametitle{Calculated phonon dispersion in (Sr/Ca)$_3$Ir$_4$Sn$_{13}$}
\centerline{\includegraphics[width=\textwidth]{\Figures/R3T4X13/CalcPhonons}}
\begin{itemize}
\item
Numerical study (VASP) reproduces lattice instability at ${\bf Q} =
(1/2 ~ 1/2 ~ 0)$.
\item
Anharmonic effects not included, push up frequency.
\item 
Simulated pressure removes instability.
\end{itemize}
\end{frame}


%%%%%%%%%%%%%%%%%%%%%%%%%%%%%%%%%%%%%%%%%%%%%%%%%%%%%%%%%%%%%%%%%%%%%%
\begin{frame}[label=CIS-4]
\frametitle{Ambient pressure quantum critical point in (Sr/Ca)$_3$Rh$_4$Sn$_{13}$}


\includegraphics[height=0.5\columnwidth]{\Figures/R3T4X13/R3Rh4Sn13-1}
\hfill
\includegraphics[height=0.53\columnwidth]{\Figures/R3T4X13/R3Rh4Sn13-2}

\centerline{\small [S. K. Goh et al. PRL {\bf 114,} 097002 (2015)]}
\begin{itemize}
\item
Rh smaller than Ir, causes further chemical pressure shift.

\item
Structural transition suppressed in Ca$_{0.9}$ Sr$_{0.1}$Rh$_4$Sn$_{13}$.

\end{itemize}
\end{frame}


%%%%%%%%%%%%%%%%%%%%%%%%%%%%%%%%%%%%%%%%%%%%%%%%%%%%%%%%%%%%%%%%%%%%%%
\subsection{Structure of 3-4-13 compounds}
%%%%%%%%%%%%%%%%%%%%%%%%%%%%%%%%%%%%%%%%%%%%%%%%%%%%%%%%%%%%%%%%%%%%%%
\begin{frame}[label=CIS-0]
\frametitle{Cubic quasiskutterudites R$_3$T$_4$X$_{13}$}

\centerline {\multiinclude[<visible@+-| +>][format=png,graphics={width=0.65\columnwidth}]{\Figures/Structures/R3T4X13/Ca3Ir4Sn13}}
%\centerline{\includegraphics[width=0.65\columnwidth]{\Figures/Structures/R3T4X13/Ca3Ir4Sn13-4}}
\begin{itemize}
\item %3-4-13 'Remeika phases'
 \only<1>{X$_1$ (e.g. Sn) atoms $\rightarrow$ bcc lattice. }
 \only<2>{X$_1$ and R (e.g. Sr, Ca) atoms $\rightarrow$ A15. }
 \only<3>{T (e.g. Ir) atoms $\rightarrow$ simple cubic lattice.}
 \only<4>{X$_2$ 12-cages around X$_1$, with T $\rightarrow$ filled
 skutterudite. }
\only<5->{Structure of R$_3$T$_4$X$_{13}$}

\item %\visible<5-> 
{Combination of
  filled skutterudite (LaRu$_4$P$_{12}$) and A15 structure (Nb$_3$Sn):
  (X'R$_3$) T$_4$X$_{12}$. }
\end{itemize}
\end{frame}

%%%%%%%%%%%%%%%%%%%%%%%%%%%%%%%%%%%%%%%%%%%%%%%%%%%%%%%%%%%%%%%%%%%%%%
\subsection{Phase diagram}
%%%%%%%%%%%%%%%%%%%%%%%%%%%%%%%%%%%%%%%%%%%%%%%%%%%%%%%%%%%%%%%%%%%%%%
\begin{frame}[label=CIS-3]
\frametitle{Structural quantum critical point in (Sr/Ca)$_3$Ir$_4$Sn$_{13}$}
\centerline{\includegraphics[width=0.6\columnwidth]{\Figures/R3T4X13/Figure5_SQCP.pdf}}

\begin{itemize}
% \item
% Combined alloying (chemical pressure) and hydrostatic pressure phase diagram.
\item
Structural transition suppressed at $p_c = 1.8 ~{\rm GPa}$ in 
Ca$_3$Ir$_4$Sn$_{13}$. Composition-tuned qcp in (Sr/Ca)$_3$Rh$_4$Sn$_{13}$.

\item
%Superconducting $T_c$ has broad dome-structure. Associated with 
Softening of optical phonon branch around $p_c$.
\end{itemize}

\vspace*{\fill}
%\vspace{3.5em}
\centerline{\makebox[\linewidth]{\rule{0.85\textwidth}{0.4pt}}}
\centerline{\scriptsize L. Klintberg PRL {\bf 109,} 237008 (2012), S. Goh  PRL {\bf 114,}  097002 (2015)}
\end{frame}


%\section{3-4-13 materials}
%%%%%%%%%%%%%%%%%%%%%%%%%%%%%%%%%%%%%%%%%%%%%%%%%%%%%%%%%%%%%%%%%%%%%%
\begin{frame}[label=CIS-1]
\frametitle{Superlattice formation in (Sr/Ca)$_3$Ir$_4$Sn$_{13}$}

\begin{columns}[t]
\column{0.7\textwidth}
\centerline{~}
\centerline{\includegraphics[width=0.8\columnwidth]{\Figures/R3T4X13/Figure1_SQCP.pdf}}
\column{0.3\textwidth}
\centerline{~}
\centerline{\includegraphics[width=\columnwidth]{\Figures/R3T4X13/Figure2a_SQCP.pdf}}
\centerline{\includegraphics[width=\columnwidth]{\Figures/R3T4X13/Figure2b_SQCP.pdf}}
\end{columns}

\begin{itemize}
% \item
% R = earth alkaline or rare earth, T = transition metal, X = group-4 (Ge, Sn).

\item
Two structure types, phase $I$ and $I'$. $I \rightarrow I'$ on cooling.

\item Both structures cubic, $I'$ lattice constant twice that of $I$.

\item $I'$= superlattice distortion on $I$. 

\end{itemize}

\end{frame}


%%%%%%%%%%%%%%%%%%%%%%%%%%%%%%%%%%%%%%%%%%%%%%%%%%%%%%%%%%%%%%%%%%%%%%
%\subsection{Phonon softening}
\begin{frame}[label=CIS-phonons]
\frametitle{Calculated phonon dispersion in (Sr/Ca)$_3$Ir$_4$Sn$_{13}$}
\centerline{\includegraphics[width=\textwidth]{\Figures/R3T4X13/CalcPhonons}}
\begin{itemize}
\item
Numerical study (VASP) reproduces lattice instability at ${\bf Q} =
(1/2 ~ 1/2 ~ 0)$.
\item
Anharmonic effects not included, push up frequency.
\item 
Simulated pressure removes instability.
\end{itemize}
\end{frame}


%%%%%%%%%%%%%%%%%%%%%%%%%%%%%%%%%%%%%%%%%%%%%%%%%%%%%%%%%%%%%%%%%%%%%%
\begin{frame}[label=CIS-4]
\frametitle{Ambient pressure quantum critical point in (Sr/Ca)$_3$Rh$_4$Sn$_{13}$}


\includegraphics[height=0.5\columnwidth]{\Figures/R3T4X13/R3Rh4Sn13-1}
\hfill
\includegraphics[height=0.53\columnwidth]{\Figures/R3T4X13/R3Rh4Sn13-2}

\centerline{\small [S. K. Goh et al. PRL {\bf 114,} 097002 (2015)]}
\begin{itemize}
\item
Rh smaller than Ir, causes further chemical pressure shift.

\item
Structural transition suppressed in Ca$_{0.9}$ Sr$_{0.1}$Rh$_4$Sn$_{13}$.

\end{itemize}
\end{frame}


%%%%%%%%%%%%%%%%%%%%%%%%%%%%%%%%%%%%%%%%%%%%%%%%%%%%%%%%%%%%%%%%%%%%%%
\subsection{Structure of 3-4-13 compounds}
%%%%%%%%%%%%%%%%%%%%%%%%%%%%%%%%%%%%%%%%%%%%%%%%%%%%%%%%%%%%%%%%%%%%%%
\begin{frame}[label=CIS-0]
\frametitle{Cubic quasiskutterudites R$_3$T$_4$X$_{13}$}

\centerline {\multiinclude[<visible@+-| +>][format=png,graphics={width=0.65\columnwidth}]{\Figures/Structures/R3T4X13/Ca3Ir4Sn13}}
%\centerline{\includegraphics[width=0.65\columnwidth]{\Figures/Structures/R3T4X13/Ca3Ir4Sn13-4}}
\begin{itemize}
\item %3-4-13 'Remeika phases'
 \only<1>{X$_1$ (e.g. Sn) atoms $\rightarrow$ bcc lattice. }
 \only<2>{X$_1$ and R (e.g. Sr, Ca) atoms $\rightarrow$ A15. }
 \only<3>{T (e.g. Ir) atoms $\rightarrow$ simple cubic lattice.}
 \only<4>{X$_2$ 12-cages around X$_1$, with T $\rightarrow$ filled
 skutterudite. }
\only<5->{Structure of R$_3$T$_4$X$_{13}$}

\item %\visible<5-> 
{Combination of
  filled skutterudite (LaRu$_4$P$_{12}$) and A15 structure (Nb$_3$Sn):
  (X'R$_3$) T$_4$X$_{12}$. }
\end{itemize}
\end{frame}

%%%%%%%%%%%%%%%%%%%%%%%%%%%%%%%%%%%%%%%%%%%%%%%%%%%%%%%%%%%%%%%%%%%%%%
\subsection{Phase diagram}
%%%%%%%%%%%%%%%%%%%%%%%%%%%%%%%%%%%%%%%%%%%%%%%%%%%%%%%%%%%%%%%%%%%%%%
\begin{frame}[label=CIS-3]
\frametitle{Structural quantum critical point in (Sr/Ca)$_3$Ir$_4$Sn$_{13}$}
\centerline{\includegraphics[width=0.6\columnwidth]{\Figures/R3T4X13/Figure5_SQCP.pdf}}

\begin{itemize}
% \item
% Combined alloying (chemical pressure) and hydrostatic pressure phase diagram.
\item
Structural transition suppressed at $p_c = 1.8 ~{\rm GPa}$ in 
Ca$_3$Ir$_4$Sn$_{13}$. Composition-tuned qcp in (Sr/Ca)$_3$Rh$_4$Sn$_{13}$.

\item
%Superconducting $T_c$ has broad dome-structure. Associated with 
Softening of optical phonon branch around $p_c$.
\end{itemize}

\vspace*{\fill}
%\vspace{3.5em}
\centerline{\makebox[\linewidth]{\rule{0.85\textwidth}{0.4pt}}}
\centerline{\scriptsize L. Klintberg PRL {\bf 109,} 237008 (2012), S. Goh  PRL {\bf 114,}  097002 (2015)}
\end{frame}


%\section{3-4-13 materials}
%%%%%%%%%%%%%%%%%%%%%%%%%%%%%%%%%%%%%%%%%%%%%%%%%%%%%%%%%%%%%%%%%%%%%%
\begin{frame}[label=CIS-1]
\frametitle{Superlattice formation in (Sr/Ca)$_3$Ir$_4$Sn$_{13}$}

\begin{columns}[t]
\column{0.7\textwidth}
\centerline{~}
\centerline{\includegraphics[width=0.8\columnwidth]{\Figures/R3T4X13/Figure1_SQCP.pdf}}
\column{0.3\textwidth}
\centerline{~}
\centerline{\includegraphics[width=\columnwidth]{\Figures/R3T4X13/Figure2a_SQCP.pdf}}
\centerline{\includegraphics[width=\columnwidth]{\Figures/R3T4X13/Figure2b_SQCP.pdf}}
\end{columns}

\begin{itemize}
% \item
% R = earth alkaline or rare earth, T = transition metal, X = group-4 (Ge, Sn).

\item
Two structure types, phase $I$ and $I'$. $I \rightarrow I'$ on cooling.

\item Both structures cubic, $I'$ lattice constant twice that of $I$.

\item $I'$= superlattice distortion on $I$. 

\end{itemize}

\end{frame}


%%%%%%%%%%%%%%%%%%%%%%%%%%%%%%%%%%%%%%%%%%%%%%%%%%%%%%%%%%%%%%%%%%%%%%
%\subsection{Phonon softening}
\begin{frame}[label=CIS-phonons]
\frametitle{Calculated phonon dispersion in (Sr/Ca)$_3$Ir$_4$Sn$_{13}$}
\centerline{\includegraphics[width=\textwidth]{\Figures/R3T4X13/CalcPhonons}}
\begin{itemize}
\item
Numerical study (VASP) reproduces lattice instability at ${\bf Q} =
(1/2 ~ 1/2 ~ 0)$.
\item
Anharmonic effects not included, push up frequency.
\item 
Simulated pressure removes instability.
\end{itemize}
\end{frame}


%%%%%%%%%%%%%%%%%%%%%%%%%%%%%%%%%%%%%%%%%%%%%%%%%%%%%%%%%%%%%%%%%%%%%%
\begin{frame}[label=CIS-4]
\frametitle{Ambient pressure quantum critical point in (Sr/Ca)$_3$Rh$_4$Sn$_{13}$}


\includegraphics[height=0.5\columnwidth]{\Figures/R3T4X13/R3Rh4Sn13-1}
\hfill
\includegraphics[height=0.53\columnwidth]{\Figures/R3T4X13/R3Rh4Sn13-2}

\centerline{\small [S. K. Goh et al. PRL {\bf 114,} 097002 (2015)]}
\begin{itemize}
\item
Rh smaller than Ir, causes further chemical pressure shift.

\item
Structural transition suppressed in Ca$_{0.9}$ Sr$_{0.1}$Rh$_4$Sn$_{13}$.

\end{itemize}
\end{frame}


%%%%%%%%%%%%%%%%%%%%%%%%%%%%%%%%%%%%%%%%%%%%%%%%%%%%%%%%%%%%%%%%%%%%%%
\subsection{Structure of 3-4-13 compounds}
%%%%%%%%%%%%%%%%%%%%%%%%%%%%%%%%%%%%%%%%%%%%%%%%%%%%%%%%%%%%%%%%%%%%%%
\begin{frame}[label=CIS-0]
\frametitle{Cubic quasiskutterudites R$_3$T$_4$X$_{13}$}

\centerline {\multiinclude[<visible@+-| +>][format=png,graphics={width=0.65\columnwidth}]{\Figures/Structures/R3T4X13/Ca3Ir4Sn13}}
%\centerline{\includegraphics[width=0.65\columnwidth]{\Figures/Structures/R3T4X13/Ca3Ir4Sn13-4}}
\begin{itemize}
\item %3-4-13 'Remeika phases'
 \only<1>{X$_1$ (e.g. Sn) atoms $\rightarrow$ bcc lattice. }
 \only<2>{X$_1$ and R (e.g. Sr, Ca) atoms $\rightarrow$ A15. }
 \only<3>{T (e.g. Ir) atoms $\rightarrow$ simple cubic lattice.}
 \only<4>{X$_2$ 12-cages around X$_1$, with T $\rightarrow$ filled
 skutterudite. }
\only<5->{Structure of R$_3$T$_4$X$_{13}$}

\item %\visible<5-> 
{Combination of
  filled skutterudite (LaRu$_4$P$_{12}$) and A15 structure (Nb$_3$Sn):
  (X'R$_3$) T$_4$X$_{12}$. }
\end{itemize}
\end{frame}

%%%%%%%%%%%%%%%%%%%%%%%%%%%%%%%%%%%%%%%%%%%%%%%%%%%%%%%%%%%%%%%%%%%%%%
\subsection{Phase diagram}
%%%%%%%%%%%%%%%%%%%%%%%%%%%%%%%%%%%%%%%%%%%%%%%%%%%%%%%%%%%%%%%%%%%%%%
\begin{frame}[label=CIS-3]
\frametitle{Structural quantum critical point in (Sr/Ca)$_3$Ir$_4$Sn$_{13}$}
\centerline{\includegraphics[width=0.6\columnwidth]{\Figures/R3T4X13/Figure5_SQCP.pdf}}

\begin{itemize}
% \item
% Combined alloying (chemical pressure) and hydrostatic pressure phase diagram.
\item
Structural transition suppressed at $p_c = 1.8 ~{\rm GPa}$ in 
Ca$_3$Ir$_4$Sn$_{13}$. Composition-tuned qcp in (Sr/Ca)$_3$Rh$_4$Sn$_{13}$.

\item
%Superconducting $T_c$ has broad dome-structure. Associated with 
Softening of optical phonon branch around $p_c$.
\end{itemize}

\vspace*{\fill}
%\vspace{3.5em}
\centerline{\makebox[\linewidth]{\rule{0.85\textwidth}{0.4pt}}}
\centerline{\scriptsize L. Klintberg PRL {\bf 109,} 237008 (2012), S. Goh  PRL {\bf 114,}  097002 (2015)}
\end{frame}


%\input{CISAppendix}
%%%%%%%%%%%%%%%%%%%%%%%%%%%%%%%%%%%%%%%%%%%%%%%%%%%%%%%%%%%%%%%%%%%%%%
\subsection{3-4-13 Related materials}
\begin{frame}[label=CIS-1]
\centerline{\includegraphics[width=1.27\textwidth]{\Figures/R3T4X13/PriorPapers}}
\end{frame}


\begin{frame}[label=CCSG]
\frametitle{Beyond
  magnetic quantum phase transitions}

\centerline{%
\multiinclude[<visible@+- | +->][graphics={width=0.7\columnwidth},format=pdf]{\Figures/ccs/CCSGPhaseDia1}
}
\visible<1->{
\centerline{\small[CeCu$_2$Si$_2$/CeCu$_2$Ge$_2$: Jaccard, e.g. Physica
  B (1999)]}}
\visible<2->{
\centerline{\small[CeCu$_2$(Si/Ge)$_2$: H. Q. Yuan, Science (2003)]}}


\vspace{-1.5ex}
\begin{center}
\begin{minipage}[t]{0.9\columnwidth}
\bi
\item<2->
Superconductivity at low $p$: \hl{magnetic} interaction
\item<2->
Superconductivity at high $p$: \hl{density (or valence) fluctuations?}
%% \item
%% Two supercond. domes \hl{merge} in CeCu$_2$Si$_2$.
\item<3->
Here, explore purely structural quantum phase transition.
\ei
\end{minipage}
\end{center}

\end{frame}




%%%%%%%%%%%%%%%%%%%%%%%%%%%%%%%%%%%%%%%%%%%%%%%%%%%%%%%%%%%%%%%%%%%%%%
\subsection{3-4-13 Related materials}
\begin{frame}[label=CIS-1]
\centerline{\includegraphics[width=1.27\textwidth]{\Figures/R3T4X13/PriorPapers}}
\end{frame}


\begin{frame}[label=CCSG]
\frametitle{Beyond
  magnetic quantum phase transitions}

\centerline{%
\multiinclude[<visible@+- | +->][graphics={width=0.7\columnwidth},format=pdf]{\Figures/ccs/CCSGPhaseDia1}
}
\visible<1->{
\centerline{\small[CeCu$_2$Si$_2$/CeCu$_2$Ge$_2$: Jaccard, e.g. Physica
  B (1999)]}}
\visible<2->{
\centerline{\small[CeCu$_2$(Si/Ge)$_2$: H. Q. Yuan, Science (2003)]}}


\vspace{-1.5ex}
\begin{center}
\begin{minipage}[t]{0.9\columnwidth}
\bi
\item<2->
Superconductivity at low $p$: \hl{magnetic} interaction
\item<2->
Superconductivity at high $p$: \hl{density (or valence) fluctuations?}
%% \item
%% Two supercond. domes \hl{merge} in CeCu$_2$Si$_2$.
\item<3->
Here, explore purely structural quantum phase transition.
\ei
\end{minipage}
\end{center}

\end{frame}




%%%%%%%%%%%%%%%%%%%%%%%%%%%%%%%%%%%%%%%%%%%%%%%%%%%%%%%%%%%%%%%%%%%%%%
\subsection{3-4-13 Related materials}
\begin{frame}[label=CIS-1]
\centerline{\includegraphics[width=1.27\textwidth]{\Figures/R3T4X13/PriorPapers}}
\end{frame}


\begin{frame}[label=CCSG]
\frametitle{Beyond
  magnetic quantum phase transitions}

\centerline{%
\multiinclude[<visible@+- | +->][graphics={width=0.7\columnwidth},format=pdf]{\Figures/ccs/CCSGPhaseDia1}
}
\visible<1->{
\centerline{\small[CeCu$_2$Si$_2$/CeCu$_2$Ge$_2$: Jaccard, e.g. Physica
  B (1999)]}}
\visible<2->{
\centerline{\small[CeCu$_2$(Si/Ge)$_2$: H. Q. Yuan, Science (2003)]}}


\vspace{-1.5ex}
\begin{center}
\begin{minipage}[t]{0.9\columnwidth}
\bi
\item<2->
Superconductivity at low $p$: \hl{magnetic} interaction
\item<2->
Superconductivity at high $p$: \hl{density (or valence) fluctuations?}
%% \item
%% Two supercond. domes \hl{merge} in CeCu$_2$Si$_2$.
\item<3->
Here, explore purely structural quantum phase transition.
\ei
\end{minipage}
\end{center}

\end{frame}




%%%%%%%%%%%%%%%%%%%%%%%%%%%%%%%%%%%%%%%%%%%%%%%%%%%%%%%%%%%%%%%%%%%%%%
\subsection{3-4-13 Related materials}
\begin{frame}[label=CIS-1]
\centerline{\includegraphics[width=1.27\textwidth]{\Figures/R3T4X13/PriorPapers}}
\end{frame}


\begin{frame}[label=CCSG]
\frametitle{Beyond
  magnetic quantum phase transitions}

\centerline{%
\multiinclude[<visible@+- | +->][graphics={width=0.7\columnwidth},format=pdf]{\Figures/ccs/CCSGPhaseDia1}
}
\visible<1->{
\centerline{\small[CeCu$_2$Si$_2$/CeCu$_2$Ge$_2$: Jaccard, e.g. Physica
  B (1999)]}}
\visible<2->{
\centerline{\small[CeCu$_2$(Si/Ge)$_2$: H. Q. Yuan, Science (2003)]}}


\vspace{-1.5ex}
\begin{center}
\begin{minipage}[t]{0.9\columnwidth}
\bi
\item<2->
Superconductivity at low $p$: \hl{magnetic} interaction
\item<2->
Superconductivity at high $p$: \hl{density (or valence) fluctuations?}
%% \item
%% Two supercond. domes \hl{merge} in CeCu$_2$Si$_2$.
\item<3->
Here, explore purely structural quantum phase transition.
\ei
\end{minipage}
\end{center}

\end{frame}



