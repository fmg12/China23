%%%%%%%%%%%%%%%%%%%%%%%%%%%%%%%%%%%%%%%%%%%%%%%%%%%%%%%%%%%%%%%%%%%%%%
\section{(Sr/Ca)$_3$Ir$_4$Sn$_{13}$}
% %%%%%%%%%%%%%%%%%%%%%%%%%%%%%%%%%%%%%%%%%%%%%%%%%%%%%%%%%%%%%%%%%%%%%%
% \begin{frame}[label=CIS-1]
% \frametitle{Quasiskutterudite material family R$_3$T$_4$X$_{13}$}
% \begin{columns}[t]
% \column{0.5\textwidth}
% \begin{itemize}
% % \item
% % R = earth alkaline or rare earth, T = transition metal, X = group-4 (Ge, Sn).
% \item
% Combination of filled skutterudite (LaRu$_4$P$_{12}$) and A15 structure (Nb$_3$Sn): (X'R$_3$) T$_4$X$_{12}$. 
% \item
% Two structure types, both cubic: phase $I$ and $I'$. In $I'$ the lattice constant is doubled: superlattice distortion on $I$. 
% \item
% Here, investigate Sr$_3$Ir$_4$Sn$_{13}$ and substitution series to  Ca$_3$Ir$_4$Sn$_{13}$.

% \item
% Transition $I \rightarrow I'$ on cooling, superlattice formation.
% \end{itemize}
% {\small \centerline{[Klintberg PRL (2012)]}}
% \column{0.5\textwidth}
% \centerline{~}
% \centerline {\includegraphics[width=\columnwidth]{\Figures/R3T4X13/Superlattice_QCP_resub_Figure1}}
% \end{columns}
% \end{frame}

%%%%%%%%%%%%%%%%%%%%%%%%%%%%%%%%%%%%%%%%%%%%%%%%%%%%%%%%%%%%%%%%%%%%%%
\begin{frame}[label=CIS-1]
\frametitle{Quasiskutterudite material family R$_3$T$_4$X$_{13}$}

\centerline{\includegraphics[width=0.8\columnwidth]{\Figures/R3T4X13/Sr3Ir4Sn13_amb-FS-2.pdf}}
\begin{itemize}
% \item
% R = earth alkaline or rare earth, T = transition metal, X = group-4 (Ge, Sn).
\item
Combination of filled skutterudite (LaRu$_4$P$_{12}$) and A15 structure (Nb$_3$Sn): (X'R$_3$) T$_4$X$_{12}$. 

\item
Here, investigate Sr$_3$Ir$_4$Sn$_{13}$ and substitution series to  Ca$_3$Ir$_4$Sn$_{13}$.

\end{itemize}
{\small \centerline{[Klintberg, Goh PRL (2012)]}}

\end{frame}


%%%%%%%%%%%%%%%%%%%%%%%%%%%%%%%%%%%%%%%%%%%%%%%%%%%%%%%%%%%%%%%%%%%%%%
\begin{frame}[label=CIS-1]
\frametitle{Quasiskutterudite material family R$_3$T$_4$X$_{13}$}

\begin{columns}[t]
\column{0.7\textwidth}
\centerline{~}
\centerline{\includegraphics[width=0.8\columnwidth]{\Figures/R3T4X13/Figure1_SQCP.pdf}}
\column{0.3\textwidth}
\centerline{~}
\centerline{\includegraphics[width=\columnwidth]{\Figures/R3T4X13/Figure2a_SQCP.pdf}}
\centerline{\includegraphics[width=\columnwidth]{\Figures/R3T4X13/Figure2b_SQCP.pdf}}
\end{columns}

\begin{itemize}
% \item
% R = earth alkaline or rare earth, T = transition metal, X = group-4 (Ge, Sn).

\item
Two structure types exist, both cubic: phase $I$ and $I'$. In $I'$ the lattice constant is doubled: superlattice distortion on $I$. 

\item
Transition $I \rightarrow I'$ on cooling, superlattice formation.
\end{itemize}

\end{frame}


%%%%%%%%%%%%%%%%%%%%%%%%%%%%%%%%%%%%%%%%%%%%%%%%%%%%%%%%%%%%%%%%%%%%%%
\begin{frame}[label=CIS-2]
\frametitle{Superlattice transition and superconductivity are tuned by
  composition and pressure in (Sr/Ca)$_3$Ir$_4$Sn$_{13}$}
\begin{columns}[t]
\column{0.5\textwidth}
\vspace{-0.7em}
\centerline{~}
%\vspace{2em}
\centerline{\includegraphics[width=\columnwidth]{\Figures/R3T4X13/Superlattice_QCP_resub_Figure2}}
\column{0.5\textwidth}
\centerline{~}
\visible<2->{\centerline {\includegraphics[width=\columnwidth]{\Figures/R3T4X13/Superlattice_QCP_resub_Figure3}}}
\end{columns}

\begin{itemize}
\item
\visible<1->{Structural transition $T^*$ reduced by Ca-substitution
  and pressure.}
\item
\visible<2->{$T_c$ rises on approaching critical pressure, \\ $T$-linear
  resistivity at $p_c$.}
\end{itemize}


\end{frame}

%%%%%%%%%%%%%%%%%%%%%%%%%%%%%%%%%%%%%%%%%%%%%%%%%%%%%%%%%%%%%%%%%%%%%%
\begin{frame}[label=CIS-phonons]
\frametitle{Role of phonon dispersion near critical pressure in (Sr/Ca)$_3$Ir$_4$Sn$_{13}$}
\centerline{\includegraphics[width=0.8\columnwidth]{\Figures/R3T4X13/Phonons}}
\begin{itemize}
\item
Optical mode associated with superlattice transition goes soft near
$p_c$.

\item
Dispersion presumably linear, but entire, narrow-width branch is
lowered.

\item
Can cause linear $\rho(T)$ for $k_B T>\hbar \Omega$ (frequency scale
of soft branch), as
\end{itemize}
\[ \Delta\rho_{ph}(T) \propto \sum_{\bf q} \alpha_{(tr) \bf q}^2 T \left(\partial n_{\bf
  q}/\partial T\right)_{\omega_{\bf q}} \rightarrow  T \sum
  \omega_{\bf q}^{-1} \]


\end{frame}

%%%%%%%%%%%%%%%%%%%%%%%%%%%%%%%%%%%%%%%%%%%%%%%%%%%%%%%%%%%%%%%%%%%%%%
\begin{frame}[label=CIS-3]
\frametitle{Structural quantum critical point in (Sr/Ca)$_3$Ir$_4$Sn$_{13}$}
\centerline{\includegraphics[width=0.6\columnwidth]{\Figures/R3T4X13/Figure5_SQCP.pdf}}

\begin{itemize}
\item
Combined alloying (chemical pressure) and hydrostatic pressure phase diagram.

\item
Structural transition suppressed at $p_c = 1.8 ~{\rm GPa}$ in Ca$_3$Ir$_4$Sn$_{13}$.

\item
Superconducting $T_c$ has broad dome-structure. Associated with softening of optical mode around $p_c$?

\end{itemize}
\end{frame}



%%% Local Variables: 
%%% mode: latex
%%% TeX-master: "GroTalk"
%%% End: 
