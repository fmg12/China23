
%%%%%%%%%%%%%%%%%%%%%%%%%%%%%%%%%%%%%%%%%%%%%%%%%%%%%%%%%%%%%%%%%%%%%%
\subsection{Aperiodic compounds}
%%%%%%%%%%%%%%%%%%%%%%%%%%%%%%%%%%%%%%%%%%%%%%%%%%%%%%%%%%%%%%%%%%%%%%
\begin{frame}
\frametitle{Phasons in compounds}
\centerline{\includegraphics[width=\textwidth]{\Figures/Bi/Bi3/Phonons/Literature1}}

\vspace{1.5em}
\begin{itemize}
\item
Neutron scattering shows 1D-nature of phonon spectrum.
\item
Large literature on theory, but still few experiments.
\item
Magnetic systems?
\end{itemize}

\vspace*{\fill}
\vspace{0.5em}
\centerline{\makebox[\linewidth]{\rule{0.85\textwidth}{0.4pt}}}
\centerline{\scriptsize I. U. Heilmann  {\it et al.} Phys. Rev. B {\bf
    20,} 751 (1979)}
\centerline{\scriptsize X. Chen {\it et al.} Phys. Rev. B {\bf 94,} 134309 (2016).}
\end{frame}




%%%%%%%%%%%%%%%%%%%%%%%%%%%%%%%%%%%%%%%%%%%%%%%%%%%%%%%%%%%%%%%%%%%%%%
\begin{frame}[label=BiIntro]
\frametitle{Bi-III phase: host-guest structure}
\centerline{\includegraphics[width=\columnwidth]{\Figures/Bi/IncommStructLabelled}}
\visible<2->{
\centerline{\includegraphics[width=\columnwidth]{\Figures/Bi/PhasonModecrop}}
}
\begin{itemize}
\item <visible@2->Phason is like fourth acoustic mode (but damped).

% \item Reported ferromagnetic below $T_c \simeq 15~\rm K$, at least one more low-$T$ transition.

% \item Ferromagnetism suppressed at moderate $p_c \simeq 20~{\rm kbar}$.
\end{itemize}

\vspace*{\fill}
\vspace{1.5em}
\centerline{\makebox[\linewidth]{\rule{0.85\textwidth}{0.4pt}}}
\centerline{\scriptsize McMahon, Degtyareva, Nelmes PRL {\bf 85}, 4896
  (2000), Reed and Ackland, PRL {\bf 84,} 5580 (2000)}
\end{frame}



%%%%%%%%%%%%%%%%%%%%%%%%%%%%%%%%%%%%%%%%%%%%%%%%%%%%%%%%%%%%%%%%%%%%%%
%\subsection{Strong coupling superconductivity}
%%%%%%%%%%%%%%%%%%%%%%%%%%%%%%%%%%%%%%%%%%%%%%%%%%%%%%%%%%%%%%%%%%%%%%
\begin{frame}[label=BiSuper2]
  \frametitle{Enhanced electron-phonon coupling in high pressure Bi-III}
  
  \centerline{\includegraphics[width=0.8\columnwidth]{\Figures/Bi/Bi3/BiCritFieldFigure}}
  %\centerline{\scriptsize Phil Brown \hl{P6.3 on Thursday
  %    afternoon}}
  
  \begin{itemize}
  % \item $T_c \simeq 7.2 ~ \text{K}$, like Pb (neighbour in periodic table).
  % \item contrast with Pb: linear $\rho(T)$ at low $T$,
  %   high slope.
  %\item $\rho(T)$ saturates at high $T \rightarrow$ anharmonicity?
  \item Scattering from low energy phonons:
  $\rho \propto T \lambda/\Omega_p^2$\\
  ($\Omega_p = $ plasma freq.)
  
  \item Slope $\rho'$ and calculated $\Omega_p$ give
    $\lambda\sim 2.8$. Similar $\lambda$ from $H_{c2}$ fit.
  \item Strong coupling, type II superconductor. 
  \item Mass enhancement $\propto (1+\lambda)$
    explains short coherence length $\simeq 116$ \AA.
  \end{itemize}
  \end{frame}


  %%%%%%%%%%%%%%%%%%%%%%%%%%%%%%%%%%%%%%%%%%%%%%%%%%%%%%%%%%%%%%%%%%%%%%
\subsection{Role of approximants}
%%%%%%%%%%%%%%%%%%%%%%%%%%%%%%%%%%%%%%%%%%%%%%%%%%%%%%%%%%%%%%%%%%%%%%
\begin{frame}
\frametitle{Role of approximants, demonstrated in potassium}

\centerline{\includegraphics[width=0.9\textwidth]{\data/K/PhononCalc/Approximants}}

\begin{itemize}
\item
Low-lying phason (sliding) modes in large approximants.
\end{itemize}

\vspace*{\fill}
\centerline{\makebox[\linewidth]{\rule{0.85\textwidth}{0.4pt}}}
\centerline{\scriptsize K. Atalar to be published (2023)}

\end{frame}

%%%%%%%%%%%%%%%%%%%%%%%%%%%%%%%%%%%%%%%%%%%%%%%%%%%%%%%%%%%%%%%%%%%%%%
\begin{frame}
\frametitle{e-ph coupling in potassium approximants}

\centerline{\includegraphics[width=0.9\textwidth]{\data/K/PhononCalc/Lambda}}

\begin{itemize}
\item
Obtain $\lambda$ directly from the calculated phonon spectrum.
\item
Soft modes in large approximants produce $\lambda > 2$.

\end{itemize}

\vspace*{\fill}

\centerline{\makebox[\linewidth]{\rule{0.85\textwidth}{0.4pt}}}
\centerline{\scriptsize K. Atalar to be published (2023)}

\end{frame}


%%%%%%%%%%%%%%%%%%%%%%%%%%%%%%%%%%%%%%%%%%%%%%%%%%%%%%%%%%%%%%%%%%%%%%
\begin{frame}[label=SbRes]
%\frametitle{High pressure incommensurate host-guest structure Antimony-II}
\frametitle{Antimony and further examples}
%\vspace{-3em}


\centerline{\includegraphics[width=0.95\columnwidth]{\data/Sb/Puthipong0522/ResAllFig}}

%\centerline{\scriptsize poster by Phil Brown, \hl{P6.3 on Thursday afternoon}}

\begin{itemize}
\item Structural transition (Sb-II $\rightarrow$ Sb-IV)
\item $T_c \simeq 3.5 ~ \text{K}$,  $\rho(T)\propto T^2$ at low $T$ $\leftarrow$ damped phonons(?)
\item Next: Ba, Rb, K, Sc, As, Sr, Nowotny phases
% \item Upper critical field $B_{c2}\simeq 2~\text{T}$, suggesting
% type-II.
%\item Compare to Pb, neighbour in periodic table, $T_c \simeq 7.2~\text{K}$.
%\item<visible@2-> Incommensurate host-guest structure of Bi-III.

\end{itemize}

% \vspace*{\fill}
% %\vspace{0.5em}
% \centerline{\makebox[\linewidth]{\rule{0.85\textwidth}{0.4pt}}}
% \centerline{\scriptsize also Li PRB {\bf 95,} 024510  (2017)}
%\vspace*{\fill}
%\centerline{\makebox[\linewidth]{\rule{0.85\textwidth}{0.4pt}}}
%\centerline{\scriptsize P. Brown Sci. Adv. {\bf 4:}eaao4793 (2018)}
\end{frame}

%%%%%%%%%%%%%%%%%%%%%%%%%%%%%%%%%%%%%%%%%%%%%%%%%%%%%%%%%%%%%%%%%%%%%%
%\subsection{Linear resistivity}
%%%%%%%%%%%%%%%%%%%%%%%%%%%%%%%%%%%%%%%%%%%%%%%%%%%%%%%%%%%%%%%%%%%%%%
\begin{frame}
  %\vspace{-2em}
  \frametitle{Universal prefactors for linear $\rho(T)$ -- Planckian dissipation}
  \centerline{\multiinclude[<+- | visible@+>] [graphics={width=0.8\columnwidth},format=pdf]{\Figures/Bi/Bi3/Test}}
  
  %\vspace{-1em}
  \begin{itemize}
  \item <visible@1-> Observation: $\hbar \tau^{-1} \sim k_B
    T$ (corresponds to $\alpha \simeq 1$).
  
  \item <visible@2-> Bi-III and other strong coupling s/c: $\alpha
    \simeq 2\pi \lambda$.
  % \item <visible@6-> High carrier concentration in high-pressure structure.
  \end{itemize}
  
  \vspace*{\fill}
  \centerline{\makebox[\linewidth]{\rule{0.85\textwidth}{0.4pt}}}
  \centerline{\scriptsize J. Bruin, H. Sakai, R. S. Perry, A. P. Mackenzie, Science {\bf 339,} 804 (2013)} 
  \centerline{\scriptsize S. A. Hartnoll, Nature Phys. {\bf 11,} 54 (2015)} 
  
  \end{frame}
  
  
