
%%%%%%%%%%%%%%%%%%%%%%%%%%%%%%%%%%%%%%%%%%%%%%%%%%%%%%%%%%%%%%%%%%%%%%
\subsection{High pressure structures}
%%%%%%%%%%%%%%%%%%%%%%%%%%%%%%%%%%%%%%%%%%%%%%%%%%%%%%%%%%%%%%%%%%%%%%
\begin{frame}[label=BiRes]
\frametitle{High pressure phases of Bismuth}
%\vspace{-3em}
\centerline{\multiinclude[<+- | visible@+>] [graphics={width=0.9\columnwidth},format=pdf]{\Figures/Bi/Bi3/BiOverviewLog}}

%\vspace{-2em}
\begin{itemize}
\item <visible@1-> Semimetal with $\sim 1$ e$^-$ per $10^5$ atoms at
  low pressure.

\item <visible@2-> Carrier concentration drops with increasing pressure.

\item <visible@6-> Bi-II, III: high carrier concentration, superconductivity.

% \item <visible@6-> High carrier concentration in high-pressure structure.
\end{itemize}

\vspace*{\fill}
\centerline{\makebox[\linewidth]{\rule{0.85\textwidth}{0.4pt}}}
\centerline{\scriptsize P. Brown Physics Procedia {\bf 75,} 29 (2015), Li PRB {\bf 95,} 024510  (2017)}
\end{frame}

%%%%%%%%%%%%%%%%%%%%%%%%%%%%%%%%%%%%%%%%%%%%%%%%%%%%%%%%%%%%%%%%%%%%%%
\subsection{$\rho \propto T$, quasiperiodic structure}
%%%%%%%%%%%%%%%%%%%%%%%%%%%%%%%%%%%%%%%%%%%%%%%%%%%%%%%%%%%%%%%%%%%%%%

\begin{frame}[label=BiSuper1]
\frametitle{Superconductivity and $T$-linear resistivity in high pressure Bi-III}
\centerline{\multiinclude[<+- | visible@+>] [graphics={width=0.95\columnwidth},format=pdf]{\Figures/Bi/Bi3/BiRes1Figure}}
%\centerline{\scriptsize poster by Phil Brown, \hl{P6.3 on Thursday afternoon}}

\begin{itemize}
\item $T_c \simeq 7.2 ~ \text{K}$, linear $\rho(T)$ at low $T$.
% \item Upper critical field $B_{c2}\simeq 2~\text{T}$, suggesting
% type-II.
\item Compare to Pb, neighbour in periodic table, $T_c \simeq 7.2~\text{K}$.
\item<visible@2-> Incommensurate host-guest structure of Bi-III.

\end{itemize}

% \vspace*{\fill}
% %\vspace{0.5em}
% \centerline{\makebox[\linewidth]{\rule{0.85\textwidth}{0.4pt}}}
% \centerline{\scriptsize also Li PRB {\bf 95,} 024510  (2017)}
\vspace*{\fill}
\centerline{\makebox[\linewidth]{\rule{0.85\textwidth}{0.4pt}}}
\centerline{\hlb{\scriptsize P. Brown Sci. Adv. {\bf 4:}eaao4793 (2018)}}
\end{frame}

%%%%%%%%%%%%%%%%%%%%%%%%%%%%%%%%%%%%%%%%%%%%%%%%%%%%%%%%%%%%%%%%%%%%%%
\subsection{Sliding mode}
%%%%%%%%%%%%%%%%%%%%%%%%%%%%%%%%%%%%%%%%%%%%%%%%%%%%%%%%%%%%%%%%%%%%%%
\begin{frame}[label=BiIntro]
\frametitle{Bi-III phase: host-guest structure}
\centerline{\includegraphics[width=0.85\columnwidth]{\Figures/Structures/Bismuth/Bi3Struct}}

\visible<2->{

\centerline{\includegraphics[width=\columnwidth]{\Figures/Bi/PhasonModecrop}}
}

\begin{itemize}
\item <visible@2->Phason is like fourth acoustic mode (but damped).

\end{itemize}

\vspace*{\fill}
\vspace{1.5em}
\centerline{\makebox[\linewidth]{\rule{0.85\textwidth}{0.4pt}}}
\centerline{\scriptsize McMahon, Degtyareva, Nelmes PRL {\bf 85}, 4896
  (2000), Reed and Ackland, PRL {\bf 84,} 5580 (2000)}
\end{frame}

%%%%%%%%%%%%%%%%%%%%%%%%%%%%%%%%%%%%%%%%%%%%%%%%%%%%%%%%%%%%%%%%%%%%%%
%\subsection{Sliding mode}
%%%%%%%%%%%%%%%%%%%%%%%%%%%%%%%%%%%%%%%%%%%%%%%%%%%%%%%%%%%%%%%%%%%%%%
\begin{frame}<handout:0>
  \frametitle{Bi-III phason (sliding) mode}
          % \transduration<0-1>{0.5}
          % \multiinclude[<visible@+-| +>][format=png, graphics={width=\textwidth}]{something}
  \centerline{\animategraphics[autoplay,loop,height=8cm]{8}{\Figures/Bi/Bi3/Phonons/animations/Bi3Mode1-}{0}{15}}
  \end{frame}
  

%%%%%%%%%%%%%%%%%%%%%%%%%%%%%%%%%%%%%%%%%%%%%%%%%%%%%%%%%%%%%%%%%%%%%%
\begin{frame}
\frametitle{Bi-III weak dispersion of phason mode perpendicular to chains}
        % \transduration<0-1>{0.5}
        % \multiinclude[<visible@+-| +>][format=png, graphics={width=\textwidth}]{something}
\centerline{\animategraphics[autoplay,loop,height=8cm]{2}{\Figures/Bi/Bi3/Phonons/animations/Bi3Mode2-}{0}{1}}

\end{frame}

%%%%%%%%%%%%%%%%%%%%%%%%%%%%%%%%%%%%%%%%%%%%%%%%%%%%%%%%%%%%%%%%%%%%%%
%\subsection{Phonon dispersion}
%%%%%%%%%%%%%%%%%%%%%%%%%%%%%%%%%%%%%%%%%%%%%%%%%%%%%%%%%%%%%%%%%%%%%%
\begin{frame}
\frametitle{Bi-III phason branch: flat and soft}
\centerline{\includegraphics[width=0.7\textwidth]{\Figures/Bi/Bi3/Phonons/Monserrat0217/PhononDispersionFigure}}
\begin{itemize}
\item
Low-lying phason modes (2 chains per unit cell) in
42-atom approximant
\item
Little dispersion perpendicular to $c$-axis. % , monatomic-chain dispersion
% parallel $c$.
\item
1D dispersion $\rightarrow$ large contribution to e$^-$-phonon $\lambda$.
\end{itemize}

\end{frame}



%%%%%%%%%%%%%%%%%%%%%%%%%%%%%%%%%%%%%%%%%%%%%%%%%%%%%%%%%%%%%%%%%%%%%%
\subsection{Role of approximants}
%%%%%%%%%%%%%%%%%%%%%%%%%%%%%%%%%%%%%%%%%%%%%%%%%%%%%%%%%%%%%%%%%%%%%%
\begin{frame}
\frametitle{Role of approximants, demonstrated in potassium}

\centerline{\includegraphics[width=0.9\textwidth]{\data/K/PhononCalc/Approximants}}

\begin{itemize}
\item
Low-lying phason (sliding) modes in large approximants.
\end{itemize}

\vspace*{\fill}
\centerline{\makebox[\linewidth]{\rule{0.85\textwidth}{0.4pt}}}
\centerline{\scriptsize K. Atalar to be published (2023)}

\end{frame}

%%%%%%%%%%%%%%%%%%%%%%%%%%%%%%%%%%%%%%%%%%%%%%%%%%%%%%%%%%%%%%%%%%%%%%
\begin{frame}
\frametitle{e-ph coupling in potassium approximants}

\centerline{\includegraphics[width=0.9\textwidth]{\data/K/PhononCalc/Lambda}}

\begin{itemize}
\item
Obtain $\lambda$ directly from the calculated phonon spectrum.
\item
Soft modes in large approximants produce $\lambda > 2$.

\end{itemize}

\vspace*{\fill}

\centerline{\makebox[\linewidth]{\rule{0.85\textwidth}{0.4pt}}}
\centerline{\scriptsize K. Atalar to be published (2023)}

\end{frame}



%%% Local Variables: 
%%% mode: latex
%%% TeX-master: "GroTalk.tex"
%%% End: 
