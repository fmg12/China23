
%%%%%%%%%%%%%%%%%%%%%%%%%%%%%%%%%%%%%%%%%%%%%%%%%%%%%%%%%%%%%%%%%%%%%%
%\subsection{In$_5$Bi$_3$ and Sb-II}
%%%%%%%%%%%%%%%%%%%%%%%%%%%%%%%%%%%%%%%%%%%%%%%%%%%%%%%%%%%%%%%%%%%%%%
\begin{frame}[label=In5Bi3]
\frametitle{Commensurate approximant structure In$_5$Bi$_3$}
%\vspace{-3em}

\centerline{\includegraphics[width=0.95\columnwidth]{\data/In5Bi3/In5Bi3/In5Bi3FigLabels}}
%\centerline{\scriptsize poster by Phil Brown, \hl{P6.3 on Thursday afternoon}}

\begin{itemize}
\item In$_5$Bi$_3$ structure similar to 32-atom approximant for Bi-III.
\item $T_c \simeq \SI{4.2}{\kelvin}$, \hlb{$\rho(T) \propto T^2$ at low $T$.}
\item Structure stabilised by spin-orbit coupling

\end{itemize}


\vspace*{\fill}
\centerline{\makebox[\linewidth]{\rule{0.85\textwidth}{0.4pt}}}
\centerline{\scriptsize S. Chen J. Phys. Mater. {\bf 3,} 015007 (2020)}

\end{frame}

%%%%%%%%%%%%%%%%%%%%%%%%%%%%%%%%%%%%%%%%%%%%%%%%%%%%%%%%%%%%%%%%%%%%%%
\begin{frame}[label=SbRes]
    %\frametitle{High pressure incommensurate host-guest structure Antimony-II}
\frametitle{Antimony and further examples}
    %\vspace{-3em}
    
    
\centerline{\includegraphics[width=0.95\columnwidth]{\data/Sb/FiguresSb/105kbar/Res105Fig2}}
    
    %\centerline{\scriptsize poster by Phil Brown, \hl{P6.3 on Thursday afternoon}}
    
\begin{itemize}
    \item Structural transition (Sb-II $\rightarrow$ Sb-IV)
    \item $T_c \simeq 3.5 ~ \text{K}$,  \hlb{$\rho(T)\propto T^2$ at low $T$} $\leftarrow$ damped phonons(?)
    \item Next: Ba, Rb, K, Sc, As, Sr, Nowotny phases
    % \item Upper critical field $B_{c2}\simeq 2~\text{T}$, suggesting
    % type-II.
    %\item Compare to Pb, neighbour in periodic table, $T_c \simeq 7.2~\text{K}$.
    %\item<visible@2-> Incommensurate host-guest structure of Bi-III.
    
\end{itemize}
% \vspace*{\fill}
% %\vspace{0.5em}
% \centerline{\makebox[\linewidth]{\rule{0.85\textwidth}{0.4pt}}}
% \centerline{\scriptsize also Li PRB {\bf 95,} 024510  (2017)}
%\vspace*{\fill}
%\centerline{\makebox[\linewidth]{\rule{0.85\textwidth}{0.4pt}}}
%\centerline{\scriptsize P. Brown Sci. Adv. {\bf 4:}eaao4793 (2018)}
\end{frame}


%%%%%%%%%%%%%%%%%%%%%%%%%%%%%%%%%%%%%%%%%%%%%%%%%%%%%%%%%%%%%%%%%%%%%%
\begin{frame}[label=T2Origin]
%\frametitle{High pressure incommensurate host-guest structure Antimony-II}
\frametitle{Damped phonons: $\rho \propto T^2$ at low $T$, $\propto T$ for $k_B T>\hbar \omega$}

\begin{columns}[t]
\column {0.5\textwidth}
\centerline{~}
\vspace{-1.5em}
\begin{itemize}
\item Consequences of phonon damping:
\item
e$^-$-phonon scattering
\[
\Delta\rho \propto \sum_q q^k T \frac{\partial}{\partial T} \overline{x_T(q)^2}
\]
\item
Fluctuation-dissipation theorem 
\[
\overline{x_T(q)^2} = \frac{2}{\pi}\int_0^\infty \chi''_{q\omega} n_{\omega} d\omega
\]

\item
Damped oscillator response $\chi''_{q\omega}/\omega =\text{const.}$ at low $\omega$.

\end{itemize}
\vspace{1em}
\[
\hspace{5em}\Delta\rho \propto \int_0^\infty \omega n_\omega d\omega = \int_0^\infty \frac{\omega}{e^{\hbar \omega/k_B T}-1} d\omega \propto T^2
\]


\column {0.5\textwidth}
\centerline{~}
\centerline{\includegraphics[width=0.95\columnwidth]{\data/Figures/Lectures/e-phonon/StructFact.pdf}}

\end{columns}
\end{frame}


\subsection{Overdamped phonons}
%%%%%%%%%%%%%%%%%%%%%%%%%%%%%%%%%%%%%%%%%%%%%%%%%%%%%%%%%%%%%%%%%%%%%%
\begin{frame}[label=T1Origin]
    %\frametitle{High pressure incommensurate host-guest structure Antimony-II}
    \frametitle{Overdamped phonons: crossover temperature to $\rho \propto  T$ reduced}
    
    \begin{columns}[t]
    \column {0.5\textwidth}
    \centerline{~}
    \vspace{-1.5em}
 
    \begin{itemize}
        \setlength\itemsep{0.5em}
    \item
    Observed $\rho \propto T \rightarrow$ \\
    phonon spectrum must be concentrated at low $\omega$
    \item \hl{Overdamped} phonons: 
    Poles shift from $\pm\omega_q$ to $-i\Gamma$ with $\Gamma=\omega_q^2/\gamma$
    \item
    Because $\Gamma \ll \omega_q$, enter $T$-linear regime at lower $T$ than without damping.
    \item Details matter for asymp\-totic low-$T$ form (as for magnetic case).
    \item Also want $C(T)$, $\kappa(T)$, superconductivity.
    \end{itemize}
    
    
    \column {0.5\textwidth}
    \centerline{~}
    \centerline{\includegraphics[width=0.95\columnwidth]{\data/Figures/Lectures/e-phonon/StructFactOverDampedFig.pdf}}
    
    \end{columns}
    \vspace*{\fill}
%\vspace{0.5em}
    \centerline{\makebox[\linewidth]{\rule{0.85\textwidth}{0.4pt}}}
    \centerline{\scriptsize also Ochoa and Fernandes arXiv:2302.00043 (2023), Jiang arXiv:2305.05407 (2023)}
\end{frame}
    
    

% \vspace*{\fill}
% %\vspace{0.5em}
% \centerline{\makebox[\linewidth]{\rule{0.85\textwidth}{0.4pt}}}
% \centerline{\scriptsize also Li PRB {\bf 95,} 024510  (2017)}
%\vspace*{\fill}
%\centerline{\makebox[\linewidth]{\rule{0.85\textwidth}{0.4pt}}}
%\centerline{\scriptsize P. Brown Sci. Adv. {\bf 4:}eaao4793 (2018)}
% \end{frame}



%%% Local Variables: 
%%% mode: latex
%%% TeX-master: "GroTalk.tex"
%%% End: 
