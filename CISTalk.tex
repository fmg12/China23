%%%%%%%%%%%%%%%%%%%%%%%%%%%%%%%%%%%%%%%%%%%%%%%%%%%%%%%%%%%%%%%%%%%%%%

%\subsection{(Sr/Ca)$_3$Ir$_4$Sn$_{13}$}
%\section{Structure}

% %%%%%%%%%%%%%%%%%%%%%%%%%%%%%%%%%%%%%%%%%%%%%%%%%%%%%%%%%%%%%%%%%%%%%%
% \begin{frame}[label=CIS-1]
% \frametitle{Quasiskutterudite material family R$_3$T$_4$X$_{13}$}
% \begin{columns}[t]
% \column{0.5\textwidth}
% \begin{itemize}
% % \item
% % R = earth alkaline or rare earth, T = transition metal, X = group-4 (Ge, Sn).
% \item
% Combination of filled skutterudite (LaRu$_4$P$_{12}$) and A15 structure (Nb$_3$Sn): (X'R$_3$) T$_4$X$_{12}$. 
% \item
% Two structure types, both cubic: phase $I$ and $I'$. In $I'$ the lattice constant is doubled: superlattice distortion on $I$. 
% \item
% Here, investigate Sr$_3$Ir$_4$Sn$_{13}$ and substitution series to  Ca$_3$Ir$_4$Sn$_{13}$.

% \item
% Transition $I \rightarrow I'$ on cooling, superlattice formation.
% \end{itemize}
% {\small \centerline{[Klintberg PRL (2012)]}}
% \column{0.5\textwidth}
% \centerline{~}
% \centerline {\includegraphics[width=\columnwidth]{\Figures/R3T4X13/Superlattice_QCP_resub_Figure1}}
% \end{columns}
% \end{frame}

%%%%%%%%%%%%%%%%%%%%%%%%%%%%%%%%%%%%%%%%%%%%%%%%%%%%%%%%%%%%%%%%%%%%%%
\begin{frame}[plain,label=Conc]
\frametitle {Soft phonons at structural qcp and in quasiperiodic materials}
\includegraphics[width=1.2\textwidth]{\data/R3Ir4Sn13/Collaborators/CollabCISblue}
\end{frame}


%%%%%%%%%%%%%%%%%%%%%%%%%%%%%%%%%%%%%%%%%%%%%%%%%%%%%%%%%%%%%%%%%%%%%%
\subsection{Structural transition in R$_3$T$_4$Sn$_{13}$}
%%%%%%%%%%%%%%%%%%%%%%%%%%%%%%%%%%%%%%%%%%%%%%%%%%%%%%%%%%%%%%%%%%%%%%
\begin{frame}[label=CIS-1]
\frametitle{Cubic quasi-skutterudites R$_3$T$_4$Sn$_{13}$}
%\frametitle{(Sr/Ca)$_3$Ir$_4$Sn$_{13}$}

\centerline{\includegraphics[width=0.8\columnwidth]{\Figures/R3T4X13/Sr3Ir4Sn13_amb-Struct3}}
\begin{itemize}
\item
Second-order lattice instability at $T^* \simeq 147~\rm K$ in
Sr$_3$Ir$_4$Sn$_{13}$, superconductivity at $T_c \simeq 5~\rm K$.
%\item
% Investigate Sr$_3$Ir$_4$Sn$_{13}$ and substitution series to  Ca$_3$Ir$_4$Sn$_{13}$.

\end{itemize}
\vspace*{\fill}
%\vspace{3.5em}
\centerline{\makebox[\linewidth]{\rule{0.85\textwidth}{0.4pt}}}
\centerline{\scriptsize L. Klintberg, S. Goh et al. PRL  {\bf 109,} 237008 (2012)}
\end{frame}




%%%%%%%%%%%%%%%%%%%%%%%%%%%%%%%%%%%%%%%%%%%%%%%%%%%%%%%%%%%%%%%%%%%%%%
\subsection{Structural qcp, $\rho \propto T$}
%%%%%%%%%%%%%%%%%%%%%%%%%%%%%%%%%%%%%%%%%%%%%%%%%%%%%%%%%%%%%%%%%%%%%%
\begin{frame}[label=CIS-2]
\frametitle{Superlattice transition and superconductivity are tuned by
  composition and pressure in (Sr/Ca)$_3$Ir$_4$Sn$_{13}$}
\begin{columns}[t]
\column{0.5\textwidth}
\vspace{-1.25em}
\centerline{~}
%\vspace{2em}
\centerline{\includegraphics[height=0.96\columnwidth]{\Figures/R3T4X13/Superlattice_QCP_resub_Figure2-annot}}
\column{0.5\textwidth}
\centerline{~}
\visible<2->{\centerline {\includegraphics[height=0.98\columnwidth]{\Figures/R3T4X13/CISRTnoInsetLabel}}}
\end{columns}

\begin{itemize}
\item
\visible<1->{Structural transition $T^*$ reduced by Ca-substitution
  and pressure.}
\item
\visible<2->{$T_c$ rises on approaching critical pressure.}
\item
\visible<3->{\hl {$T$-linear
  resistivity at $p_c$.}}
\end{itemize}

\vspace*{\fill}
%\vspace{3.5em}
\centerline{\makebox[\linewidth]{\rule{0.85\textwidth}{0.4pt}}}
\centerline{\scriptsize L. Klintberg, S. Goh et al. PRL  {\bf 109,} 237008 (2012)}
\end{frame}

%%%%%%%%%%%%%%%%%%%%%%%%%%%%%%%%%%%%%%%%%%%%%%%%%%%%%%%%%%%%%%%%%%%%%%
%\subsection{Phonon softening}
\begin{frame}[label=CIS-phonons]
\frametitle{Role of phonon dispersion near critical pressure in (Sr/Ca)$_3$Ir$_4$Sn$_{13}$}
\centerline{\includegraphics[width=0.8\columnwidth]{\Figures/R3T4X13/Phonons}}
\begin{itemize}
\item
Optical mode associated with superlattice transition goes soft near
$p_c$.

% \item
% Dispersion presumably linear, but entire, narrow-width branch is
% lowered.

\item
Can cause linear $\rho(T)$ for $k_B T>\hbar \Omega$ (frequency scale
of soft branch), as
\end{itemize}
\begin{align*} \Delta\rho_{ph}(T) \propto \sum_{\bf q} \alpha_{(tr) \bf q}^2 &T (\partial  \underuparrow{n_{\bf q}}/\partial T )_{\omega_{\bf q}} \rightarrow  T \sum
  \omega_{\bf q}^{-1} (\propto T \lambda ) \\
&{\scriptstyle \left(e^{\hbar \omega /k_B T} - 1\right)^{-1}  }
\end{align*}



\end{frame}





%%% Local Variables: 
%%% mode: latex
%%% TeX-master: "GroTalk"
%%% End: 
