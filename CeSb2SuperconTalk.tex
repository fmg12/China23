%%%%%%%%%%%%%%%%%%%%%%%%%%%%%%%%%%%%%%%%%%%%%%%%%%%%%%%%%%%%%%%%%%%%%%
\subsection{High $p$ phase}
%%%%%%%%%%%%%%%%%%%%%%%%%%%%%%%%%%%%%%%%%%%%%%%%%%%%%%%%%%%%%%%%%%%%%%

\begin{frame}[label=NormalState]
%\frametitle{High pressure incommensurate host-guest structure Antimony-II}
\frametitle{Electronic states in the high pressure structure of CeSb$_2$}
%\vspace{-3em}

\centerline{\includegraphics[width=0.95\columnwidth]{\data/CeSb2/FiguresCeSb2/2022Paper/NormalState}}
%\centerline{\scriptsize poster by Phil Brown, \hl{P6.3 on Thursday afternoon}}

\begin{itemize}
\item \hl{Superconductivity} below \SI{250}{\milli\kelvin}.
\item Increasing $p \rightarrow$ rapid reduction of low-$T$ scattering. %, $\rho(T)$ flattens.
\item $T-$linear resistivity at critical pressure.

\end{itemize}


\end{frame}




%%%%%%%%%%%%%%%%%%%%%%%%%%%%%%%%%%%%%%%%%%%%%%%%%%%%%%%%%%%%%%%%%%%%%%
\subsection{Phase diagram}
%%%%%%%%%%%%%%%%%%%%%%%%%%%%%%%%%%%%%%%%%%%%%%%%%%%%%%%%%%%%%%%%%%%%%%

\begin{frame}[label=PhaseDia]
%\frametitle{High pressure incommensurate host-guest structure Antimony-II}
\frametitle{Phase diagram and quantum critical point in CeSb$_2$}
%\vspace{-3em}

%\framezoom<1><2>[border=4](6.8cm,1.2cm)(1.4cm,1.5cm)
\centerline{\includegraphics[width=0.95\columnwidth]{\data/CeSb2/FiguresCeSb2/2022Paper/PhaseDia}}
%\centerline{\scriptsize poster by Phil Brown, \hl{P6.3 on Thursday afternoon}}

\begin{itemize}
\item Transition signatures in high $p$ heat capacity and resistivity.
\item Extrapolate to qcp near $\SI{32}{\kilo\bar}$, superconducting dome.
\item Structure locally non-centrosymmetric around Ce.
% \item Upper critical field $B_{c2}\simeq 2~\text{T}$, suggesting
% type-II.
%\item Compare to Pb, neighbour in periodic table, $T_c \simeq 7.2~\text{K}$.
%\item<visible@2-> Incommensurate host-guest structure of Bi-III.

\end{itemize}

\vfill
\centerline{\makebox[\linewidth]{\rule{0.85\textwidth}{0.4pt}}}

\centerline{\scriptsize Squire PRL {\bf 131,} 026001 (2023)}

\end{frame}


\subsection{$T_c$, high $B_{c2}$, $\rho \propto T$}
%%%%%%%%%%%%%%%%%%%%%%%%%%%%%%%%%%%%%%%%%%%%%%%%%%%%%%%%%%%%%%%%%%%%%%

\begin{frame}[label=KeyResult]
\frametitle{Superconductivity, low coherence temperature, high critical field }
%\vspace{-3em}

\centerline{\includegraphics[width=0.9\columnwidth]{\data/CeSb2/FiguresCeSb2/2022Paper/Super31p6Fig}}
\begin{itemize}
\item $T$-linear resistivity, quickly saturates below $\SI{20}{\kelvin}$.
\item $B_{c2}$ is high, $>8\times$ Pauli limit.
\item Inverted S-shape of critical field curve.
\end{itemize}
\end{frame}


\subsection{Pauli limit violation}
%%%%%%%%%%%%%%%%%%%%%%%%%%%%%%%%%%%%%%%%%%%%%%%%%%%%%%%%%%%%%%%%%%%%%%
\begin{frame}[label=SuperconTable]
\frametitle{Enhanced critical fields in strongly correlated electron systems}
\begin{columns}[t]
\column{0.66\textwidth}
\centerline{~}
\includegraphics[width=\textwidth]{\data/CeSb2/FiguresCeSb2/PauliLimiting/Table}

\begin{itemize}
\item
Conventional: $B_\text{Pauli} \simeq 1.84 T_c \SI{}{\tesla/\kelvin}$
\item
Heavy fermion materials \\
violate Pauli limit
\item
Strong coupling: $B^*_\text{Pauli} \rightarrow 1.5 (1+\lambda) T_c (\SI{}{\tesla/\kelvin})$
\end{itemize}
\column{0.34\textwidth}
\centerline{~}
\includegraphics[width=0.9\textwidth]{\data/CeSb2/FiguresCeSb2/PauliLimiting/Aperis}
\end{columns}

%\begin{itemize}
%\item
\vspace{1em}
\begin{center}
Even-frequency singlet and odd-frequency triplet mix in field.
%\item 
Pauli limiting in heavy fermions  needs more work.
\end{center}
%\end{itemize}


\vspace{0em}
\centerline{\makebox[\linewidth]{\rule{0.85\textwidth}{0.4pt}}}

\centerline{\scriptsize Schossmann \& Carbotte, PRB {\bf 39,} 4210 (1989)}


\end{frame}


%%% Local Variables: 
%%% mode: latex
%%% TeX-master: "GroTalk.tex"
%%% End: 
