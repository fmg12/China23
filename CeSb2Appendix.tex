
%%%%%%%%%%%%%%%%%%%%%%%%%%%%%%%%%%%%%%%%%%%%%%%%%%%%%%%%%%%%%%%%%%%%%%
\subsection{CeSb$_2$}

%%%%%%%%%%%%%%%%%%%%%%%%%%%%%%%%%%%%%%%%%%%%%%%%%%%%%%%%%%%%%%%%%%%%%%
\begin{frame}[label=CeSb2Intro]
    \frametitle{Ferromagnetic Kondo lattice CeSb$_2$}
    \centerline{\includegraphics[width=\columnwidth]{\data/CeSb2/FiguresCeSb2/LitData/Kagayama1}}
    
    \vspace{1.5em}
    \begin{itemize}
    \item Ferromagnetism disappears at moderate $p_c \simeq \SI{20}{\kilo\bar}$.
    \item Can we observe a ferromagnetic qcp?
    \end{itemize}
    
    \vspace{3 em}
    \centerline{\makebox[\linewidth]{\rule{0.85\textwidth}{0.4pt}}}
    
    \centerline{\scriptsize [Kagayama Physica B (2000, 2005)]}
    
    \end{frame}
    
\subsection{High-$p$ XRD}
%%%%%%%%%%%%%%%%%%%%%%%%%%%%%%%%%%%%%%%%%%%%%%%%%%%%%%%%%%%%%%%%%%%%%%
\begin{frame}[label=CeSb2xray]
\frametitle{High pressure x-ray diffraction in CeSb$_2$}
\centerline{\includegraphics[width=\columnwidth]{\data/CeSb2/FiguresCeSb2/DLS2012/DLS2012xrdFig} }
\centerline{\small [with P. Niklowitz, H. Wilhelm, T. Giles, RHUL/Diamond 2012]}

\begin{itemize}
\item Structural transition crossed below 10 kbar at room temperature.

\item Magnetism collapses because of structural transition.

\item High pressure structure?
\end{itemize}

\end{frame}
    

\subsection{High-$p$ structural change}
%%%%%%%%%%%%%%%%%%%%%%%%%%%%%%%%%%%%%%%%%%%%%%%%%%%%%%%%%%%%%%%%%%%%%%
\begin{frame}[label=CeSb2Res]
\frametitle{High pressure resistivity in CeSb$_2$}
\vspace{-3em}
\centerline{\multiinclude[<+- | visible@+>] [graphics={width=\columnwidth},format=pdf]{\data/CeSb2/FiguresCeSb2/p30b/CeSb2RTmagn} }
\vspace{-2em}
\begin{itemize}
\item <visible@2-> Transitions shift little initially, then abrupt change.

\item <visible@3-> Hysteresis at high temperature, moving to lower $T$.

\item <visible@3-> Suggests lattice instability above 300 K at ambient pressure.
\end{itemize}

\end{frame}


%%%%%%%%%%%%%%%%%%%%%%%%%%%%%%%%%%%%%%%%%%%%%%%%%%%%%%%%%%%%%%%%%%%%%%
\begin{frame}[label=CeSb2xray-2]
\frametitle{Resolution of high pressure structure in CeSb$_2$}
\begin{columns}[c]

\begin{column}{0.3\textwidth}
\centerline{~}
\centerline{\includegraphics[width=\columnwidth]{\Figures/Structures/CeSb2/CeSb2.png}
}
\end{column}
\vrule{}
\visible<2->{
\begin{column}{0.35\textwidth}
\centerline{~}
\centerline{\includegraphics[width=\columnwidth]{\Figures/Structures/CeSb2/CeSb2_p12a.png}}
\end{column} }
\visible<3->{
\begin{column}{0.35\textwidth}
\centerline{~}
\centerline{\includegraphics[width=\columnwidth]{\Figures/Structures/CeSb2/CeSb2_p12b.png}}
\end{column}
}
\end{columns}


\begin{columns}
\column{0.3\textwidth}
\centerline{~}
\visible<2->{
\column{0.35\textwidth}
\centerline{View 1}}
\visible<3->{
\column{0.35\textwidth}
\centerline{View 2} }

\end{columns}

\begin{columns}
\column{0.3\textwidth}
\begin{center}
Low pressure:\\
 SmSb$_2$-structure\\
(orthorhombic, Cmca)
\end{center}

\visible<2->{
\column{0.7\textwidth}
\begin{center}
High pressure: \\
EuSb$_2$-structure\\
(monoclinic, P21/m)
\end{center} }
\end{columns}
\end{frame}


%%%%%%%%%%%%%%%%%%%%%%%%%%%%%%%%%%%%%%%%%%%%%%%%%%%%%%%%%%%%%%%%%%%%%%
\begin{frame}[label=CeSb2xray-2]
\frametitle{Resolution of high pressure structure in CeSb$_2$}
%\begin{columns}[c]

%\begin{column}{0.3\textwidth}
%\centerline{~}
%\centerline{\includegraphics[width=\columnwidth]{\Figures/Structures/CeSb2/CeSb2.png}}
\centerline{\includegraphics[width=\columnwidth]{\data/CeSb2/FiguresCeSb2/DLS2022/XRDRefinements}}

\centerline{YbSb$_2$ structure matches high $p$ XRD pattern best.}

\end{frame}
    

\subsection{Lattice pars. vs. $p$}
%%%%%%%%%%%%%%%%%%%%%%%%%%%%%%%%%%%%%%%%%%%%%%%%%%%%%%%%%%%%%%%%%%%%%%
\begin{frame}[label=LatticePars]
\frametitle{Evolution of structure with pressure in CeSb$_2$, numerical results}
\centerline{\includegraphics[width=0.9\columnwidth]{\data/CeSb2/FiguresCeSb2/DLS2022/P-Dependence}}
%\centerline{Lattice parameters extracted by indexing a pronounced XRD peak}

\end{frame}

%%%%%%%%%%%%%%%%%%%%%%%%%%%%%%%%%%%%%%%%%%%%%%%%%%%%%%%%%%%%%%%%%%%%%%
\begin{frame}[label=CeSb2HighestP]
\frametitle{Mesoscopic phase separation in intermediate pressure CeSb$_2$}
\centerline{\includegraphics[width=\textwidth]{\data/CeSb2/FiguresCeSb2/PhaseCoexistOverview.pdf}}

\begin{itemize}
\item
Phase coexistence near structural transition.
\item
High pressure structure present as minority phase at low pressures.

\end{itemize}
\end{frame}

%%%%%%%%%%%%%%%%%%%%%%%%%%%%%%%%%%%%%%%%%%%%%%%%%%%%%%%%%%%%%%%%%%%%%%
\begin{frame}[label=SuperconTable]
\frametitle{Critical fields in strongly correlated electron systems}
\includegraphics[width=0.9\textwidth]{\data/CeSb2/FiguresCeSb2/2022Paper/SCTable.jpg}
\end{frame}



\subsection{High-$p$ muSR}
%%%%%%%%%%%%%%%%%%%%%%%%%%%%%%%%%%%%%%%%%%%%%%%%%%%%%%%%%%%%%%%%%%%%%%
\begin{frame}[label=CeSb2MagnTrans]
\frametitle{Magnetic transition in high $p$ phase of CeSb$_2$}
\vspace{0em}
\centerline{\includegraphics[width=0.8\columnwidth]{\data/CeSb2/FiguresCeSb2/HighTHighPResist/CeSb2StructTransFig}}

\begin{itemize}
\item At $\SI{21.5}{\kilo\bar}$, kink in $\rho(T)$ at $\sim \SI{2}{\kelvin}$.
\item Suggests a magnetic transition.
\item Follow up with muSR.
\end{itemize}

\end{frame}




%%%%%%%%%%%%%%%%%%%%%%%%%%%%%%%%%%%%%%%%%%%%%%%%%%%%%%%%%%%%%%%%%%%%%%
\subsection{Towards a qcp in the high-$p$ structure}
%%%%%%%%%%%%%%%%%%%%%%%%%%%%%%%%%%%%%%%%%%%%%%%%%%%%%%%%%%%%%%%%%%%%%%
\begin{frame}[label=CeSb2HighestP]
\frametitle{Heavy fermion state in high pressure structure of CeSb$_2$}


\centerline{\includegraphics[width=0.96\columnwidth]{\data/CeSb2/FiguresCeSb2/HighPBoth.png}}

\begin{itemize}
\item
High pressure CeSb$_2$ has very low Kondo temperature.

\item
Magnetic transition at $T_m$, suppressed at 25kbar.

\item
In field, Fermi liquid $T^2$ resistivity with reduced prefactor.

\end{itemize}
\vspace{0em}
\centerline{\makebox[\linewidth]{\rule{0.85\textwidth}{0.4pt}}}

\centerline{\scriptsize Preparatory work Y. Zou, V. Fedoseev $(\simeq 2015)$}

\end{frame}







%%% Local Variables: 
%%% mode: latex
%%% TeX-master: "GroTalk.tex"
%%% End: 
