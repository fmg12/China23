%%%%%%%%%%%%%%%%%%%%%%%%%%%%%%%%%%%%%%%%%%%%%%%%%%%%%%%%%%%%%%%%%%%%%%
%%\section{Introduction to magnetism}
%%%%%%%%%%%%%%%%%%%%%%%%%%%%%%%%%%%%%%%%%%%%%%%%%%%%%%%%%%%%%%%%%%%%%%
\begin{frame}[label=MethodsIntro]
  \frametitle{Quantum Matter -- \\ superconductivity, magnetism, novel electronic states in metals}

\begin{columns}[c]
\column{0.6\textwidth}
\centerline{~}
\tableofcontents

\column{0.4\textwidth}
\centerline{~}
\includegraphics[width=0.5\columnwidth]{\Figures/Lectures/QuantOsc/FermiSurface.jpg}
\includegraphics[width=0.5\columnwidth]{\Figures/Lectures/AlirezaCell/MagnCell}\\
\includegraphics[width=0.5\columnwidth]{\Figures/photos/Furnaces/NbFe2MirrorFurnace.jpg}
\includegraphics[width=0.5\columnwidth]{\Figures/cps/cpsphasesnormal} \\

\includegraphics[width=0.9\columnwidth,clip=on]{\Figures/Phasedias/phasedias08}
\end{columns}



\end{frame}

%%%%%%%%%%%%%%%%%%%%%%%%%%%%%%%%%%%%%%%%%%%%%%%%%%%%%%%%%%%%%%%%%%%%%
\section{Quantum matter}
%%%%%%%%%%%%%%%%%%%%%%%%%%%%%%%%%%%%%%%%%%%%%%%%%%%%%%%%%%%%%%%%%%%%%
\begin{frame}[label=Moredifferent]
\frametitle{More is different}

\begin{columns}[c]
\column{0.5\textwidth}
\centerline{\includegraphics[width=0.8\columnwidth]{\Figures/photos/PWA}}

\column{0.5\textwidth}
{\bf ...at each new level of complexity, entirely new properties
  appear, and the understanding of this behavior requires research
  which I think is as fundamental in its nature as any other.} \\

\centerline{\scriptsize [P. W. Anderson, More is different, 1972]}
\end{columns}
\vspace{2em}
\pause
\hl{Quantum Matter:}  quantum statistics is essential. Freezing-out of
normal modes. \\

\hl{Examples:} phonons below Debye temperature, electrons in metals,
helium at low temperature.
\end{frame}


%%%%%%%%%%%%%%%%%%%%%%%%%%%%%%%%%%%%%%%%%%%%%%%%%%%%%%%%%%%%%%%%%%%%%
%\section{Quantum phase transitions}
%%%%%%%%%%%%%%%%%%%%%%%%%%%%%%%%%%%%%%%%%%%%%%%%%%%%%%%%%%%%%%%%%%%%%
\begin{frame}[label=ElecStates]
\frametitle{Electronic self-organisation}

\centerline{\multiinclude[<visible@+-| +->][format=pdf,graphics={width=0.8\columnwidth}]{\Figures/FreeElec/FreeElecLayers}}
\vspace{2ex}

\begin{itemize}
\item<visible@5-> Diversity of condensates\\ {\small (e.g. spin/charge
density wave, (spin-) Peierls, structural,
Pomeranchuk, metamagnetic, nematic, multipolar 
hidden order)}

\item<visible@6-> Interacting-electron chemistry.

\item<visible@6-> Purity needed to allow complex order.
\end{itemize}

\end{frame}

%%%%%%%%%%%%%%%%%%%%%%%%%%%%%%%%%%%%%%%%%%%%%%%%%%%%%%%%%%%%%%%%%%%%%
\begin{frame}[label=ElecLiquid]
\frametitle{Tuning the quasiparticle interaction}

\centerline{\multiinclude[<visible@+-| +->][format=pdf,graphics={width=\columnwidth}]{\Figures/FreeElec/ElecLiquid}}
\vspace{2ex}

%% \visible {\centerline{\parbox{0.9\columnwidth}{and many others, e.g. charge
%% density wave, (spin-) Peierls transition, structural changes,
%% Pomeranchuk instability (Sr$_3$Ru$_2$O$_7$), metamagnetic transitions,
%% hidden order (URu$_2$Si$_2$), ... }}}

\end{frame}


%%%%%%%%%%%%%%%%%%%%%%%%%%%%%%%%%%%%%%%%%%%%%%%%%%%%%%%%%%%%%%%%%%%%%%
\subsection{Band magnetism}
%%%%%%%%%%%%%%%%%%%%%%%%%%%%%%%%%%%%%%%%%%%%%%%%%%%%%%%%%%%%%%%%%%%%%%
\begin{frame}[label=BorderFM1]
\frametitle{The border of band ferromagnetism -- in broad strokes}

\vspace{-2ex}
\begin{minipage}{\textwidth}
\begin{columns}[c]
\column{0.5\textwidth}
\begin{itemize}
  \item
    \hl{Band magnet} (ZrZn$_2$, Ni$_3$Al, MnSi), Pauli susceptibility:
\end{itemize}
\column{0.5\textwidth}
$\chi_0 \propto g(E_F)$
\end{columns}
\end{minipage}

\pause
\begin{minipage}{\textwidth}
\begin{columns}[b]
\column{0.5\textwidth}
\begin{itemize}
  \item
    Exchange-enhanced:
\end{itemize}
\column{0.5\textwidth}
$\chi=\frac{\chi_0}{1-I\chi_0} $
\end{columns}
\end{minipage}

\pause
\begin{minipage}{\textwidth}
\begin{columns}[b]
\column{0.5\textwidth}
\begin{itemize}
  \item
    Landau \hl{free energy:} 
\end{itemize}
\column{0.5\textwidth}
$ F=\mu_0\left(\frac{a}{2}M^2 + \frac{b}{4}M^4- H M
    \right) $
\end{columns}
\end{minipage}

\pause
\begin{minipage}{\textwidth}
\begin{columns}[b]
\column{0.5\textwidth}
\begin{itemize}
  \item
    \hl{Equation of state} at $T=0$: 
\end{itemize}
\column{0.5\textwidth}
$H=a M + b M^3 $
\end{columns}
\end{minipage}

\pause
\begin{minipage}{\textwidth}
\begin{columns}[b]
\column{0.5\textwidth}
\begin{itemize}
  \item
    Add $T$-dependent \hl{fluctuations:} 
\end{itemize}
\column{0.5\textwidth}
$H=\overline H + h$, $M = \overline M + m$
\end{columns}
\end{minipage}

\pause
\begin{minipage}{\textwidth}
\begin{columns}[b]
\column{0.5\textwidth}
\begin{itemize}
  \item
    \hl{Average:}
\end{itemize}
\column{0.5\textwidth}
$ \overline H = a \overline M + b \overline M^3 + 3b \overline M
    \left< m^2 \right> $
\end{columns}
\end{minipage}

\pause
\begin{minipage}{\textwidth}
\begin{columns}[b]
\column{0.5\textwidth}
\begin{itemize}
  \item
    \hl{Modified equation of state} at finite $T$:
\end{itemize}
\column{0.5\textwidth}
    $\overline H = \left( a + \Delta a(T) \right) \overline M + b \overline
    M^3 $ 
\end{columns}
\end{minipage}

\pause
\begin{beamercolorbox}{postit}
\[ \chi^{-1}(T) = \chi^{-1}(0) + \Delta a(T) \simeq \chi^{-1}(0) + 3 b \left<
    m^2 \right>(T) \]
\end{beamercolorbox}
 
\end{frame}



%%%%%%%%%%%%%%%%%%%%%%%%%%%%%%%%%%%%%%%%%%%%%%%%%%%%%%%%%%%%%%%%%%%%%%
\subsection{MnSi: first order transitions and anomalous power laws}
%%%%%%%%%%%%%%%%%%%%%%%%%%%%%%%%%%%%%%%%%%%%%%%%%%%%%%%%%%%%%%%%%%%%%%

\begin{frame}[label=BorderFM2]
\frametitle{Quantum criticality on the border of ferromagnetism: MnSi}

\vspace{-4ex}

\begin{columns}[t]
  
  \column{0.33\textwidth}
    \centerline{~}
    \visible<1->{\includegraphics[width=\columnwidth]{\Figures/BorderFM/FM1}}

  \column{0.66\textwidth}
  \begin{columns}[t]
    \column{0.49\textwidth}
      \centerline{~}
      \visible<2->{\includegraphics[width=\columnwidth]{\Figures/BorderFM/pfleiderer97-1.pdf}}
      \column{0.49\textwidth}
      \centerline{~}
      \visible<3->{\includegraphics[width=\columnwidth]{\Figures/BorderFM/pfleiderer97-2.pdf}}
    \end{columns}
    \visible<2->{\centerline{\tiny[MnSi: Pfleiderer {\it et al.~} PRB {\bf 55} 8330 (1997)]}}
\end{columns}

\begin{minipage}{\textwidth}
  \begin{columns}[t]
    \column{0.33\textwidth}
    \begin{itemize}
    \item<1->
      Pressure \hl{tunes transition} temperature.
    \end{itemize}
    \column{0.33\textwidth}
    \bi
    \item<2-> Ferro\-mag\-netism \hl{disappears.}
    \ei
    \column{0.33\textwidth}
    \bi
    \item<3-> Scattering cross-section \hl{diverges.}
    \ei
    
  \end{columns}
\end{minipage}

\end{frame}

%%%%%%%%%%%%%%%%%%%%%%%%%%%%%%%%%%%%%%%%%%%%%%%%%%%%%%%%%%%%%%%%%%%%%%
\subsection{Superconductivity and magnetism}
%%\subsection{CePd$_2$Si$_2$}
%%%%%%%%%%%%%%%%%%%%%%%%%%%%%%%%%%%%%%%%%%%%%%%%%%%%%%%%%%%%%%%%%%%%%%
\begin{frame}[label=CPS]
\frametitle{CePd$_2$Si$_2$: heavy-fermion magnet to unconventional superconductor}
\begin{columns}[t]
  \column{0.37\textwidth}
  \begin{itemize}
  \item<1-> Antiferro\-magnet below $T_N\simeq 10 ~\mathrm K$.

  \item<2-> $T_N$ depends on pressure.

  \item<4->  Magnetism suppressed near 2.8 GPa.

  \item<5-> Anomalous resistivity $T$-dependence.
  \end{itemize}

\vspace{1em}
\centerline{\scriptsize [Mathur {\em et al.}, Nature {\bf 394} (1998) 39]}
  \column{0.63\textwidth}
  \centerline{}
  \onslide+<1->
  \centerline{\multiinclude[<visible@+- | +->][graphics={width=0.95\columnwidth},format=pdf]{\Figures/cps/cpsphasesnormal}}
\end{columns}
\onslide+<6->
\begin{center}
\hl{Superconductivity and anomalous normal state.}
\end{center}
\end{frame}



%%%%%%%%%%%%%%%%%%%%%%%%%%%%%%%%%%%%%%%%%%%%%%%%%%%%%%%%%%%%%%%%%%%%%%
\begin{frame}[label=MagnInter]
\frametitle{Magnetic interaction in metals on the threshold of magnetism}

\begin{columns}[t]
\column{0.5\textwidth}
\centerline{~}
\centerline{\includegraphics[width=0.8\columnwidth]{\Figures/CritConcepts/MagnInter1}}
\onslide<2->
\begin{itemize}
\item<2->
\hl{Magnetic moment} of quasiparticle 1 creates \hl{exchange field $h_1$}.

\item<3->
Exchange field $h_1$ leads to \hl{magnetisation} $m \propto \chi$. 

\item<4->
\hl{Quasiparticle 2} is affected by resulting exchange interaction.
\end{itemize}

\column{0.5\textwidth}
\centerline{}
\onslide<5->
%% \begin{tabular}[t]{ll}
%% Effect. inter$^n$ & $V(\mu_1, \mu_2) = -\lambda^2 \chi_{q\omega} \vec\mu_1 \cdot \vec\mu_2$ \tabularnewline
%%  & \tabularnewline
%% Susceptibility & $\chi_{\vec q \omega}^{-1} = \chi_\vec q^{-1} \left(1-i\omega/\Gamma_\vec q\right) $ \tabularnewline
%% \end{tabular}
\[V(\mu_1, \mu_2) = -\lambda^2 \chi_{q\omega} \vec\mu_1 \cdot \vec\mu_2\]
\[\chi_{\vec q \omega}^{-1} = \chi_\vec q^{-1} \left(1-i\omega/\Gamma_\vec q\right) \]

\visible<6->{\includegraphics[width=\columnwidth]{\Figures/CritConcepts/FMwave} }
\end{columns}

\onslide<6->{%
\centerline{\parbox{8cm}{\small [Interaction potential due to \\ moving quasiparticle in an incipient ferromagnet]}}}
\end{frame}

%%%%%%%%%%%%%%%%%%%%%%%%%%%%%%%%%%%%%%%%%%%%%%%%%%%%%%%%%%%%%%%%%%%%%%
\begin{frame}[label=ThreshMagn]
\frametitle{Superconductivity wide-spread on the threshold of magnetism}
\begin{columns}[t]
\vspace{-4ex}
\column{0.5\textwidth}
\centerline{~}
\visible<+->{\includegraphics[width=0.9\columnwidth]{\Figures/cps/cpssuper}}

\visible<+->{\includegraphics[width=0.9\columnwidth]{\Figures/cein3/cein3super}\\
\centerline{\scriptsize{[N. D. Mathur, Nature {\bf 394}, 39 (1998)]}}}

\column{0.5\textwidth}
\centerline{~}
\visible<+->{\includegraphics[width=0.9\columnwidth]{\Figures/uge2/uge2super.jpg}\\
\centerline{\scriptsize{[S. S. Saxena, Nature {\bf 406} 587 (2000)]}}}

\vspace{2ex}

% \visible<4->{
% \centerline{\includegraphics[width=0.85\columnwidth]{\Figures/Lectures/PistonCylinderCell/FThroughPhoto}}}

\visible<4->{
Further examples:

\hl{CeCu$_2$Si$_2$}, CeNi$_2$Ge$_2$, CeRh$_2$Si$_2$, CeCu$_2$,
CeCu$_5$Au, CeIrIn$_5$, URhGe, ...

}
\end{columns}
\end{frame}




%%%%%%%%%%%%%%%%%%%%%%%%%%%%%%%%%%%%%%%%%%%%%%%%%%%%%%%%%%%%%%%%%%%%%%
\subsection{Heavy fermion compounds}
%%%%%%%%%%%%%%%%%%%%%%%%%%%%%%%%%%%%%%%%%%%%%%%%%%%%%%%%%%%%%%%%%%%%%%
\begin{frame}[label=Superconductivity1]
  \frametitle{The first correlated-electron superconductor, a heavy fermion system}
\centerline{\includegraphics[width=\columnwidth]{\Figures/Lectures/HeavyFermion/steglich1}}
\vspace{1em}
\centerline{\small [Steglich {\em et al.}, Phys. Rev. Lett. {\bf 43} (1979) 1892]}
\end{frame}

%%%%%%%%%%%%%%%%%%%%%%%%%%%%%%%%%%%%%%%%%%%%%%%%%%%%%%%%%%%%%%%%%%%%%%
\begin{frame}[label=Superconductivity2]
  \frametitle{The first correlated-electron superconductor, a heavy fermion system}

\begin{columns}[t]
\column{0.5\textwidth}
\centerline{~}
\includegraphics[width=\columnwidth]{\Figures/Lectures/HeavyFermion/steglichRho}
\bi
\item $T_c \simeq 0.6~\rm K$ is high, when compared to effective bandwidth ($\sim 10~\rm K$)

\item
Jump in heat capacity $\sim$ electronic heat capacity $\Rightarrow$ heavy fermions superconduct.

\ei

\centerline{\scriptsize[Steglich {\em et al.}, Phys. Rev. Lett. {\bf 43} (1979) 1892]}
\column{0.5\textwidth}
\centerline{~}
\includegraphics[width=\columnwidth]{\Figures/Lectures/HeavyFermion/steglichHeatCap}
\end{columns}
\end{frame}

%%%%%%%%%%%%%%%%%%%%%%%%%%%%%%%%%%%%%%%%%%%%%%%%%%%%%%%%%%%%%%%%%%%%%%
\begin{frame}[label=dHvA]
  \frametitle{Quasiparticles detected in de Haas-van Alphen experiments}

\begin{columns}[t]
\column{0.5\textwidth}
\centerline{~}
\centerline{\includegraphics[width=0.8\columnwidth]{\Figures/Lectures/HeavyFermion/UPt3FS}}
\begin{center} 
\scriptsize UPt$_3$ [Taillefer and Lonzarich, \\ Phys. Rev. Lett. {\bf 60} 1570 (1988)]
\end{center}

\column{0.5\textwidth}
\centerline{~}
\bi
\setlength{\itemsep}{1em}
%%\item
%%\hl{Quantum oscillation} studies

\item
Magnetic susceptibility (or resistivity, or heat capacity, or ...) \hl{oscillates} as function of \hl{ magnetic field}

\item
Reveal several \hl{Fermi surface sheets}

\item
Temperature dependence of oscillation amplitude $\rightarrow$ \hl{effective mass}.

\item
Measured effective masses consistent with \hl{heat capacity results.}

\ei
\end{columns}

\end{frame}

%%%%%%%%%%%%%%%%%%%%%%%%%%%%%%%%%%%%%%%%%%%%%%%%%%%%%%%%%%%%%%%%%%%%%%
\section{Techniques}
\subsection{Quantum oscillations}
%%%%%%%%%%%%%%%%%%%%%%%%%%%%%%%%%%%%%%%%%%%%%%%%%%%%%%%%%%%%%%%%%%%%%%
\begin{frame}[label=QuantOsc]
\frametitle{High magnetic field: quantum oscillation measurements reveal electronic structure}

\visible<1->{\centerline{\includegraphics[width=0.8\columnwidth]{\Figures/Lectures/QuantOsc/Sr2RuO4Osc}}}


\visible<2->{
\begin{columns}[c]
\column{0.45\textwidth}
\centerline{\includegraphics[width=0.8\columnwidth]{\Figures/Lectures/QuantOsc/DiffStruct}}

\column{0.1\textwidth}

\column{0.45\textwidth}
\centerline{\includegraphics[width=0.8\columnwidth]{\Figures/Lectures/QuantOsc/Sr2RuO4Freq}}

\end{columns}

\begin{columns}[c]
\column{0.45\textwidth}
\centerline{\Large $\downarrow$}

\column{0.1\textwidth}
\column{0.45\textwidth}
\centerline{\Large $\downarrow$}

\end{columns}

\begin{columns}[c]

\column{0.45\textwidth}
\centerline{\includegraphics[width=0.65\columnwidth]{\Figures/Lectures/QuantOsc/StructSRO}}

\column{0.1\textwidth}

\column{0.45\textwidth}
\centerline{\includegraphics[width=0.8\columnwidth]{\Figures/Lectures/QuantOsc/FermiSurface}}
\end{columns}
}
\end{frame}

%%%%%%%%%%%%%%%%%%%%%%%%%%%%%%%%%%%%%%%%%%%%%%%%%%%%%%%%%%%%%%%%%%%%%%
\subsection{Low temperature}
%%%%%%%%%%%%%%%%%%%%%%%%%%%%%%%%%%%%%%%%%%%%%%%%%%%%%%%%%%%%%%%%%%%%%%
\begin{frame}[label=LowTemp]
\frametitle{Low temperatures}

\begin{columns}[t]
\column{0.45\textwidth}
\centerline{~}
\includegraphics[width=\columnwidth]{\Figures/photos/Dilfridge}

\column{0.55\textwidth}
\centerline{~}

Temperature is key control parameter

Methods of attaining low T:

\begin{itemize}
\item
Pumped bath  (0.3 K)
\item
Dilution refrigerator	 (10 mK)
\item
Adiabatic demagnetisation (0.1 mK)
\end{itemize}

Prerequisite for:

\begin{itemize}
\item
subtle many body states (superfluidity, Fractional Quantum Hall Effect, ...)
\item
high sensitivity detection
\item
long decoherence times (quantum computation)
\end{itemize}

\end{columns}
\end{frame}



%%%%%%%%%%%%%%%%%%%%%%%%%%%%%%%%%%%%%%%%%%%%%%%%%%%%%%%%%%%%%%%%%%%%%%
\subsection{Sample synthesis}
%%%%%%%%%%%%%%%%%%%%%%%%%%%%%%%%%%%%%%%%%%%%%%%%%%%%%%%%%%%%%%%%%%%%%%
\begin{frame}[label=CrystalGrowth]
\frametitle{Growing large single crystals}

\centerline{\multiinclude[<visible@+-| +->][graphics={width=\columnwidth},format=pdf]{\Figures/photos/Furnaces/SampleSynth}}
\end{frame}


%%%%%%%%%%%%%%%%%%%%%%%%%%%%%%%%%%%%%%%%%%%%%%%%%%%%%%%%%%%%%%%%%%%%%%
\begin{frame}[label=Susceptibility]
\frametitle{Nonmagnetic piston-cylinder cell for pressures up to 35 kbar}

\includegraphics[width=\columnwidth]{\Figures/Lectures/PistonCylinderCell/CellDiagram.pdf}

\centerline {\scriptsize [see, e.g. I. R. Walker, Rev. Sci. Instrum. {\bf 70} (1999) 3402]}
\end{frame}


%%%%%%%%%%%%%%%%%%%%%%%%%%%%%%%%%%%%%%%%%%%%%%%%%%%%%%%%%%%%%%%%%%%%%%
%\subsection{Pressure}
%%%%%%%%%%%%%%%%%%%%%%%%%%%%%%%%%%%%%%%%%%%%%%%%%%%%%%%%%%%%%%%%%%%%%%
\begin{frame}[label=Pressure ranges]
\frametitle{Pressure -- the new magnetic field}

\centerline{\multiinclude[<visible@+-| +->][graphics={width=\columnwidth},format=pdf]{\Figures/Pressure/PressureDevel}}


\end{frame}



%%%%%%%%%%%%%%%%%%%%%%%%%%%%%%%%%%%%%%%%%%%%%%%%%%%%%%%%%%%%%%%%%%%%%%
\begin{frame}[label=AlirezaCell]
\frametitle{Next step: Sample in diamond anvil cell, SQUID magnetometer}

\begin{columns}[t]
\column{0.45\textwidth}
\vspace{-3em}
\centerline{~}
\centerline{\includegraphics[width=\columnwidth]{\Figures/Lectures/AlirezaCell/MagnCell}}


\column{0.55\textwidth}
\centerline{\hspace{2em}Ca$_2$RuO$_4$} 
\vspace{-1em}

\only<beamer>{
\only<1> { \centerline{\includegraphics[width=0.8\columnwidth]{\Figures/Lectures/AlirezaCell/HystFMCa2RuO4.pdf}}}
}

\only<2- | handout:1> {\centerline{\includegraphics[width=0.8\columnwidth]{\Figures/Lectures/AlirezaCell/MagnFMCa2RuO4.pdf}}}

\end{columns}

\begin{columns}
\column{0.7\textwidth}
\bi
\item
Ultra-high purity BeCu.

\item
Minimise background.

\ei
\centerline{\scriptsize[Patricia Alireza, J. Phys. Soc. Jpn. {\bf 76} (2007) Suppl. A, 216]} 

\column{0.2\textwidth}
\centerline{~}
\centerline{\includegraphics[width=\columnwidth]{\Figures/Lectures/AlirezaCell/PatriciaPhoto}}
\column{0.1\textwidth}

\end{columns}

\end{frame}

%%%%%%%%%%%%%%%%%%%%%%%%%%%%%%%%%%%%%%%%%%%%%%%%%%%%%%%%%%%%%%%%%%%%%%
\begin{frame}[label=DACSuscept]
\frametitle{Susceptibility measurement with pickup coil in diamond anvil cell}
\begin{columns}[t]

\column{0.4\textwidth}
\centerline{~}
\centerline{\includegraphics[width=\columnwidth]{\Figures/Lectures/AlirezaCell/SuscCell1}}

\column{0.6\textwidth}
\centerline{~}
\becbox{0.7}
$V_{induced} = B A N \omega f \chi$
\encbox
\begin{center}
{\scriptsize ($B= $ modulation field, $A =$ area, $N =$ no. turns, \\$\omega = 2\pi\cdot$ modulation frequency, $f =$ filling factor)}
\end{center}

\centerline{Maximise filling factor! ($10^{-7} \rightarrow 0.3$)}
\end{columns}

\begin{columns}[t]
\column{0.4\textwidth}
\centerline{~}
\centerline{\includegraphics[width=\columnwidth]{\Figures/Lectures/AlirezaCell/SuscCellCoilSet}}

\column{0.2\textwidth}
\centerline{~}
\centerline{\includegraphics[width=\columnwidth]{\Figures/Lectures/AlirezaCell/SuscCellCoil}}


\column{0.4\textwidth}
\bi
\item Culet $\sim$ 1 mm.
\item Modulation coil (1), pickup-coil pair (2), compensation coil (3).
\item Al$_2$O$_3$-insulated metal gasket.
\ei

{\scriptsize [Alireza, Rev. Sci. Inst. {\bf 74} (2003) 4728]} 

\end{columns}

\end{frame}

%%%%%%%%%%%%%%%%%%%%%%%%%%%%%%%%%%%%%%%%%%%%%%%%%%%%%%%%%%%%%%%%%%%%%%
\begin{frame}[label=DACdHvA]
\frametitle{High precision susceptibility measurements}
\begin{columns}[t]

\column{0.65\textwidth}
\centerline{Sr$_2$RuO$_4$}
\only<beamer>{
\only<1>{\centerline{\includegraphics[width=\columnwidth]{\Figures/Lectures/AlirezaCell/SuperconSignal}}}}
\only<2- | handout:1> {\centerline{\includegraphics[width=\columnwidth]{\Figures/Lectures/AlirezaCell/dHvASignal.pdf}}}

\centerline{\scriptsize [Goh, Current Applied Physics {\bf 8} (2008) 304]}

\column{0.35\textwidth}
\centerline{~}
\bi
\item<1-> Sr$_2$RuO$_4$ is unconventional superconductor.
\item<1-> Follow $T_c$ vs. pressure.
\item<2-> \hl{Quantum oscillations} (dHvA) observed.
\item<2-> Evolution of Fermi surface and effective masses with pressure.
\ei
 
\end{columns}
\end{frame}
%%\centerline{\includegraphics[width=\columnwidth]{\Figures/Lectures/AlirezaCell/SuscFMCa2RuO4}}

%%%%%%%%%%%%%%%%%%%%%%%%%%%%%%%%%%%%%%%%%%%%%%%%%%%%%%%%%%%%%%%%%%%%%%
\subsection{Super anvils}
%%%%%%%%%%%%%%%%%%%%%%%%%%%%%%%%%%%%%%%%%%%%%%%%%%%%%%%%%%%%%%%%%%%%%%
\begin{frame}[label=SuperAnvils]
\frametitle{Pattern leads into the anvil: Superanvils}

% \only<beamer>{%
% \only<2>{%
% \centerline{\includegraphics[height=0.75\textheight]{\Figures/photos/DesignerAnvils/WiPresentSEM-crop}}
% \centerline{\small{[Susceptibility pickup coil, CVD diamond, Lawrence Livermore]}}
% }

% \only<1>{%
% \centerline{\includegraphics[angle=90,height=0.75\textheight]{\Figures/photos/DesignerAnvils/Edinburgh1-crop}}
% \centerline{\small{[K. Kamenev, CSEC Edinburgh]}}
% }
% }

\only<1-| handout:1>{%
\begin{columns}[t]
\column{0.5\textwidth}
\centerline{~}
\centerline{\includegraphics[height=0.5\textheight]{\Figures/photos/DesignerAnvils/WiPresentSEM-crop}}

\column{0.5\textwidth}
\centerline{~}
\centerline{Erbium}
\centerline{\includegraphics[height=0.5\textheight]{\Figures/photos/DesignerAnvils/ErbiumSusc}}
\end{columns}
\vspace{1em}
\centerline{\small[Jackson, Phys. Rev. B {\bf 71} (2005) 184416]}
}

\bi
\item
  Pattern leads \hl{into the anvil.}
\item
   \hl{Reliability,} reproducibility, mass production.
\item
  Requires \hl{microlithography,} chemical vapour deposition (diamonds),
laser ablation (Al$_2$O$_3$).  
% \item
%   heat capacity, thermal expansion, thermo\-power, resistivity,
%   magnetisation, quantum oscillations, ...
\ei

\end{frame}


%%%%%%%%%%%%%%%%%%%%%%%%%%%%%%%%%%%%%%%%%%%%%%%%%%%%%%%%%%%%%%%%%%%%%%
\section{Exploration}
\subsection{Materials}
%%%%%%%%%%%%%%%%%%%%%%%%%%%%%%%%%%%%%%%%%%%%%%%%%%%%%%%%%%%%%%%%%%%%%
\begin{frame}[label=Diversity]

\frametitle{Plenty of room in materials space}
\only<beamer>{%
\only<1>{%
\vspace{2em}
\centerline{\includegraphics[angle=90,width=0.9\columnwidth]{\Figures/Diversity/PerSysen.jpg}}
}
}
\pause
\centerline{\multiinclude[<visible@+-| +->][graphics={width=0.9\columnwidth},format=pdf]{\Figures/Diversity/Diversity}}

\end{frame}


%%%%%%%%%%%%%%%%%%%%%%%%%%%%%%%%%%%%%%%%%%%%%%%%%%%%%%%%%%%%%%%%%%%%%
\subsection{Quantum phase transitions}
%%%%%%%%%%%%%%%%%%%%%%%%%%%%%%%%%%%%%%%%%%%%%%%%%%%%%%%%%%%%%%%%%%%%%
\begin{frame}[label=phasedias]
\frametitle{\small Plenty of variety at the bottom: tuning leads to discovery}
\vspace{-1.ex}
\centerline{%
\includegraphics[width=0.9\columnwidth,clip=on]{\Figures/Phasedias/phasedias08}}
\end{frame}


\only<beamer>{%
\appendix

%%%%%%%%%%%%%%%%%%%%%%%%%%%%%%%%%%%%%%%%%%%%%%%%%%%%%%%%%%%%%%%%%%%%%%
\begin{frame}[label=QuanPhase]
\frametitle{Quantum phase transitions}
\begin{columns}[t]
\column{0.5\textwidth}

\hl{Phase diagram}
\centerline{\multiinclude[<visible@+-| +->][graphics={width=0.9\columnwidth},format=pdf]{\Figures/QuanPhase/QuanPhase}}

\column{0.5\textwidth}
\begin{itemize}
\item<1->
Low temperature state (blue), e.g. \hl{magnetism.} 
\item<1->
\hl{Ordered state melts} with increasing temperature (thermal fluct.).
\item<1->
Quantum control parameter $x_1$, $x_2$
(e.g. \hl{pressure}, magnetic field, composition).
\item<1->
Ordered state melts \hl{at $T=0$,} as function of $x_1$, $x_2$ (quantum fluctuations).

\item<2->
New order (red), e.g. \hl{superconductivity.}
\item<2->
Recipe for discoveries?!

\end{itemize}
\end{columns}
\end{frame}


%%%%%%%%%%%%%%%%%%%%%%%%%%%%%%%%%%%%%%%%%%%%%%%%%%%%%%%%%%%%%%%%%%%%%%
\subsection{Quantum criticality}
%%%%%%%%%%%%%%%%%%%%%%%%%%%%%%%%%%%%%%%%%%%%%%%%%%%%%%%%%%%%%%%%%%%%%
\begin{frame}[label=Quancrit]
\frametitle{Quantum critical point}
%% \begin{columns}[t]
%% \column{0.68\textwidth}
%% \centerline{~}

\centerline{\includegraphics[width=0.85\textwidth]{\Figures/QuanCrit/QuanPhaseDiaEn}}

%% \column{0.32\textwidth}
%% \centerline{~}
%% \pause
%% \includegraphics[width=\columnwidth]{\Figures/QuanCrit/QuanCritCone1}
%% \\ \\
%% %%\pause
%% \includegraphics[width=\columnwidth]{\Figures/QuanCrit/QuanCritCone2}

%% \end{columns}

\end{frame}


%%%%%%%%%%%%%%%%%%%%%%%%%%%%%%%%%%%%%%%%%%%%%%%%%%%%%%%%%%%%%%%%%%%%%%
\subsection{Quantum criticality}
%%%%%%%%%%%%%%%%%%%%%%%%%%%%%%%%%%%%%%%%%%%%%%%%%%%%%%%%%%%%%%%%%%%%%%
\begin{frame}[label=quancrit3]
\frametitle{Quantum criticality}
\begin{columns}[t]
\column{0.68\textwidth}
\centerline{~}
\includegraphics[width=\columnwidth]{\Figures/QuanCrit/QuanPhaseDiaEn}

\column{0.32\textwidth}
\centerline{~}
%%\pause
\centerline{\includegraphics[width=\columnwidth]{\Figures/QuanCrit/QuanCritCone1}}

%%\pause
\centerline{\includegraphics[width=\columnwidth]{\Figures/QuanCrit/QuanCritCone2}}

\end{columns}

\end{frame}


%%%%%%%%%%%%%%%%%%%%%%%%%%%%%%%%%%%%%%%%%%%%%%%%%%%%%%%%%%%%%%%%%%%%%%
\begin{frame}[label=CCSNormal1]{Unconventional normal state}
  \vspace{-1ex}
  \begin{columns}[t]
    \column{0.5\textwidth}
    \centerline{~}
    \onslide+<2->{\includegraphics[width=0.9\columnwidth]{\Figures/cps/cpsNormal}}

    \column{0.5\textwidth}
    \centerline{~}
    \vspace{-2ex}
    \begin{itemize}
    \item
      Heat cap. $C/T \neq ~{const.}$
    \item
      Resist. $\rho-\rho_0 \sim T^x$, $x<2$.
    \end{itemize}

    \centerline{Theory: asymptotic low-$T$ form}
    \vspace{-0.5ex}
    {\scriptsize %
    \begin{tabular}[t]{|l|c|c|c|c|}
      \hline
      & FL & 3D AF  & 2D AF & 3D F \tabularnewline
      \hline
      $C/T$ & $\gamma$  & $\gamma - \sqrt T$ & $\log T$ & $\log T$\tabularnewline
      \hline
      $\Delta \rho$ &  $T^2$ & $T^{3/2}$ & $T^1$ & $T^{5/3}$ \tabularnewline 
      \hline
    \end{tabular} }

    \vspace{3.5ex} 
    \onslide+<3->{\includegraphics[width=\columnwidth]{\Figures/yrs/yrscrit}}
  \end{columns}
\end{frame}



%%%%%%%%%%%%%%%%%%%%%%%%%%%%%%%%%%%%%%%%%%%%%%%%%%%%%%%%%%%%%%%%%%%%%%
\subsection{Novel states of matter}
%%%%%%%%%%%%%%%%%%%%%%%%%%%%%%%%%%%%%%%%%%%%%%%%%%%%%%%%%%%%%%%%%%%%%
\begin{frame}[label=NewStates]
\frametitle{Novel states of matter}

Low temperature order \hl{(condensates):}

\begin{itemize}
\item<1->
Non-cuprate high-$T_c$ superconductors: metallise Mott insulators.

\item<2->
Finite angular momentum ($\ell>0$) particle-hole pairing: search at
threshold of magnetic order, long-range quasiparticle interaction.

\item<3->
Orbital order, valence transitions, Fermionic condensates, ...

\end{itemize}
\begin{columns}[t]
\onslide<4->
\column{0.4\columnwidth}
Electron liquids (in particular \hl{non-Fermi liquids):}

\begin{itemize}
\item
Local criticality: two-band systems, Yb compounds.

\item
Spin liquids: frustrated magnets.

\end{itemize}

\onslide<2->
\column{0.6\columnwidth}
\centerline{~}
\begin{tabular*}{\columnwidth}{|p{0.2 \columnwidth}|@{\extracolsep{\fill}}p{0.3 \columnwidth} | p{0.3 \columnwidth}|}
\hline
& p-p & p-h \tabularnewline
\hline
$\ell=0$ & BCS superconductor & Ferro\-magnet \tabularnewline
\hline
$\ell=2$ & High-$T_c$ super\-conductor & \hl{?!}\tabularnewline
\hline
\end{tabular*}
\end{columns}

\end{frame}


%%%%%%%%%%%%%%%%%%%%%%%%%%%%%%%%%%%%%%%%%%%%%%%%%%%%%%%%%%%%%%%%%%%%%%
\begin{frame}[label=Susceptibility]
\frametitle{Susceptibility measurements in piston-cylinder cells}
\begin{columns}[b]
\column{0.5\textwidth}
\centerline{~}

\includegraphics[angle=-90,width=1.3\columnwidth]{\Figures/Lectures/PistonCylinderCell/SuscSetup.pdf}

\centerline{~}

\column{0.5\textwidth}
\centerline{~}
\raggedleft{ \includegraphics[angle=90,width=0.25\columnwidth]{\Figures/Lectures/PistonCylinderCell/SuscPhoto.pdf}\hspace{5em}}

\vspace{1em}
\bi
\item Modulation field $\sim 1$ G at $\sim 20$ Hz.
\item Resolution $\Delta \chi \sim 10^{-6}$  (SI).
\item Cancellation $\sim 10^{-3}$ by adjusting pick-up coil.
\ei
\centerline{~}

\end{columns}

\end{frame}




%%%%%%%%%%%%%%%%%%%%%%%%%%%%%%%%%%%%%%%%%%%%%%%%%%%%%%%%%%%%%%%%%%%%%%
\subsection{Focussed Ion Beam}
%%%%%%%%%%%%%%%%%%%%%%%%%%%%%%%%%%%%%%%%%%%%%%%%%%%%%%%%%%%%%%%%%%%%%%
\begin{frame}[label=FIB1]
\frametitle{Sample-handling on the microscale}

\only<beamer>{%

\only<1>{%
\centerline{\includegraphics[height=0.8\textheight]{\Figures/Pressure/FIB/BeforeContacts.jpg}}
\centerline{\small{Electron microscope, Cambridge}}
}

\only<2>{%
\centerline{\includegraphics[height=0.8\textheight]{\Figures/Pressure/FIB/FirstContacts.jpg}}
\centerline{\small{Focussed ion beam, platinum deposition}}
}

\only<3>{%
\centerline{\includegraphics[height=0.8\textheight]{\Figures/Pressure/FIB/FirstContactsDetail.jpg}}
\centerline{\small{Platinum deposition, detail}}
}

\only<4>{%
\centerline{\includegraphics[height=0.8\textheight]{\Figures/Pressure/FIB/FirstContactsProblem.jpg}}
\centerline{\small{Interrupted leads due to gap}}
}

\only<5>{%
\centerline{\includegraphics[height=0.8\textheight]{\Figures/Pressure/FIB/SecondContact.jpg}}
\centerline{\small{Bridge the gap with vertical stubs}}
}
}
\only<6-| handout:1>{%
%%\includegraphics[height=0.33\textheight]{\Figures/Pressure/FIB/BeforeContacts.jpg}
%%\hfill
\includegraphics[height=0.45\textheight]{\Figures/Pressure/FIB/FirstContacts.jpg}
\hfill
\includegraphics[height=0.45\textheight]{\Figures/Pressure/FIB/SecondContact.jpg}
\vspace{1.5em}
\bi
\item
\hl{Focussed ion beam:} Image, shape and deposit.
\item
\hl{Micropositioning} (e.g. Omniprobe).
\item
Bridging \hl{gaps} is possible, but causes high resistance.
\ei
}
\end{frame}

%%%%%%%%%%%%%%%%%%%%%%%%%%%%%%%%%%%%%%%%%%%%%%%%%%%%%%%%%%%%%%%%%%%%%%
\subsection{Relaxation rates}
%%%%%%%%%%%%%%%%%%%%%%%%%%%%%%%%%%%%%%%%%%%%%%%%%%%%%%%%%%%%%%%%%%%%%%
\begin{frame}[label=quancrit1]
\frametitle{Quantum criticality-1}

\begin{columns}[t]
\column{0.6\textwidth}
\centerline{~}

Fluctuating \hl{local order parameter} $m(\vec r, \vec t)$, response
function (susceptibility)
\[ \chi_{\vec q \omega} = \chi_{\vec q} \frac{1}{1-i\omega/\Gamma_\vec q} \]

\begin{itemize}
\item Relaxation rate $\Gamma_\vec q \sim {\chi_\vec q}^{-1}$
\item $\Gamma_\vec q \rightarrow 0$ at $T_N$, $\bf q=\vec Q$
\end{itemize}

\column{0.4\textwidth}
\centerline{~}
\includegraphics[width=\columnwidth,clip=on]{\Figures/QuanCrit/GammaQ-0}
\end{columns}

\vspace{4ex}
\centerline{\hl{Power spectrum} of $m$:
$\langle |m_{\vec q\omega}|^2 \rangle = \left(n_\omega + \frac{1}{2}\right) Im(\chi_{\vec q \omega})$}

\end{frame}


%%%%%%%%%%%%%%%%%%%%%%%%%%%%%%%%%%%%%%%%%%%%%%%%%%%%%%%%%%%%%%%%%%%%%%
\subsection{Thermal excitation}
%%%%%%%%%%%%%%%%%%%%%%%%%%%%%%%%%%%%%%%%%%%%%%%%%%%%%%%%%%%%%%%%%%%%%%
\begin{frame}[label=quancrit2]{Quantum criticality-2}

\begin{columns}[t]
  \column{0.6\textwidth}
  \centerline{~}
  \[ \chi_{\vec q \omega} = \chi_{\vec q} \frac{1}{1-i\omega/\Gamma_\vec q} \]

  \[ \langle |m_{\vec q\omega}|^2 \rangle = \left(n_\omega + \frac{1}{2}\right) \Im(\chi_{\vec q \omega}) \]

\begin{itemize}
\item<2->
  Integrals over $\vec q$ and $\omega$, e.g.
\end{itemize}

  \column{0.4\textwidth}
  \centerline{~}
  \multiinclude[<visible@+- | +->][graphics={width=\columnwidth,clip=on},format=pdf]{\Figures/QuanCrit/GammaQ}

\end{columns}
\visible<2->{\[ \langle m(\vec r) m(0) \rangle \propto \int{d^3 \vec q \chi_\vec q e^{-i \vec q \vec r}} \int {d\omega \left(n_\omega + \frac {1}{2}\right) \frac{\omega/\Gamma_\vec q}{1+\omega^2/\Gamma_\vec q^2}} \]

\raggedleft{(Bose-factor $n_\omega \rightarrow \frac{k_B T}{\hbar\omega}$ if $k_B
T >> \hbar \omega$)}}

\begin{itemize}
\item<3->
  \hl{Classical} critical behaviour: $k_B T >> \hbar \Gamma_\vec q$ for
  critical modes $\vec q$.  Integrals separable, frequency integral
  gives $k_B T$ (\hl{equipartition}). Only static susc.  $\chi_\vec q$
  survives.  

\item<4->
  \hl{Quantum critical} behaviour: $k_B T$ falls inside dispersion
  $\Gamma_\vec q$.  Dynamics (via $\Gamma_\vec q$) influence result.
\end{itemize}

\end{frame}
}

%%% Local variables: 
%%% mode: latex
%%% TeX-master: "Grosche2"
%%% End: 
