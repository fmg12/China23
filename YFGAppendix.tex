
%%%%%%%%%%%%%%%%%%%%%%%%%%%%%%%%%%%%%%%%%%%%%%%%%%%%%%%%%%%%%%%%%%%%%%%
%\subsection{New superconductors}
%%%%%%%%%%%%%%%%%%%%%%%%%%%%%%%%%%%%%%%%%%%%%%%%%%%%%%%%%%%%%%%%%%%%%%%
%\begin{frame}[label=LaH10]
%\frametitle{Record $T_c$ in high-pressure LaH$_{10}$}
%\centerline{\includegraphics[width=\textwidth]{\Figures/Phasedias/UnconvSupercon/LaH10}}
%%\vspace{0em}
%
%\begin{center}
%$T_c$ extends up to \SI{283}{\kelvin}, {\bf maybe much higher}
%
%\end{center}
%\small{
%\begin{itemize}
%%Success of computationally assisted materials discovery.
%\item
%Can LaH$_{10}$-style superconductivity extend to ambient pressure?
%\item
%Probably not; such high phonon frequencies require compression.
%\item
%\textcolor{red}{Need alternative pairing interaction with high energy scale:} \\
%\textcolor{red}{\bf magnetic interaction in metals reaches
%  up to 3,000 K.}
%\end{itemize}}
%\end{frame}
%
%%%%%%%%%%%%%%%%%%%%
\begin{frame}[label=KFA-YFG-Motivation]
\frametitle{From KFe$_2$As$_2$ to YFe$_2$Ge$_2$}
\centerline{\includegraphics[width=\textwidth]{\data/yfe2ge2/FiguresYFG/Overview/KFA-YFG-Motivation-crop}}

\vspace{2em}

\begin{itemize}
\item
Search for analogues to Fe-As superconductors.

\item
KFe$_2$As$_2$ ($T_c \simeq 3.8 ~\text{K}$): high $C/T \simeq 100
~\text{mJ/mol K}^{2}$.

\item
YFe$_2$Ge$_2$ has similarly high $C/T$, apparently same Fe oxidation
number as in
KFe$_2$As$_2$.
\end{itemize}
\centerline{\hl{Origin of high $C/T$?}}
%\centerline{among the highest in transition metal compounds?}

\vspace*{\fill}
% \vspace{1.5em}
\centerline{\makebox[\linewidth]{\rule{0.85\textwidth}{0.4pt}}}
\centerline{\scriptsize Reid Supercond. Sci. Technol. {\bf 25,} 084013 (2012), Avila JMMM {\bf 270,} 51 (2004)}
\end{frame}


%%%%%%%%%%%%%%%%%%%%%%%%%%%%%%%%%%%%%%%%%%%%%%%%%%%%%%%%%%%%%%%%%%%%%%
\subsection{The route to YFe$_2$Ge$_2$}
%%%%%%%%%%%%%%%%%%%%%%%%%%%%%%%%%%%%%%%%%%%%%%%%%%%%%%%%%%%%%%%%%%%%%%
\begin{frame}[label=YFGIntro1]
\frametitle{From narrow-band, $f$-electron low-$T_c$ to $d$-electron high-$T_c$}
\centerline{\includegraphics[width=\columnwidth]{\Figures/PhaseDias/CePd2Si2/cpsbfaComparison}}

\vspace{1em}
\centerline{We need more unconventional superconductors!}

\vspace*{\fill}
%\vspace{3.5em}
\centerline{\makebox[\linewidth]{\rule{0.85\textwidth}{0.4pt}}}
\centerline{\scriptsize Mathur Nature {\bf 394,} 39 (1998), Hashimoto
  Science {\bf 336,} 1554 (2012)}
\end{frame}



%%%%%%%%%%%%%%%%%%%%%%%%%%%%%%%%%%%%%%%%%%%%%%%%%%%%%%%%%%%%%%%%%%%%%%
\subsection{The (Lu/Y)Fe$_2$Ge$_2$ system}
%%%%%%%%%%%%%%%%%%%%%%%%%%%%%%%%%%%%%%%%%%%%%%%%%%%%%%%%%%%%%%%%%%%%%%
\begin{frame}[label=EarlyWork]
\frametitle{Early work on (Y/Lu)Fe$_2$Ge$_2$}
\centerline{\includegraphics[width=\textwidth]{\data/yfe2ge2/FiguresYFG/Overview/YFGEarlyWork}}
\begin{itemize} %\itemsep 8pt
\item
Antiferromagnetic transition in LuFe$_2$Ge$_2$ near 10 K.

% Unusually high $\gamma = C/T$ of about 100 mJ/mol K$^2$.
% \item
% $V_0 ($YFe$_2$Ge$_2) = 164.8 \Ang^3$, $V_0 ($LuFe$_2$Ge$_2) =
% 159.3 \Ang^3$ 

% \hfill $\implies \Delta V_0/V_0 \simeq 3.5 \%$
% (corresponds to about 4 GPa).
%\item Require around 3 GPa to tune YFe$_2$Ge$_2$ to qcp.

\item
Magnetic susceptibility $\chi_{SI} \sim 10^{-3}$. \\ Moderately enhanced Wilson ratio $R_W \simeq 2.5$.

% \item
% Fe oxidation state appears same as in KFe$_2$As$_2$ ($+2.5$).

\end{itemize}
\end{frame}

%%%%%%%%%%%%%%%%%%%%%%%%%%%%%%%%%%%%%%%%%%%%%%%%%%%%%%%%%%%%%%%%%%%%%%


%%%%%%%%%%%%%%%%%%%%%%%%%%%%%%%%%%%%%%%%%%%%%%%%%%%%%%%%%%%%%%%%%%%%%
\begin{frame}[label=YFGIntro2]
  \frametitle{Magnetism and enhanced $C/T$ in  (Lu/Y)Fe$_2$Ge$_2$}
  \centerline{\includegraphics[width=\columnwidth]{\data/yfe2ge2/FiguresYFG/YFGLFGCompare-crop.pdf}}
  \begin{itemize} %\itemsep 8pt
  \item LuFe$_2$Ge$_2$ transition at $T_N \sim \SI{10}{\kelvin}$, YFe$_2$Ge$_2$ no magnetic order
  \item
  Unusually high $\gamma = C/T$ of about 100 mJ/mol K$^2$.
  \item
  $V_0 ($YFe$_2$Ge$_2) = 164.8 \Ang^3$, $V_0 ($LuFe$_2$Ge$_2) =
  159.3 \Ang^3$ 
  
  \hfill $\implies \Delta V_0/V_0 \simeq 3.5 \%$
  (corresponds to about 4 GPa).
  %\item Require around 3 GPa to tune YFe$_2$Ge$_2$ to qcp.
  
  \item
  Magnetic susceptibility $\chi_{SI} \sim 10^{-3}$. \\ Moderately enhanced Wilson ratio $R_W \simeq 2.5$.
  
  %\item
  %Fe oxidation state appears same as in KFe$_2$As$_2$ ($+2.5$).
  
  \end{itemize}
  \end{frame}

%%%%%%%%%%%%%%%%%%%%%%%%%%%%%%%%%%%%%%%%%%%%%%%%%%%%%%%%%%%%%%%%%%%%%%
\subsection{Neutrons}
%%%%%%%%%%%%%%%%%%%%%%%%%%%%%%%%%%%%%%%%%%%%%%%%%%%%%%%%%%%%%%%%%%%%%%
\begin{frame}[label=YFGNeutrons]
\frametitle{Neutron scattering in YFe$_2$Ge$_2$}
\centerline{\includegraphics[width=\columnwidth]{\data/yfe2ge2/FiguresYFG/Neutrons/Wo19}}

\end{frame}



%%%%%%%%%%%%%%%%%%%%%%%%%%%%%%%%%%%%%%%%%%%%%%%%%%%%%%%%%%%%%%%%%%%%%%
\subsection{First-generation samples}
%%%%%%%%%%%%%%%%%%%%%%%%%%%%%%%%%%%%%%%%%%%%%%%%%%%%%%%%%%%%%%%%%%%%%%
\begin{frame}[label=YFGFirstGen]
\frametitle{Superconductivity in YFe$_2$Ge$_2$, first-generation samples}
\centerline{\includegraphics[width=\columnwidth]{\data/yfe2ge2/FiguresYFG/SampleSummaries/FirstGenSamples/FirstGenSamplesSummary}}
\begin{itemize}
\item
Polycrystals (induction-furnace) or single crystals (flux).
\item
Resistance ratios $< 60$. 
\item
Resistive transitions but no heat capacity anomaly at $T_c$.
\end{itemize}

\vspace*{\fill}
% \vspace{1.5em}
\centerline{\makebox[\linewidth]{\rule{0.85\textwidth}{0.4pt}}}
\centerline{\scriptsize [Zou Phys Status Solidi (RRL) {\bf 8,} 928
  (2014); Kim Phil. Mag. {\bf 95,} 804 (2015)]}
\end{frame}


%%%%%%%%%%%%%%%%%%%%%%%%%%%%%%%%%%%%%%%%%%%%%%%%%%%%%%%%%%%%%%%%%%%%%%
\subsection{Growth studies}
%%%%%%%%%%%%%%%%%%%%%%%%%%%%%%%%%%%%%%%%%%%%%%%%%%%%%%%%%%%%%%%%%%%%%%
\begin{frame}[label=YFGSecondGen]
\frametitle{Growth improvements lead to bulk superconductivity in polycrystals}
\centerline{\includegraphics[width=\columnwidth]{\data/yfe2ge2/FiguresYFG/SampleSummaries/SecondGenSamples/SecondGenSamplesOverview}}


\end{frame}

%%%%%%%%%%%%%%%%%%%%%%%%%%%%%%%%%%%%%%%%%%%%%%%%%%%%%%%%%%%%%%%%%%%%%%
\begin{frame}[label=YFGSecondGen]
\frametitle{Growth study in YFe$_2$Ge$_2$ polycrystals}
% \framezoom<1><2>(0cm,0cm)(2cm,1.5cm)
% \framezoom<1><3>(1cm,3cm)(2cm,1.5cm)
% \framezoom<1><4>(3cm,2cm)(3cm,2cm)
%\TPGrid{10}{10}
\centerline{\includegraphics[width=\columnwidth]{\data/yfe2ge2/FiguresYFG/SampleSummaries/SecondGenSamples/GrowthStudyResults.png}}
\begin{textblock}{4}[0.5, 0.5](7, -6.5)
\visible<2->{
\begin{beamercolorbox}{postit}
\begin{center}
\textcolor{red}{\small Fe-rich melt\\
ensures full \\occupation of Fe-site}
\end{center}
\end{beamercolorbox}
}
\end{textblock}
\end{frame}




%%%%%%%%%%%%%%%%%%%%%%%%%%%%%%%%%%%%%%%%%%%%%%%%%%%%%%%%%%%%%%%%%%%%%%
\subsection{High purity YFe$_2$Ge$_2$}
%%%%%%%%%%%%%%%%%%%%%%%%%%%%%%%%%%%%%%%%%%%%%%%%%%%%%%%%%%%%%%%%%%%%%%
\begin{frame}[label=YFGThirdGen]
\frametitle{Towards high quality single crystals of YFe$_2$Ge$_2$}

\begin{columns}[T]
\column{0.7\textwidth}
\begin{itemize}
\item
{\small Sn flux growth from high quality polycrystals (grown from Fe-rich melt).

\item
Experimenting with temperature profiles, crucible orientation, growth protocol has gradually produced \hl{RRR $\mathbf{\sim 500}$.}

\item
Sharp bulk transitions. Resistivity still $T^{3/2}$. }
\end{itemize}

\column{0.3\textwidth}
\centerline{\includegraphics[width=\columnwidth]{\data/yfe2ge2/FiguresYFG/SampleSummaries/GrowthSpring19/SamplesPhoto}}
\end{columns}

\vspace{1em}
{\includegraphics[width=1.02\columnwidth]{\data/yfe2ge2/FiguresYFG/SampleSummaries/GrowthSpring19/HeatCapResistivity}}


\end{frame}

%%%%%%%%%%%%%%%%%%%%%%%%%%%%%%%%%%%%%%%%%%%%%%%%%%%%%%%%%%%%%%%%%%%%%%
\subsection{Crystal growth}
%%%%%%%%%%%%%%%%%%%%%%%%%%%%%%%%%%%%%%%%%%%%%%%%%%%%%%%%%%%%%%%%%%%%%%
\begin{frame}[plain,label=YFGBulkSupercon]
\frametitle{Bulk superconductivity in high quality single crystals of YFe$_2$Ge$_2$}
\includegraphics[width=1.2\textwidth]{\data/yfe2ge2/FiguresYFG/GrowthHorizontalFlux2019/BulkSupercon}
\end{frame}


%%%%%%%%%%%%%%%%%%%%%%%%%%%%%%%%%%%%%%%%%%%%%%%%%%%%%%%%%%%%%%%%%%%%%%
\subsection{Heat capacity}
%%%%%%%%%%%%%%%%%%%%%%%%%%%%%%%%%%%%%%%%%%%%%%%%%%%%%%%%%%%%%%%%%%%%%%
\begin{frame}[label=YFGHeatCapKFA]
\frametitle{Heat capacity to very low temperatures}

\centerline{\includegraphics[width=\columnwidth]{\data/yfe2ge2/FiguresYFG/HeatCap/Dresden0519/HCYFGKFACompare}}
\begin{itemize}
\item
Dilution fridge measurements confirm sharp bulk transition in new
generation of YFe$_2$Ge$_2$ single crystals.
\item
Upturn at low temperature (nuclear contribution) prevents definitive
conclusion about gap structure and residual $\gamma$.
\item
Similarity with KFe$_2$As$_2$. 
\end{itemize}

\end{frame}


%%%%%%%%%%%%%%%%%%%%%%%%%%%%%%%%%%%%%%%%%%%%%%%%%%%%%%%%%%%%%%%%%%%%%%
\subsection{muSR}
%%%%%%%%%%%%%%%%%%%%%%%%%%%%%%%%%%%%%%%%%%%%%%%%%%%%%%%%%%%%%%%%%%%%%%
\begin{frame}[label=YFGmuSRIntro]
\frametitle{Transverse field muSR measurements}
\[
A_{TF} = A_0 \exp(-(\sigma t)^2/2) \cos\left(\gamma \mu
  \langle  B\rangle t+\phi\right )
+ A_{bg} \cos\left(\gamma \mu B_{bg} t + \phi \right)
\]
\centerline{\includegraphics[width=0.9\columnwidth]{\data/yfe2ge2/FiguresYFG/muSR2019/muSRtDependence}}
\centerline{First attempt at ISIS with Pabitra Biswas and Adroja Devashibhai}
\end{frame}


%%%%%%%%%%%%%%%%%%%%%%%%%%%%%%%%%%%%%%%%%%%%%%%%%%%%%%%%%%%%%%%%%%%%%%
\begin{frame}[label=YFGmuSRSummary]
\frametitle{Transverse field muSR}

\centerline{\includegraphics[width=\columnwidth]{\data/yfe2ge2/FiguresYFG/muSR2019/muSRResultSummary}}
\end{frame}

%%%%%%%%%%%%%%%%%%%%%%%%%%%%%%%%%%%%%%%%%%%%%%%%%%%%%%%%%%%%%%%%%%%%%%
%\subsection{Comparison to KFe$_2$As$_2$}
\subsection{Fermi surface structure}
%\subsection{$c-$axis contraction}
%%%%%%%%%%%%%%%%%%%%%%%%%%%%%%%%%%%%%%%%%%%%%%%%%%%%%%%%%%%%%%%%%%%%%%
\begin{frame}[label=CollapsedTetragonala]
\frametitle{High pressure KFe$_2$As$_2$ similar to YFe$_2$Ge$_2$}

\centerline{\includegraphics[width=0.95\columnwidth]{\data/yfe2ge2/FiguresYFG/FermiSurface/CompareKFA-YFG4}}

%\becbox{0.9}
\begin{center}
YFe$_2$Ge$_2$ Fermi surface resembles that of
  KFe$_2$As$_2$ in \hl{collapsed tetragonal} phase.
\end{center}
%\encbox
\vspace{-2em}
{\small 
\begin{columns}[c]
\column{0.65\textwidth}
\begin{center}
\begin{tabular}{l|c|c}
%\begin{tabular}{|l|D{.}{.}{2.3}|D{.}{.}{2.3}|}
RFe$_2$X$_2$ & \multicolumn{1}{c|}{$c/a$} &\multicolumn{1}{c}{X-X dist. (\AA)} \\
\hline
uct KFe$_2$As$_2$ ($p = 0$) & 3.608 & 4.089  \\
ct KFe$_2$As$_2$ (21 GPa) &  2.491 & 2.544 \\
YFe$_2$Ge$_2$ & 2.639 & 2.721 \\
%Covalent bond & & 2.46 \\
\hline
\end{tabular}
\end{center}

\column{0.35\textwidth}
~\vspace{2em}
\begin{center}
Ge-Ge or As-As bond: \\
Fe $d^{5.5} \rightarrow d^{6.5}$
\end{center}
\end{columns}
}
\end{frame}


%%%%%%%%%%%%%%%%%%%%%%%%%%%%%%%%%%%%%%%%%%%%%%%%%%%%%%%%%%%%%%%%%%%%%%
\begin{frame}[label=YFGFermiSurface]
\frametitle{YFe$_2$Ge$_2$ calculated Fermi surface sheets}

\begin{columns}[T]
\column{0.55\textwidth}
\centerline{~}
\centerline{\includegraphics[width=\columnwidth]{\data/yfe2ge2/FiguresYFG/YFGFS.pdf}}

\column{0.45\textwidth}
\centerline{\includegraphics[width=0.9\columnwidth]{\data/yfe2ge2/FiguresYFG/QO/QOTable}}
\centerline{(our Wien2k calc.)}
\vspace{0.5em}

{\small DFT predicts $\gamma \simeq 16 ~\text{mJ/molK}^2$ vs. measured
$100~\text{mJ/molK}^2$. \\
\vspace{1.5em}
Expect measured masses of order $10~m_e$.
}

% \begin{itemize} \itemsep 20pt
% \item
% Dominant feature: large hole pocket (4) enclosing face of BZ.

% \item
% Plus cylindrical electron pocket in corner of BZ.

% \item
% Very different from standard iron-arsenide case.


% \item short c-axis causes 3D FS.

% \item
% Covalent Ge-Ge bonds $\rightarrow$ Fe oxidation state differs from
% that in KFe$_2$As$_2$.

%\end{itemize}

\end{columns}



\vspace*{\fill}
%\vspace{3.5em}
\centerline{\makebox[\linewidth]{\rule{0.85\textwidth}{0.4pt}}}
\centerline{\scriptsize[Singh PRB {\bf 89,} 024505, Subedi PRB {\bf
    89,} 024504 (2014)]}
\end{frame}



%%%%%%%%%%%%%%%%%%%%%%%%%%%%%%%%%%%%%%%%%%%%%%%%%%%%%%%%%%%%%%%%%%%%%%
\begin{frame}[label=YFGQO0519]
\frametitle{May 2019 quantum oscillation data}

\centerline{\includegraphics[width=\columnwidth]{\data/yfe2ge2/FiguresYFG/QO/QO0519Results}}
\end{frame}


%%%%%%%%%%%%%%%%%%%%%%%%%%%%%%%%%%%%%%%%%%%%%%%%%%%%%%%%%%%%%%%%%%%%%%
\begin{frame}[label=TorqueInteraction]
\frametitle{Origin of the nonlinear torque interaction}

\begin{itemize}
\item For torque $\tau$, cantilever floppiness $\alpha$, unperturbed rotation angle $\theta_0$
\end{itemize}
\[
\tau(\theta)=\tau(\theta_0+\alpha\tau) \simeq \tau(\theta_0)[1+\alpha \partial \tau/\partial\theta]
\]

\begin{itemize}
\item 
Nonlinear term $\tau \partial \tau/\partial\theta$ mixes oscillations at different frequencies $F_i$, if QO frequencies themselves depend strongly on rotation angle:
\end{itemize}
\begin{eqnarray*}
    \tilde\tau(\theta)&=&\sum_i R_i \sin\left(\frac{2\pi F_i}{B}\right) + \nonumber \\
    && \frac{\alpha\pi}{B}\sum_{ij} R_i R_j \frac{\partial F_j}{\partial \theta}\sin\left(\frac{2\pi}{B}(F_i+F_j)\right) + \nonumber \\
    && \frac{\alpha\pi}{B}\sum_{ij} R_i R_j \frac{\partial F_j}{\partial \theta}\sin\left(\frac{2\pi}{B}(F_i-F_j)\right) \quad .
\end{eqnarray*}

\vspace*{\fill}
%\vspace{3.5em}
\centerline{\makebox[\linewidth]{\rule{0.85\textwidth}{0.4pt}}}
\begin{center}
{\scriptsize[e.g. Shoenberg, Magnetic Oscillations in Metals, 1st ed. (Cambridge University Press, Cambridge, 1988), J. Vanderkooy and W. R. Datars, Can. J. Phys. 46, 1215 (1968)]}
\end{center}


\end{frame}


%%%%%%%%%%%%%%%%%%%%%%%%%%%%%%%%%%%%%%%%%%%%%%%%%%%%%%%%%%%%%%%%%%%%%%
\subsection{KFe$_2$As$_2$ / YFe$_2$Ge$_2$}
%%%%%%%%%%%%%%%%%%%%%%%%%%%%%%%%%%%%%%%%%%%%%%%%%%
%%%%%%%%%%%%%%%%%%%%%%%%%%%%%%%%%%%%%%%%%%%%%%%%%%%%%%%%%%%%%%%%%%%%%%
\begin{frame}[label=YFGIntro2b]
\frametitle{Magnetic quantum phase transition in (Lu/Y)Fe$_2$Ge$_2$}
\centerline{\includegraphics[width=0.9\columnwidth]{\data/yfe2ge2/FiguresYFG/PhaseDiaAnnotPerSys.pdf}}
\begin{itemize}
\item
$V_0 ($YFe$_2$Ge$_2) = 164.8 \Ang^3$, $V_0 ($LuFe$_2$Ge$_2) =
159.3 \Ang^3$ 

\hfill $\implies \Delta V_0/V_0 \simeq 3.5 \%$
(corresponds to about 4 GPa).
%\item Require around 3 GPa to tune YFe$_2$Ge$_2$ to qcp.
\end{itemize}
\end{frame}



% %%%%%%%%%%%%%%%%%%%%%%%%%%%%%%%%%%%%%%%%%%%%%%%%%%%%%%%%%%%%%%%%%%%%%%
% \begin{frame}[label=YFG}
% % \column{0.36\textwidth}
% \begin{itemize}
% \item 
% Upper critical field gives short coherence length $\xi \simeq 120 \Ang$.

% \item
% High $C/T \implies$ $\xi \sim 166 \Ang$.

% \end{itemize}
% % \item<2->
% % \vspace{0.2em}
% % But lack of heat capacity anomaly in flux-grown 
% %   crystals  
% % {\small [Ran arXiv:1408.3319]}
% % \end{itemize}
% % \end{columns} 
% \end{frame}



%%%%%%%%%%%%%%%%%%%%%%%%%%%%%%%%%%%%%%%%%%%%%%%%%%%%%%%%%%%%%%%%%%%%%%
%\section{2nd generation}
%\subsection{Transport}
%%%%%%%%%%%%%%%%%%%%%%%%%%%%%%%%%%%%%%%%%%%%%%%%%%%%%%%%%%%%%%%%%%%%%%
\begin{frame}[label=YFGRes211]
\frametitle{Superconductivity and anomalous normal state in YFe$_2$Ge$_2$}
% \begin{columns}[t]
% \column{0.64\textwidth}
% \centerline{~}
\centerline{\includegraphics[width=0.9\columnwidth]{\data/yfe2ge2/FiguresYFG/Res/RF34Spring2015/Vx2upsm20T.pdf}}
%\begin{turn}{90} {\hspace{5em} \scriptsize[Chen PRL {\bf 116,} 127001 (2016)]} \end{turn}}

%\centerline{\scriptsize[Chen PRL {\bf 116,} 127001 (2016)]}

% \column{0.36\textwidth}
\begin{itemize}
\item
Sharp superconducting transition with $T_c =
1.83~\mathrm{K}$.

\item
 Anomalous normal state: $\rho(T) \sim \rho_0 + T^{3/2}$ %, \\
%High $C/T \sim 100~{\rm mJ/(molK^2)}$.

%\end{itemize}
% \item<2->
% %\vspace{0.2em}
% Lack of heat capacity anomaly in flux-grown 
%   crystals  \\
% {\scriptsize [Kim Phil. Mag. {\bf 95} 804 (2015)]}
\item<2->
Growth improvements (Fe-rich melt, annealing) $\rightarrow$ \\ bulk superconductivity and RRR $100-200$.
\end{itemize}
% \end{columns} 

% \visible<2->{
% \centerline{Konstantin Semeniuk \hl{P6.1 on Thursday afternoon}}}

\vspace*{\fill}
\centerline{\makebox[\linewidth]{\rule{0.85\textwidth}{0.4pt}}}
\centerline{\scriptsize Chen PRL {\bf 116,} 127001 (2016)}
\end{frame}



%\subsection{High-$p$ s/c in KFe$_2$As$_2$}
%%%%%%%%%%%%%%%%%%%%%%%%%%%%%%%%%%%%%%%%%%%%%%%%%%%%%%%%%%%%%%%%%%%%%%
\begin{frame}[label=CollapsedTetragonalSupercon]
\frametitle{Superconductivity in high pressure KFe$_2$As$_2$}

\begin{columns}[c]
\column{0.55\textwidth}
\centerline{~}
\centerline{\includegraphics[width=\columnwidth]{\data/yfe2ge2/FiguresYFG/KFAYingComposite}}
% \centerline{\small[Ying arXiv:1501.00330 (2015)]}
% \centerline{\small[Nakajima PRB {\bf 91,} 060508 (2015)]}

\column{0.44\textwidth}
\begin{itemize} \itemsep 7pt
\item
Superconductivity at $\sim 10~ \text{K}$ in high pressure
KFe$_2$As$_2$.

\item
Jump in $T_c$ coincides with transition to collapsed tetragonal
structure.

\item
Enhancement caused by appearance of electron pockets in
BZ corners. \\
% {\small[Guterding PRB {\bf 91,} 140503 (2015).]} 

\item
As and Ge p-states affect magnetism differently. 


\item
High $C/T \implies$ $\xi \sim 166 \Ang$.


\end{itemize}

%\vspace{1.5em}
\end{columns}
 

\vspace*{\fill}
\vspace{1.5em}
\centerline{\makebox[\linewidth]{\rule{0.85\textwidth}{0.4pt}}}
\centerline{\scriptsize Ying arXiv:1501.00330 (2015), Nakajima PRB
  {\bf 91,} 060508 (2015)}
\centerline{\scriptsize  Guterding PRB {\bf 91,} 140503 (2015), PRL
  {\bf 118,} 017204 (2017)}
\end{frame}





%\subsection{Multiband superconductivity?}
%%%%%%%%%%%%%%%%%%%%%%%%%%%%%%%%%%%%%%%%%%%%%%%%%%%%%%%%%%%%%%%%%%%%%%
\begin{frame}[label=YFGHeatCap3]
\frametitle{Multiband superconductivity in YFe$_2$Ge$_2$?}
\centerline{\includegraphics[width=\columnwidth]{\data/yfe2ge2/FiguresYFG/HeatCap/RF34B05/HCPlotNoInsetKFA.pdf}}

\begin{itemize}
\item 
Extrapolated residual $\gamma \simeq 40~{\rm mJ/(mol K^2)}$.

\item
True $C/T$ for $T \rightarrow 0$ may be lower -- see KFe$_2$As$_2$.

\end{itemize}

\end{frame}




%%%%%%%%%%%%%%%%%%%%%%%%%%%%%%%%%%%%%%%%%%%%%%%%%%%%%%%%%%%%%%%%%%%%%%
\begin{frame}[label=YFGIntro1]
\frametitle{Superconductivity on the threshold of antiferromagnetism}
\centerline{\includegraphics[width=\columnwidth]{\Figures/PhaseDias/CPSFeAsCompare-crop2}}

\begin{itemize}
% \item
% ThCr$_2$Si$_2$ structure.

\item
Search for analogues to Fe-As superconductors.

\item
KFe$_2$As$_2$ ($T_c \simeq 3.8 ~\text{K}$): high $C/T \simeq 100
~\text{mJ/mol K}^{2}$.

\item
YFe$_2$Ge$_2$ has similarly high $C/T$, apparently same Fe oxidation
number as in
KFe$_2$As$_2$.

\end{itemize}

\vspace*{\fill}
%\vspace{3.5em}
\centerline{\makebox[\linewidth]{\rule{0.85\textwidth}{0.4pt}}}
\centerline{\scriptsize Mathur Nature {\bf 394,} 39 (1998), Nandi PRL
  {\bf 104,} 057006 (2010)}
\centerline{\scriptsize Avila JMMM {\bf 270,} 51 (2004)}
\end{frame}

%%%%%%%%%%%%%%%%%%%%%%%%%%%%%%%%%%%%%%%%%%%%%%%%%%%%%%%%%%%%%%%%%%%%%%
\begin{frame}[label=YFGIntro2]
\frametitle{Magnetism and enhanced $C/T$ in  (Lu/Y)Fe$_2$Ge$_2$}
\centerline{\includegraphics[width=\columnwidth]{\data/yfe2ge2/FiguresYFG/YFGLFGCompare-crop.pdf}}
\begin{itemize} \itemsep 8pt
\item
Unusually high $\gamma = C/T$ of about 100 mJ/mol K$^2$.
\item
$V_0 ($YFe$_2$Ge$_2) = 164.8 \Ang^3$, $V_0 ($LuFe$_2$Ge$_2) =
159.3 \Ang^3$ 

\hfill $\implies \Delta V_0/V_0 \simeq 3.5 \%$
(corresponds to about 4 GPa).
%\item Require around 3 GPa to tune YFe$_2$Ge$_2$ to qcp.

\item
Magnetic susceptibility $\chi_{SI} \sim 10^{-3}$. \\ Moderately enhanced Wilson ratio $R_W \simeq 2.5$.

\item
Fe oxidation state appears same as in KFe$_2$As$_2$ ($+2.5$).

\end{itemize}
\end{frame}



%%%%%%%%%%%%%%%%%%%%%%%%%%%%%%%%%%%%%%%%%%%%%%%%%%%%%%%%%%%%%%%%%%%%%%
\begin{frame}[label=KFAYFGComparison1]
\frametitle{YFe$_2$Ge$_2$ vs. KFe$_2$As$_2$: role of X-X bonds}

\centerline{\includegraphics[width=0.95\columnwidth]{\data/yfe2ge2/FiguresYFG/FermiSurface/CompareKFA-YFG1}}

%\vspace{1em}

\begin{center}
\begin{tabular}{l|c|c}
RFe$_2$X$_2$ & $c/a$ & X-X dist. (\AA) \\
\hline
YFe$_2$Ge$_2$ & 2.638 & 2.721 \\
KFe$_2$As$_2$ & 3.608 & 4.089  \\
\hline
\end{tabular}
\end{center}
%\centerline{\small YFe$_2$Ge$_2$ Theorie: [Singh PRB {\bf 89,} 024505, Subedi PRB {\bf
%    89,} 024504 (2014)]}


\end{frame}


\subsection{Growth improvements}
%%%%%%%%%%%%%%%%%%%%%%%%%%%%%%%%%%%%%%%%%%%%%%%%%%%%%%%%%%%%%%%%%%%%%%
\begin{frame}[label=YFGSampleGrowth]
\frametitle{YFe$_2$Ge$_2$ optimising sample growth}
\centerline{\includegraphics[width=0.76\columnwidth]{\data/yfe2ge2/FiguresYFG/Overview/RRR-Tc2-mod}}

\begin{itemize}
\item 
Induction furnace. Prereaction. Annealing
1h at 1250$^\circ$, \\ 1 week at 800$^\circ$. 

\item
From RRR > 70, observe sharp resistive transitions and superconducting
heat capacity anomalies.

\end{itemize}

\end{frame}

%%%%%%%%%%%%%%%%%%%%%%%%%%%%%%%%%%%%%%%%%%%%%%%%%%%%%%%%%%%%%%%%%%%%%%
\begin{frame}[label=CompositionOverview]
\frametitle{YFe$_2$Ge$_2$ annealing and composition tuning}
\includegraphics[width=\textwidth]{\data/yfe2ge2/FiguresYFG/SampleSummaries/RRRsTcsSummary.pdf}
\vspace{1em}
\begin{itemize}
\item
Fe-rich nominal composition crucial for high
RRR and bulk superconductivity.
\item
Similarities with complexity encountered in CeCu$_2$Si$_2$.
\end{itemize}

\end{frame}


%%%%%%%%%%%%%%%%%%%%%%%%%%%%%%%%%%%%%%%%%%%%%%%%%%%%%%%%%%%%%%%%%%%%%%
\subsection{Heat capacity}
%%%%%%%%%%%%%%%%%%%%%%%%%%%%%%%%%%%%%%%%%%%%%%%%%%%%%%%%%%%%%%%%%%%%%%
\begin{frame}[label=YFGHeatCap2]
\frametitle{YFe$_2$Ge$_2$ superconducting heat capacity anomaly}
\centerline{\includegraphics[width=0.86\columnwidth]{\data/yfe2ge2/FiguresYFG/HeatCap/RF34B05/HCPlotInsetMag2}}




\begin{itemize}
% \item 
% Improved resistivity ratio (RRR $> 100$).
\item
Residual $\gamma < 50~{\rm mJ/(mol K^2)}$:
Alien phase unlikely.

\item 
Upper critical field gives short coherence length $\xi \simeq 120
\Ang$.
\end{itemize}
\end{frame}

%%%%%%%%%%%%%%%%%%%%%%%%%%%%%%%%%%%%%%%%%%%%%%%%%%%%%%%%%%%%%%%%%%%%%%
%\subsection{$c$-axis contraction}
%%%%%%%%%%%%%%%%%%%%%%%%%%%%%%%%%%%%%%%%%%%%%%%%%%%%%%%%%%%%%%%%%%%%%%
\begin{frame}[label=CollapsedTetragonal]
\frametitle{KFe$_2$As$_2$ pressure-induced $c$-axis collapse}

\centerline{\includegraphics[width=0.65\columnwidth]{\data/yfe2ge2/FiguresYFG/CollapsedTetragonal}}

\begin{itemize}
\item 
c-axis contraction: As-As dimerisation \\
{\small [e.g. Hoffmann J. Phys. Chem. 1985].} \\
As-As distance = $4.09 \Ang$ $\rightarrow$ $2.54
\Ang$. \\
 Covalent bond distance is $2.46 \Ang$

\item
Bond formation releases two electrons per f.u. $\rightarrow$ changes band
filling as well as $c$-axis hopping.
% YFe$_2$Ge$_2$ Fermi surface resembles that of
%   KFe$_2$As$_2$ in \hl{collapsed tetragonal} phase.
\end{itemize}

\end{frame}



\subsection{YFe$_2$Ge$_2$ growth}
%%%%%%%%%%%%%%%%%%%%%%%%%%%%%%%%%%%%%%%%%%%%%%%%%%%%%%%%%%%%%%%%%%%%%%
\begin{frame}[label=YFGPrep]
\frametitle{YFe$_2$Ge$_2$ growth and characterisation}

\centerline{\includegraphics[width=\columnwidth]{\data/yfe2ge2/FiguresYFG/YFGPrepare-crop.pdf}}
\begin{itemize}
\item
Flux growth (with Sn) or radio-frequency induction melting. 
\item
Resistance ratios $\sim 50$ after annealing at $> 800 ~^\circ$C. Can reach
RRR $>100$.
\item 
EDX and x-ray; occasional $\alpha$-Fe inclusions $\sim 1\%$.
\end{itemize}
\end{frame}


%%%%%%%%%%%%%%%%%%%%%%%%%%%%%%%%%%%%%%%%%%%%%%%%%%%%%%%%%%%%%%%%%%%%%%
\subsection{Motivation}

%%%%%%%%%%%%%%%%%%%%%%%%%%%%%%%%%%%%%%%%%%%%%%%%%%%%%%%%%%%%%%%%%%%%%%
%\subsection{The (Lu/Y)Fe$_2$Ge$_2$ system}
%%%%%%%%%%%%%%%%%%%%%%%%%%%%%%%%%%%%%%%%%%%%%%%%%%%%%%%%%%%%%%%%%%%%%%



%%%%%%%%%%%%%%%%%%%%%%%%%%%%%%%%%%%%%%%%%%%%%%%%%%%%%%%%%%%%%%%%%%%%%%
\subsection{Evidence for superconductivity}
%%%%%%%%%%%%%%%%%%%%%%%%%%%%%%%%%%%%%%%%%%%%%%%%%%%%%%%%%%%%%%%%%%%%%%
\begin{frame}[label=YFGSupercon]
\frametitle{YFe$_2$Ge$_2$ superconductivity -- first
  generation samples}

\centerline{\includegraphics[width=0.9\columnwidth]{\data/yfe2ge2/FiguresYFG/YFGSuperconMag-crop.pdf}}
\centerline{\small [Zou, Physica Status Solidi (RRL) {\bf 8,} 928
  (2014)]}
\begin{itemize}
\item
Superconducting below $\simeq 1.5~\rm K$. $T_c$ and jump height depend on RRR.
\item
Large volume fractions observed in zfc magnetometry.
\end{itemize}
\end{frame}


%%%%%%%%%%%%%%%%%%%%%%%%%%%%%%%%%%%%%%%%%%%%%%%%%%%%%%%%%%%%%%%%%%%%%%
\begin{frame}[label=YFGCritField211]
\frametitle{Upper critical field and coherence length}
% \begin{columns}[t]
% \column{0.64\textwidth}
% \centerline{~}
\centerline{\includegraphics[width=0.9\columnwidth]{\data/yfe2ge2/FiguresYFG/Res/RF34Spring2015/YFGResHc2InsetLine.pdf}}

\centerline{\small[Chen PRL {\bf 116,} 127001 (2016)]}

\end{frame}

%%%%%%%%%%%%%%%%%%%%%%%%%%%%%%%%%%%%%%%%%%%%%%%%%%%%%%%%%%%%%%%%%%%%%%
%\subsection{The (Lu/Y)Fe$_2$Ge$_2$ system}
%%%%%%%%%%%%%%%%%%%%%%%%%%%%%%%%%%%%%%%%%%%%%%%%%%%%%%%%%%%%%%%%%%%%%%
\begin{frame}[label=YFGMagn]
\frametitle{YFe$_2$Ge$_2$ magnetic properties}

\centerline{\includegraphics[width=\columnwidth]{\data/yfe2ge2/FiguresYFG/MagNormal.pdf}}
\begin{itemize}
\item
Magnetisation consistent with literature reports: \\ moderately enhanced paramagnet.
\item
Weak ferromagnetic signature indicates small amount of iron-rich alien phase.
\end{itemize}
\end{frame}





%%%%%%%%%%%%%%%%%%%%%%%%%%%%%%%%%%%%%%%%%%%%%%%%%%%%%%%%%%%%%%%%%%%%%%
%\subsection{The (Lu/Y)Fe$_2$Ge$_2$ system}
%%%%%%%%%%%%%%%%%%%%%%%%%%%%%%%%%%%%%%%%%%%%%%%%%%%%%%%%%%%%%%%%%%%%%%
\begin{frame}[label=YFGCritField]
\frametitle{YFe$_2$Ge$_2$ resistive upper critical field $H_{c2}$}
\begin{columns}[t]
\column{0.64\textwidth}
\centerline{~}
\centerline{\includegraphics[width=\columnwidth]{\data/yfe2ge2/FiguresYFG/YFGCritField2.pdf}}

\column{0.36\textwidth}
\begin{itemize}
\item \itemsep 20pt
$B_{c2}$ gives $\xi \simeq 105 \Ang$.

\item
High $C/T \implies$ expect $\xi \sim 130 \Ang$.

\item
Anomalous normal state: \\$\rho(T) \sim \rho_0 + T^{3/2}$

\item<2->
\vspace{0.2em}
But lack of heat capacity anomaly in flux-grown 
  crystals  
{\small [Ran arXiv:1408.3319]}
\end{itemize}
\end{columns} 
\end{frame}



%%%%%%%%%%%%%%%%%%%%%%%%%%%%%%%%%%%%%%%%%%%%%%%%%%%%%%%%%%%%%%%%%%%%%%
%\subsection{The (Lu/Y)Fe$_2$Ge$_2$ system}
%%%%%%%%%%%%%%%%%%%%%%%%%%%%%%%%%%%%%%%%%%%%%%%%%%%%%%%%%%%%%%%%%%%%%%
\begin{frame}[label=YFGHeatCap1]
\frametitle{YFe$_2$Ge$_2$ superconducting heat capacity anomaly in a
  different batch, isotropic gap}
\centerline{\includegraphics[width=0.86\columnwidth]{\data/yfe2ge2/FiguresYFG/HeatCap/RF32A03/YFGRF32A03B}}

\begin{itemize}
\item 
Improved resistivity ratio (RRR $\simeq 100$).
%\item
Residual $\gamma \simeq 50~{\rm mJ/(mol K^2)}$.


\end{itemize}

\end{frame}



%%%%%%%%%%%%%%%%%%%%%%%%%%%%%%%%%%%%%%%%%%%%%%%%%%%%%%%%%%%%%%%%%%%%%%
\subsection{Low-$T$ extrapolation}
%%%%%%%%%%%%%%%%%%%%%%%%%%%%%%%%%%%%%%%%%%%%%%%%%%%%%%%%%%%%%%%%%%%%%%
\begin{frame}[label=YFGHeatCap3]
\frametitle{YFe$_2$Ge$_2$ superconducting heat capacity anomaly -- 2}
\centerline{\includegraphics[width=0.86\columnwidth]{\data/yfe2ge2/FiguresYFG/HeatCap/RF34B05/HCComparison}}

\begin{center}
\begin{tabular}{l|c}
Extrapolation scheme & normal fraction \\
\hline
Linear (line nodes) & $24 \%$ \\
Quadratic (point nodes) & $42 \%$  \\
BCS (isotropic) & $53 \%$ \\
\hline
\end{tabular}
\end{center}

\begin{itemize}
\item Rise starts at resistive $T_c$, but main anomaly is at $\sim
  1~\mathrm{K}$.
\end{itemize}
\end{frame}


%%%%%%%%%%%%%%%%%%%%%%%%%%%%%%%%%%%%%%%%%%%%%%%%%%%%%%%%%%%%%%%%%%%%%%
\begin{frame}[label=YFGHeatCap]
\frametitle{YFe$_2$Ge$_2$ superconducting heat capacity anomaly in a
 different batch}
\centerline{\includegraphics[width=0.86\columnwidth]{\data/yfe2ge2/FiguresYFG/HeatCap/RF32A03/HCPlot}}

\begin{itemize}
\item 
Improved resistivity ratio (RRR $\simeq 100$).
%\item
Residual $\gamma \simeq 50~{\rm mJ/(mol K^2)}$.


\end{itemize}

\end{frame}




%%%%%%%%%%%%%%%%%%%%%%%%%%%%%%%%%%%%%%%%%%%%%%%%%%%%%%%%%%%%%%%%%%%%%%
\section{Search for a qcp}
%\subsection{The (Lu/Y)Fe$_2$Ge$_2$ system}
%%%%%%%%%%%%%%%%%%%%%%%%%%%%%%%%%%%%%%%%%%%%%%%%%%%%%%%%%%%%%%%%%%%%%%
\begin{frame}[label=YFGIntro2b]
\frametitle{Approach magnetic quantum phase transition in
  (Lu/Y)Fe$_2$Ge$_2$ by applied pressure?}
\centerline{\includegraphics[width=0.9\columnwidth]{\data/yfe2ge2/FiguresYFG/PhaseDiaAnnotPerSys.pdf}}

\begin{itemize}
\item
$V_0 ($YFe$_2$Ge$_2) = 164.8 \Ang^3$, $V_0 ($LuFe$_2$Ge$_2) =
159.3 \Ang^3$ 

\hfill $\implies \Delta V_0/V_0 \simeq 3.5 \%$
(corresponds to about 4 GPa).

\item Require around 3 GPa to tune YFe$_2$Ge$_2$ to qcp.
\end{itemize}
\end{frame}



\subsection{High $p$}
%%%%%%%%%%%%%%%%%%%%%%%%%%%%%%%%%%%%%%%%%%%%%%%%%%%%%%%%%%%%%%%%%%%%%%
\begin{frame}[label=YFGHighP]
\frametitle{YFe$_2$Ge$_2$ superconductivity at high pressure}
\centerline{\includegraphics[width=0.86\columnwidth]{\data/yfe2ge2/FiguresYFG/HighP/YFGPressureResCurve.pdf}}

\centerline{\small[Konstantin Semeniuk, Poster Mo. A-P70]}

\begin{itemize}
\item 
$T_c$ rises with pressure by about $15 \%$, then flattens
off. Transition becomes sharper.

\item
Normal state $\rho(T)$ slope decreases with pressure.

\end{itemize}

\end{frame}


%%%%%%%%%%%%%%%%%%%%%%%%%%%%%%%%%%%%%%%%%%%%%%%%%%%%%%%%%%%%%%%%%%%%%%
%\subsection{The (Lu/Y)Fe$_2$Ge$_2$ system}
%%%%%%%%%%%%%%%%%%%%%%%%%%%%%%%%%%%%%%%%%%%%%%%%%%%%%%%%%%%%%%%%%%%%%%
\begin{frame}[label=CollapsedTetragonal1]
\frametitle{KFe$_2$As$_2$ c-axis collapse}

\centerline{\includegraphics[width=0.75\columnwidth]{\data/yfe2ge2/FiguresYFG/CollapsedTetragonal}}

\begin{itemize}
\item
YFe$_2$Ge$_2$ Fermi surface resembles that of
  KFe$_2$As$_2$ in \hl{collapsed tetragonal} phase.
\end{itemize}

\end{frame}

%%%%%%%%%%%%%%%%%%%%%%%%%%%%%%%%%%%%%%%%%%%%%%%%%%%%%%%%%%%%%%%%%%%%%%
\begin{frame}[plain,label=YFGCounting]
\frametitle{Resolving the Fermi surface in YFe$_2$Ge$_2$}
%\frametitle{Electron counting in YFe$_2$Ge$_2$}
\includegraphics[width=1.2\textwidth]{\data/yfe2ge2/FiguresYFG/QO/ElectronCounting}
\end{frame}

%%%%%%%%%%%%%%%%%%%%%%%%%%%%%%%%%%%%%%%%%%%%%%%%%%%%%%%%%%%%%%%%%%%%%%
\begin{frame}[plain,label=YFGMass]
\frametitle{Mass study in YFe$_2$Ge$_2$}
\includegraphics[width=1.2\textwidth]{\data/yfe2ge2/FiguresYFG/QO/MassStudy}
\end{frame}



%%%%%%%%%%%%%%%%%%%%%%%%%%%%%%%%%%%%%%%%%%%%%%%%%%%%%%%%%%%%%%%%%%%%%%
\begin{frame}[plain,label=YFGCounting]
\frametitle{Electron counting in YFe$_2$Ge$_2$}
\includegraphics[width=1.2\textwidth]{\data/yfe2ge2/FiguresYFG/QO/ElectronCounting}
\end{frame}


%%%%%%%%%%%%%%%%%%%%%%%%%%%%%%%%%%%%%%%%%%%%%%%%%%%%%%%%%%%%%%%%%%%%%%
\begin{frame}[label=YFGEPocket]
\frametitle{Finding the electron pocket in YFe$_2$Ge$_2$}
\includegraphics[width=\textwidth]{\data/yfe2ge2/FiguresYFG/QO/HFMLQOPlotFig}

\begin{itemize}
\item
Nijmegen 10/'20 and 6/'21 torque magnetometry in dilution fridge, 38 T system
\item
Side-lobes, caused by nonlinear torque interaction
\item
Demonstrate low frequency $\delta \simeq \SI{400}{\tesla} \rightarrow$  electron pocket D
\end{itemize}
\end{frame}

%%%%%%%%%%%%%%%%%%%%%%%%%%%%%%%%%%%%%%%%%%%%%%%%%%%%%%%%%%%%%%%%%%%%%%
\begin{frame}[plain,label=YFGEPocket2]
\frametitle{Electron pocket in YFe$_2$Ge$_2$}
\includegraphics[width=1.2\textwidth]{\data/yfe2ge2/FiguresYFG/QO/HFMLDeltaFreq3Fig}
\begin{itemize}
\item
Side-lobes (empty circles) and low-frequency oscillations (full circles) suggest \hl{cylindrical shape; high mass} % $\sim 10 m_e$ consistent with prior expectations. 
\item
Electron pocket duckbill outgrowth, sensitive to band filling or structural details, \hl{large variation in DOS} contribution
\end{itemize}

\end{frame}



%%%%%%%%%%%%%%%%%%%%%%%%%%%%%%%%%%%%%%%%%%%%%%%%%%%%%%%%%%%%%%%%%%%%%%
%\subsection{X-X dimerisation}


%%%%%%%%%%%%%%%%%%%%%%%%%%%%%%%%%%%%%%%%%%%%%%%%%%%%%%%%%%%%%%%%%%%%%%
%\section{Summary}
%%%%%%%%%%%%%%%%%%%%%%%%%%%%%%%%%%%%%%%%%%%%%%%%%%%%%%%%%%%%%%%%%%%%%%
%\begin{frame}[label=Summary]
%\frametitle{Superconducting YFe$_2$Ge$_2$ -- reference compound for
%  high pressure collapsed
%  tetragonal KFe$_2$As$_2$}
%
%\centerline{\includegraphics[width=\columnwidth]{Summary}}
%
%
%
%\end{frame}


%%%%%%%%%%%%%%%%%%%%%%%%%%%%%%%%%%%%%%%%%%%%%%%%%%%%%%%%%%%%%%%%%%%%%%
%\subsection{YFG Summary}
%%%%%%%%%%%%%%%%%%%%%%%%%%%%%%%%%%%%%%%%%%%%%%%%%%%%%%%%%%%%%%%%%%%%%%
\begin{frame}[plain,label=YFGSummary]
\frametitle{Summary YFe$_2$Ge$_2$}

\includegraphics[width=1.2\columnwidth]{\data/yfe2ge2/FiguresYFG/Overview/Summary2}
%
%% \centerline{New crystals enable detailed study of superconducting and normal state.}
%\vspace{1.5em}
%\begin{itemize}
%\item
%Origin of high heat capacity?%: consistent with magnetic fluctuation spectrum?
%% \item
%% Do quasiparticle masses agree with heat capacity?
%\item
%Superconductivity: residual $\gamma$ or tiny second gap? Nodes?
%\item
%Origin of anomalous $T^{3/2}$ normal state resistivity?
%\item
%\hl{Bulk properties as in KFe$_2$As$_2$ -- }\\ 
%\hfill \raggedleft {\hl{but different Fermi surface.}}
%\end{itemize}
\end{frame}



