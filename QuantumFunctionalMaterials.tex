%%%%%%%%%%%%%%%%%%%%%%%%%%%%%%%%%%%%%%%%%%%%%%%%%%%%%%%%%%%%%%%%%%%%%%
% \begin{frame}[plain,label=JJT]
% \frametitle{Quantum Functional Materials}

% \visible<2->{
% \begin{beamercolorbox}{postit}
% \parbox[t][3em][c]{\textwidth}{\centerline {Any new discovery contains the germ of a new industry.}}

% \end{beamercolorbox}}

% \vspace{2em}

% \begin{floatingfigure}[r]{0.3\textwidth}
% \centerline{\includegraphics[width=0.25\textwidth]{\Figures/photos/Thomson/thomson_docu}}
% \end{floatingfigure}
% {\small \hl{``Consider, for example, the discovery of the electron. 
% Could anything at first sight be more impractical} than a body, which
% could only exist in vessels from which all but a minute fraction of
% the air has been extracted, which is so small that its mass is an
% insignificant fraction of the mass of an atom of hydrogen, which
% itself is so small that a crowd of these atoms equal in number to the
% population of the whole world would be too small to be detected by any
% means then known to science?}

% \vspace{1em}
% \begin{columns}[t]
% \column{\textwidth}
% The electron owes its practical utility to its smallness.''\\
% \vspace{1em}
% \raggedleft{\small \em J. J. Thomson}

% \end{columns}


% \end{frame}




%%%%%%%%%%%%%%%%%%%%%%%%%%%%%%%%%%%%%%%%%%%%%%%%%%%%%%%%%%%%%%%%%%%%%%
%\subsection{Instrumentation}
%%%%%%%%%%%%%%%%%%%%%%%%%%%%%%%%%%%%%%%%%%%%%%%%%%%%%%%%%%%%%%%%%%%%%%
\begin{frame}[label=QFM-Measurement]
\frametitle{Instrumentation development}
\centerline{\includegraphics[width=\columnwidth]{\Figures/GroupAdvertising/Instrumentation17}}
\end{frame}

%%%%%%%%%%%%%%%%%%%%%%%%%%%%%%%%%%%%%%%%%%%%%%%%%%%%%%%%%%%%%%%%%%%%%%
\subsection{Solid state refrigeration}
%%%%%%%%%%%%%%%%%%%%%%%%%%%%%%%%%%%%%%%%%%%%%%%%%%%%%%%%%%%%%%%%%%%%%%
\begin{frame}[label=QFM-CryoRef]
\frametitle{\small Cryogenic refrigeration -- correlated  
  magnetocalorics and thermoelectrics}
\centerline{\includegraphics[width=\columnwidth]{\Figures/SolidStateRefrig/RefrigSummary}}
\end{frame}

%%%%%%%%%%%%%%%%%%%%%%%%%%%%%%%%%%%%%%%%%%%%%%%%%%%%%%%%%%%%%%%%%%%%%%
\subsection{Routes to applications}
%%%%%%%%%%%%%%%%%%%%%%%%%%%%%%%%%%%%%%%%%%%%%%%%%%%%%%%%%%%%%%%%%%%%%%
\begin{frame}[label=QFM]
\frametitle{Quantum Functional Materials}
\visible<1->{
\begin{beamercolorbox}{postit}
\parbox[t][3em][c]{\textwidth}{\centerline {Any new discovery contains the germ of a new industry.}}
\end{beamercolorbox}}
\vspace{1em}
\raggedleft{\small \em J. J. Thomson}

\visible<2->{
\centerline{\hl {Apply quantum nature of correlated matter}}
\begin{itemize}
\item Quantum coherence -- protection
\item Tunability -- precise control
\item Spin, charge, orbital states -- multiple applications
\end{itemize}

\vspace{1em}

\centerline{\hl {Research directions}}
\begin{itemize}
\item High-temperature superconductors
\item Batteries and supercapacitors
\item Multiferroics, magnetic memories and information transport
\item Solid state refrigeration and thermal energy harvesting
\item Fault-tolerant quantum computing
\end{itemize}
}
\end{frame}


%%%%%%%%%%%%%%%%%%%%%%%%%%%%%%%%%%%%%%%%%%%%%%%%%%%%%%%%%%%%%%%%%%%%%%
\begin{frame}[label=QFM]
\frametitle{Quantum Functional Materials}
\centerline{\includegraphics[width=0.95\columnwidth]{\Figures/GroupAdvertising/QFM2}}


\vspace*{\fill}
\vspace{-1.5em}
\centerline{\makebox[\linewidth]{\rule{0.85\textwidth}{0.4pt}}}
\centerline{\scriptsize Tokura, Kawasaki, Nagaosa, Nature Physics (2017)}

\end{frame}


% %%%%%%%%%%%%%%%%%%%%%%%%%%%%%%%%%%%%%%%%%%%%%%%%%%%%%%%%%%%%%%%%%%%%%%
% \begin{frame}[label=QFM]
% \frametitle{Quantum Functional Materials -- materials discovery}
% \centerline{\includegraphics[width=0.95\columnwidth]{\Figures/GroupAdvertising/QFM}}
% \end{frame}





%%% Local Variables: 
%%% mode: latex
%%% TeX-master: "GroTalk"
%%% End: 
