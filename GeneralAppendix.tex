%%%%%%%%%%%%%%%%%%%%%%%%%%%%%%%%%%%%%%%%%%%%%%%%%%%%%%%%%%%%%%%%%%%%%
%\section{Quantum phase transitions}

\begin{frame}[label=ElecStates]
\frametitle{Electronic self-organisation}

\centerline{\multiinclude[<visible@1-|1->][format=pdf,graphics={width=0.8\columnwidth}]{\Figures/FreeElec/FreeElecLayers}}
\vspace{2ex}

\begin{itemize}
\item<visible@1-> Diversity of condensates\\ {\small (e.g. spin/charge
density wave, (spin-) Peierls, structural,
Pomeranchuk, metamagnetic, nematic, multipolar 
hidden order)}

\item<visible@1-> Electron-hole, electron-electron bound states: interacting-electron chemistry.

% \item<visible@4-> Purity needed to allow complex order.
\end{itemize}

\end{frame}

% %%%%%%%%%%%%%%%%%%%%%%%%%%%%%%%%%%%%%%%%%%%%%%%%%%%%%%%%%%%%%%%%%%%%%
\begin{frame}[label=Moredifferent]
\frametitle{More is different}

\begin{columns}[c]
\column{0.5\textwidth}
\centerline{\includegraphics[width=0.8\columnwidth]{\Figures/photos/PWA}}

\column{0.5\textwidth}
{\bf ...at each new level of complexity, entirely new properties
  appear, and the understanding of this behavior requires research
  which I think is as fundamental in its nature as any other.} \\

\centerline{\scriptsize [P. W. Anderson, More is different, 1972]}
\end{columns}
\vspace{2em}
%\pause
\begin{itemize}
\item
Self-organisation of \hl{atoms} leads to diversity of crystal structures \\

\item
\hl{Electrons in solids:} quantum liquids, form variety
of low temperature quantum protected states.

\end{itemize}
\end{frame}



%%%%%%%%%%%%%%%%%%%%%%%%%%%%%%%%%%%%%%%%%%%%%%%%%%%%%%%%%%%%%%%%%%%%%%
\begin{frame}[label=Condensates]
  \frametitle{Superconductivity: particle-particle condensate, \\ \hfill  Magnetism: particle-hole condensate}

\begin{columns}[c]
\column{0.7\textwidth}
\visible<1->{\centerline{Superconductivity}}
\multiinclude[<visible@+->][graphics={width=\columnwidth},format=pdf]{\Figures/FreeElec/Supercon}

\column{0.3\textwidth}
\visible<2->{
\centerline{$\langle c_{k\uparrow} c_{-k \downarrow} \rangle \neq 0$}
}
\end{columns}

\vspace{2em}

\begin{columns}[c]
\column{0.7\textwidth}
\visible<3->{\centerline{Magnetism}}
\multiinclude[<visible@+->][graphics={width=\columnwidth},format=pdf]{\Figures/FreeElec/Magnetism}

\column{0.3\textwidth}
\visible<7->{
\centerline{$\langle c^{\dagger}_{k\sigma_1} c_{k\sigma_2} \rangle \neq 0$
}}
\end{columns}

\end{frame}


%%%%%%%%%%%%%%%%%%%%%%%%%%%%%%%%%%%%%%%%%%%%%%%%%%%%%%%%%%%%%%%%%%%%%%
\begin{frame}[label=Skyrmions]
\frametitle{Skyrmion lattice: topological defects in MnSi}

\centerline{\includegraphics[height=0.35\columnwidth]{\Figures/Lectures/TopoStates/Skyrmions1}
\includegraphics[height=0.35\columnwidth]{\Figures/Lectures/TopoStates/Skyrmions2}}

\centerline{\small{[M\"uhlbauer {\em et al.} Science {\bf 323}, 915 (2009)}}

\begin{itemize}
\item
When investigating magnetic order in MnSi using neutron scattering, Pfleiderer \& Co. discovered \hl{lattice of vortex-like topological defects} of the magnetisation: the \hl{Skyrmion lattice}. 

%\item Within a few months, this has been observed in a number of \hl{other materials.}
% \item
% \hl{Chiral} magnet, at finite field and temperature: 'A-phase'.

\item
Skyrmions have been found to interact with electrical current $\implies$ 'skyrmionics'.

\end{itemize}
\end{frame}


%%%%%%%%%%%%%%%%%%%%%%%%%%%%%%%%%%%%%%%%%%%%%%%%%%%%%%%%%%%%%%%%%%%%%%
%\subsection{Quantum oscillations}
%%%%%%%%%%%%%%%%%%%%%%%%%%%%%%%%%%%%%%%%%%%%%%%%%%%%%%%%%%%%%%%%%%%%%%
\begin{frame}[label=QuantOsc]
\frametitle{Quantum oscillations}

\visible<1->{\centerline{\includegraphics[width=0.7\columnwidth]{\Figures/Lectures/QuantOsc/Sr2RuO4Osc}}}


\visible<2->{
\centerline{\includegraphics[width=0.7\textwidth]{\Figures/FermInstab/QuantOsc-Xray.pdf}}}

\end{frame}

%%%%%%%%%%%%%%%%%%%%%%%%%%%%%%%%%%%%%%%%%%%%%%%%%%%%%%%%%%%%%%%%%%%%%%
\begin{frame}[label=dHvA]
  \frametitle{Heavy quasiparticles detected in de Haas-Alphen experiments}
\vspace{-1em}
\begin{columns}[t]
\column{0.5\textwidth}
\centerline{~}
\centerline{\includegraphics[width=0.85\columnwidth]{\Figures/Lectures/HeavyFermion/UPt3FS}}
\centerline{Quantum oscillation results in UPt$_3$}
\begin{center} 
\scriptsize [Taillefer and Lonzarich, PRL {\bf 60} 1570 (1988)]
\end{center}

\column{0.5\textwidth}
\centerline{~}
\bi
\setlength{\itemsep}{1em}

\item
Magnetic susceptibility \hl{oscillates} as function of \hl{ magnetic field}

\item
Reveal several \hl{Fermi surface sheets}

\item
Temperature dependence of oscillation amplitude $\rightarrow$ \hl{effective mass}.

\item
Measured effective masses strongly renormalised, consistent with \hl{heat capacity results.}

\ei
\becbox{0.8}
f-electrons included in Fermi surface volume!
\encbox
\end{columns}

\end{frame}


%%%%%%%%%%%%%%%%%%%%%%%%%%%%%%%%%%%%%%%%%%%%%%%%%%%%%%%%%%%%%%%%%%%%%%
\begin{frame}[label=Skyrmions]
\frametitle{Skyrmion lattice: topological defects in MnSi}

\centerline{\includegraphics[height=0.35\columnwidth]{\Figures/Lectures/TopoStates/Skyrmions1}
\includegraphics[height=0.35\columnwidth]{\Figures/Lectures/TopoStates/Skyrmions2}}

\centerline{\small{[M\"uhlbauer {\em et al.} Science {\bf 323}, 915 (2009)}}

\begin{itemize}
\item
When investigating magnetic order in MnSi using neutron scattering, Pfleiderer \& Co. discovered \hl{lattice of vortex-like topological defects} of the magnetisation: the \hl{Skyrmion lattice}. 

%\item Within a few months, this has been observed in a number of \hl{other materials.}
% \item
% \hl{Chiral} magnet, at finite field and temperature: 'A-phase'.

\item
Skyrmions have been found to interact with electrical current $\implies$ 'skyrmionics'.

\end{itemize}
\end{frame}


%%%%%%%%%%%%%%%%%%%%%%%%%%%%%%%%%%%%%%%%%%%%%%%%%%%%%%%%%%%%%%%%%%%%%

\begin{frame}[label=Diversity]

\frametitle{Plenty of room in material space}
\only<beamer>{%
\only<1>{%
\vspace{2em}
\centerline{\includegraphics[angle=90,width=0.9\columnwidth]{\Figures/Diversity/PerSysen.jpg}}
}
}
\pause                          
%\centerline{\multiinclude[<visible@1->][graphics={width=0.9\columnwidth},format=pdf]{\Figures/Diversity/Diversity}}
\centerline{\multiinclude[<visible@+-| +->][graphics={width=0.9\columnwidth},format=pdf]{\Figures/Diversity/Diversity}}

\end{frame}



%%%%%%%%%%%%%%%%%%%%%%%%%%%%%%%%%%%%%%%%%%%%%%%%%%%%%%%%%%%%%%%%%%%%%%
\begin{frame}[label=SCES2011]
\frametitle{SCES 2011 in Cambridge}
\centerline{\includegraphics[width=0.8\columnwidth]{\Figures/GroupAdvertising/SCES2011a}}

\vspace{1em}
\centerline{International conference on Strongly Correlated Electron
  Systems, 2011}
\vspace{1em}
{\scriptsize \begin{itemize}
\item
657 participants from 23 countries. 127 talks, 736 posters in 40
sessions.
\item 
Also ran CDQM 2015 in Cambridge; involved in SCES, M2S, ICM, LT
conference series. 
\end{itemize} }
\end{frame}



%%%%%%%%%%%%%%%%%%%%%%%%%%%%%%%%%%%%%%%%%%%%%%%%%%%%%%%%%%%%%%%%%%%%%
%\section{Fermi surface instabilities}
%%%%%%%%%%%%%%%%%%%%%%%%%%%%%%%%%%%%%%%%%%%%%%%%%%%%%%%%%%%%%%%%%%%%%
\begin{frame}[label=FSInstabilities]
\frametitle{Fermi surface volume change in correlated electron
  systems}
\centerline{\multiinclude[<+- | visible@+->] [graphics={width=0.75\columnwidth},format=pdf]{\Figures/FermInstab/LargeSmall2}}
\begin{itemize}
\item
Large to small Fermi surface volume changes associated with density
wave or correlated insulator state.

\item<visible@2->
What about 4f-electron systems? Above $T^*$, expect localisation associated with
f-states.

\item<visible@2->
Can this be proven by quantum oscillations?

\end{itemize}
\end{frame}

%%%%%%%%%%%%%%%%%%%%%%%%%%%%%%%%%%%%%%%%%%%%%%%%%%%%%%%%%%%%%%%%%%%%%%
% \begin{frame}[label=Superconductivity1]
%   \frametitle{The first correlated-electron superconductor}
% \centerline{\includegraphics[width=\columnwidth]{\Figures/Lectures/HeavyFermion/steglich1}}
% \vspace{1em}
% \centerline{\small [Steglich {\em et al.}, Phys. Rev. Lett. {\bf 43} (1979) 1892]}
% \end{frame}

% %%%%%%%%%%%%%%%%%%%%%%%%%%%%%%%%%%%%%%%%%%%%%%%%%%%%%%%%%%%%%%%%%%%%%%
% \begin{frame}[label=Superconductivity2]
%   \frametitle{The first correlated-electron superconductor, a heavy fermion system}

% \begin{columns}[t]
% \column{0.55\textwidth}
% \centerline{~}
% \includegraphics[width=1.05\columnwidth]{\Figures/Lectures/HeavyFermion/steglichRho}
% \begin{itemize}
% \item \hl{Colossal heat capacity:} $C/T \simeq 1 ~\rm{J/(mol K^2)}$.
% \item Interpretation: heavy quasiparticles $\Rightarrow$ \hl{heavy fermions.}
% \end{itemize}


% \column{0.45\textwidth}
% \centerline{~}
% \includegraphics[width=\columnwidth]{\Figures/Lectures/HeavyFermion/steglichHeatCap}
% \end{columns}

% \vspace{1em}
% % \begin{itemize}
% % \item
% \centerline{$\Delta C \sim C \Rightarrow$ \hl{heavy fermions superconduct.}}
% \centerline{\scriptsize[Steglich {\em et al.}, Phys. Rev. Lett. {\bf 43} (1979) 1892]}

% % \end{itemize}

% \end{frame}

%%%%%%%%%%%%%%%%%%%%%%%%%%%%%%%%%%%%%%%%%%%%%%%%%%%%%%%%%%%%%%%%%%%%%%
\begin{frame}[label=ThreshMagn]
\frametitle{Example: superconductivity on the threshold of magnetism}
\begin{columns}[t]
\vspace{-4ex}
\column{0.5\textwidth}
\centerline{~}
% \visible<+->
{\includegraphics[width=0.9\columnwidth]{\Figures/cps/cpssuper}}

% \visible<+->
{\includegraphics[width=0.9\columnwidth]{\Figures/cein3/cein3super}\\
\centerline{\scriptsize{[N. D. Mathur, Nature {\bf 394}, 39 (1998)]
    % $\sim 1000$ citations
  }}}

\column{0.5\textwidth}
\centerline{~}
% \visible<+->
{\includegraphics[width=0.9\columnwidth]{\Figures/uge2/uge2super.jpg}\\
\centerline{\scriptsize{[S. S. Saxena, Nature {\bf 406} 587 (2000)]
    % $\sim 900$ citations
  }}}

\vspace{2ex}

% \visible<4->{
% \centerline{\includegraphics[width=0.85\columnwidth]{\Figures/Lectures/PistonCylinderCell/FThroughPhoto}}}

% \visible<4->
{
Further examples:

\hl{CeCu$_2$Si$_2$}, CeNi$_2$Ge$_2$, CeRh$_2$Si$_2$, 
Ce-1-1-5, Pu-1-1-5, URhGe, ...

}
\end{columns}
\end{frame}


%%%%%%%%%%%%%%%%%%%%%%%%%%%%%%%%%%%%%%%%%%%%%%%%%%%%%%%%%%%%%%%%%%%%%%

\begin{frame}[label=BorderFM2]
\frametitle{Quantum criticality on the border of ferromagnetism}

\vspace{-4ex}

\begin{columns}[t]
  
  \column{0.33\textwidth}
    \centerline{~}
    \visible<1->{\includegraphics[width=\columnwidth]{\Figures/Lectures/BorderFM/FM0}}

  \column{0.66\textwidth}
  \begin{columns}[t]
    \column{0.49\textwidth}
      \centerline{~}
      \visible<2->{\includegraphics[width=\columnwidth]{\Figures/Lectures/BorderFM/pfleiderer97-1.pdf}}
      \column{0.49\textwidth}
      \centerline{~}
      \visible<3->{\includegraphics[width=\columnwidth]{\Figures/Lectures/BorderFM/pfleiderer97-2.pdf}}
    \end{columns}
    \visible<2->{\centerline{\small[MnSi: Pfleiderer {\it et al.~} PRB {\bf 55} 8330 (1997)]}}
\end{columns}

\begin{minipage}{\textwidth}
  \begin{columns}[t]
    \column{0.33\textwidth}
    \begin{itemize}
    \item<1->
      Pressure \hl{tunes transition} temperature.
    \end{itemize}
    \column{0.33\textwidth}
    \bi
    \item<2-> Ferro\-mag\-netism \hl{disappears.}
    \ei
    \column{0.33\textwidth}
    \bi
    \item<3-> Scattering cross-section \hl{diverges.}
    \ei
    
  \end{columns}
\end{minipage}

\begin{itemize}
\item<3->
$T-$ dependence of the resistivity ($\propto T^{3/2}$) violates Fermi liquid theory.
\end{itemize}
\end{frame}

% %%%%%%%%%%%%%%%%%%%%%%%%%%%%%%%%%%%%%%%%%%%%%%%%%%%%%%%%%%%%%%%%%%%%%%
% \subsection{MnSi: first order transitions and anomalous power laws}
% %%%%%%%%%%%%%%%%%%%%%%%%%%%%%%%%%%%%%%%%%%%%%%%%%%%%%%%%%%%%%%%%%%%%%%

% \begin{frame}[label=BorderFM2]
% \frametitle{Quantum criticality on the border of ferromagnetism: MnSi}

% \vspace{-4ex}

% \begin{columns}[t]
  
%   \column{0.33\textwidth}
%     \centerline{~}
%     \visible<1->{\includegraphics[width=\columnwidth]{\Figures/BorderFM/FM1}}

%   \column{0.66\textwidth}
%   \begin{columns}[t]
%     \column{0.49\textwidth}
%       \centerline{~}
%       \visible<2->{\includegraphics[width=\columnwidth]{\Figures/BorderFM/pfleiderer97-1.pdf}}
%       \column{0.49\textwidth}
%       \centerline{~}
%       \visible<3->{\includegraphics[width=\columnwidth]{\Figures/BorderFM/pfleiderer97-2.pdf}}
%     \end{columns}
%     \visible<2->{\centerline{\tiny[MnSi: Pfleiderer {\it et al.~} PRB {\bf 55} 8330 (1997)]}}
% \end{columns}

% \begin{minipage}{\textwidth}
%   \begin{columns}[t]
%     \column{0.33\textwidth}
%     \begin{itemize}
%     \item<1->
%       Pressure \hl{tunes transition} temperature.
%     \end{itemize}
%     \column{0.33\textwidth}
%     \bi
%     \item<2-> Ferro\-mag\-netism \hl{disappears.}
%     \ei
%     \column{0.33\textwidth}
%     \bi
%     \item<3-> Scattering cross-section \hl{diverges.}
%     \ei
    
%   \end{columns}
% \end{minipage}

% \end{frame}


% %%%%%%%%%%%%%%%%%%%%%%%%%%%%%%%%%%%%%%%%%%%%%%%%%%%%%%%%%%%%%%%%%%%%%%
% \begin{frame}[label=MagnInter]
% \frametitle{Magnetic interaction in metals on the threshold of magnetism}

% \begin{columns}[t]
% \column{0.5\textwidth}
% \centerline{~}
% \centerline{\includegraphics[width=0.8\columnwidth]{\Figures/CritConcepts/MagnInter1}}
% %\onslide<2->
% \begin{itemize}
% \item%<2->
% \hl{Magnetic moment} of quasiparticle 1 creates \hl{exchange field $h_1$}.

% \item%<3->
% Exchange field $h_1$ leads to \hl{magnetisation} $m \propto \chi$. 

% \item%<4->
% \hl{Quasiparticle 2} is affected by resulting exchange interaction.
% \end{itemize}

% \column{0.5\textwidth}
% \centerline{}
% %\onslide<5->
% %% \begin{tabular}[t]{ll}
% %% Effect. inter$^n$ & $V(\mu_1, \mu_2) = -\lambda^2 \chi_{q\omega} \vec\mu_1 \cdot \vec\mu_2$ \tabularnewline
% %%  & \tabularnewline
% %% Susceptibility & $\chi_{\vec q \omega}^{-1} = \chi_\vec q^{-1} \left(1-i\omega/\Gamma_\vec q\right) $ \tabularnewline
% %% \end{tabular}
% \[V(\mu_1, \mu_2) = -\lambda^2 \chi_{q\omega} \vec\mu_1 \cdot \vec\mu_2\]
% \[\chi_{\vec q \omega}^{-1} = \chi_\vec q^{-1} \left(1-i\omega/\Gamma_\vec q\right) \]

% %\visible<6->
% {\includegraphics[width=\columnwidth]{\Figures/CritConcepts/FMwave} }
% \end{columns}

% %\onslide<6->
% {\centerline{\parbox{8cm}{\small [Interaction potential due to \\ moving quasiparticle in an incipient ferromagnet]}}}
% \end{frame}

%%%%%%%%%%%%%%%%%%%%%%%%%%%%%%%%%%%%%%%%%%%%%%%%%%%%%%%%%%%%%%%%%%%%%%
% \begin{frame}[label=ThreshMagn]
% \frametitle{Superconductivity wide-spread on the threshold of magnetism}
% \begin{columns}[t]
% \vspace{-4ex}
% \column{0.5\textwidth}
% \centerline{~}
% \visible<+->{\includegraphics[width=0.9\columnwidth]{\Figures/cps/cpssuper}}

% \visible<+->{\includegraphics[width=0.9\columnwidth]{\Figures/cein3/cein3super}\\
% \centerline{\scriptsize{[N. D. Mathur, Nature {\bf 394}, 39 (1998)]}}}

% \column{0.5\textwidth}
% \centerline{~}
% \visible<+->{\includegraphics[width=0.9\columnwidth]{\Figures/uge2/uge2super.jpg}\\
% \centerline{\scriptsize{[S. S. Saxena, Nature {\bf 406} 587 (2000)]}}}

% \vspace{2ex}

% % \visible<4->{
% % \centerline{\includegraphics[width=0.85\columnwidth]{\Figures/Lectures/PistonCylinderCell/FThroughPhoto}}}

% \visible<4->{
% Further examples:

% \hl{CeCu$_2$Si$_2$}, CeNi$_2$Ge$_2$, CeRh$_2$Si$_2$, CeCu$_2$,
% CeCu$_5$Au, CeIrIn$_5$, URhGe, ...

% }
% \end{columns}
% \end{frame}




